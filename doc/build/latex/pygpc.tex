%% Generated by Sphinx.
\def\sphinxdocclass{report}
\documentclass[letterpaper,10pt,english,openany,oneside]{sphinxmanual}
\ifdefined\pdfpxdimen
   \let\sphinxpxdimen\pdfpxdimen\else\newdimen\sphinxpxdimen
\fi \sphinxpxdimen=.75bp\relax

\PassOptionsToPackage{warn}{textcomp}
\usepackage[utf8]{inputenc}
\ifdefined\DeclareUnicodeCharacter
% support both utf8 and utf8x syntaxes
\edef\sphinxdqmaybe{\ifdefined\DeclareUnicodeCharacterAsOptional\string"\fi}
  \DeclareUnicodeCharacter{\sphinxdqmaybe00A0}{\nobreakspace}
  \DeclareUnicodeCharacter{\sphinxdqmaybe2500}{\sphinxunichar{2500}}
  \DeclareUnicodeCharacter{\sphinxdqmaybe2502}{\sphinxunichar{2502}}
  \DeclareUnicodeCharacter{\sphinxdqmaybe2514}{\sphinxunichar{2514}}
  \DeclareUnicodeCharacter{\sphinxdqmaybe251C}{\sphinxunichar{251C}}
  \DeclareUnicodeCharacter{\sphinxdqmaybe2572}{\textbackslash}
\fi
\usepackage{cmap}
\usepackage[T1]{fontenc}
\usepackage{amsmath,amssymb,amstext}
\usepackage[shorthands=off]{babel}
\usepackage{times}
\usepackage[Bjarne]{fncychap}
\usepackage{sphinx}

\fvset{fontsize=\small}
\usepackage{geometry}

% Include hyperref last.
\usepackage{hyperref}
% Fix anchor placement for figures with captions.
\usepackage{hypcap}% it must be loaded after hyperref.
% Set up styles of URL: it should be placed after hyperref.
\urlstyle{same}
\addto\captionsenglish{\renewcommand{\contentsname}{Contents:}}

\addto\captionsenglish{\renewcommand{\figurename}{Fig.}}
\addto\captionsenglish{\renewcommand{\tablename}{Table}}
\addto\captionsenglish{\renewcommand{\literalblockname}{Listing}}

\addto\captionsenglish{\renewcommand{\literalblockcontinuedname}{continued from previous page}}
\addto\captionsenglish{\renewcommand{\literalblockcontinuesname}{continues on next page}}
\addto\captionsenglish{\renewcommand{\sphinxnonalphabeticalgroupname}{Non-alphabetical}}
\addto\captionsenglish{\renewcommand{\sphinxsymbolsname}{Symbols}}
\addto\captionsenglish{\renewcommand{\sphinxnumbersname}{Numbers}}

\addto\extrasenglish{\def\pageautorefname{page}}

\setcounter{tocdepth}{1}



\title{Documentation of the pygpc package}
\date{Oct 04, 2018}
\release{2018}
\author{Konstantin Weise}
\newcommand{\sphinxlogo}{\vbox{}}
\renewcommand{\releasename}{Release}
\makeindex
\begin{document}

\pagestyle{empty}
\maketitle
\pagestyle{plain}
\sphinxtableofcontents
\pagestyle{normal}
\phantomsection\label{\detokenize{index::doc}}



\chapter{Pygpc package}
\label{\detokenize{pygpc:module-pygpc}}\label{\detokenize{pygpc:pygpc-package}}\label{\detokenize{pygpc::doc}}\index{pygpc (module)}
A package that provides submodules in order to perform polynomial chaos uncertanty analysis on complex dynamic systems.


\section{Submodules}
\label{\detokenize{pygpc:submodules}}

\section{pygpc.gpc module}
\label{\detokenize{pygpc:module-pygpc.gpc}}\label{\detokenize{pygpc:pygpc-gpc-module}}\index{pygpc.gpc (module)}
Class that provides general polynomial chaos methods
\index{gPC (class in pygpc.gpc)}

\begin{fulllineitems}
\phantomsection\label{\detokenize{pygpc:pygpc.gpc.gPC}}\pysigline{\sphinxbfcode{\sphinxupquote{class }}\sphinxcode{\sphinxupquote{pygpc.gpc.}}\sphinxbfcode{\sphinxupquote{gPC}}}
General gPC base class
\index{N\_grid (pygpc.gpc.gPC attribute)}

\begin{fulllineitems}
\phantomsection\label{\detokenize{pygpc:pygpc.gpc.gPC.N_grid}}\pysigline{\sphinxbfcode{\sphinxupquote{N\_grid}}}
number of grid points
\begin{quote}\begin{description}
\item[{Type}] \leavevmode
int

\end{description}\end{quote}

\end{fulllineitems}

\index{N\_poly (pygpc.gpc.gPC attribute)}

\begin{fulllineitems}
\phantomsection\label{\detokenize{pygpc:pygpc.gpc.gPC.N_poly}}\pysigline{\sphinxbfcode{\sphinxupquote{N\_poly}}}
number of polynomials psi
\begin{quote}\begin{description}
\item[{Type}] \leavevmode
int

\end{description}\end{quote}

\end{fulllineitems}

\index{N\_samples (pygpc.gpc.gPC attribute)}

\begin{fulllineitems}
\phantomsection\label{\detokenize{pygpc:pygpc.gpc.gPC.N_samples}}\pysigline{\sphinxbfcode{\sphinxupquote{N\_samples}}}
number of samples xi
\begin{quote}\begin{description}
\item[{Type}] \leavevmode
int

\end{description}\end{quote}

\end{fulllineitems}

\index{N\_out (pygpc.gpc.gPC attribute)}

\begin{fulllineitems}
\phantomsection\label{\detokenize{pygpc:pygpc.gpc.gPC.N_out}}\pysigline{\sphinxbfcode{\sphinxupquote{N\_out}}}
number of output coefficients
\begin{quote}\begin{description}
\item[{Type}] \leavevmode
int

\end{description}\end{quote}

\end{fulllineitems}

\index{dim (pygpc.gpc.gPC attribute)}

\begin{fulllineitems}
\phantomsection\label{\detokenize{pygpc:pygpc.gpc.gPC.dim}}\pysigline{\sphinxbfcode{\sphinxupquote{dim}}}
number of uncertain parameters to process
\begin{quote}\begin{description}
\item[{Type}] \leavevmode
int

\end{description}\end{quote}

\end{fulllineitems}

\index{pdf\_type (pygpc.gpc.gPC attribute)}

\begin{fulllineitems}
\phantomsection\label{\detokenize{pygpc:pygpc.gpc.gPC.pdf_type}}\pysigline{\sphinxbfcode{\sphinxupquote{pdf\_type}}}
type of pdf ‘beta’ or ‘norm’
\begin{quote}\begin{description}
\item[{Type}] \leavevmode
{[}dim{]} list of str

\end{description}\end{quote}

\end{fulllineitems}

\index{pdf\_shape (pygpc.gpc.gPC attribute)}

\begin{fulllineitems}
\phantomsection\label{\detokenize{pygpc:pygpc.gpc.gPC.pdf_shape}}\pysigline{\sphinxbfcode{\sphinxupquote{pdf\_shape}}}
shape parameters of pdfs
beta-dist:   {[}{[}alpha{]}, {[}beta{]}    {]}
normal-dist: {[}{[}mean{]},  {[}variance{]}{]}
\begin{quote}\begin{description}
\item[{Type}] \leavevmode
list of list of float

\end{description}\end{quote}

\end{fulllineitems}

\index{limits (pygpc.gpc.gPC attribute)}

\begin{fulllineitems}
\phantomsection\label{\detokenize{pygpc:pygpc.gpc.gPC.limits}}\pysigline{\sphinxbfcode{\sphinxupquote{limits}}}
upper and lower bounds of random variables
beta-dist:   {[}{[}a1 …{]}, {[}b1 …{]}{]}
normal-dist: {[}{[}0 … {]}, {[}0 … {]}{]} (not used)
\begin{quote}\begin{description}
\item[{Type}] \leavevmode
list of list of float

\end{description}\end{quote}

\end{fulllineitems}

\index{order (pygpc.gpc.gPC attribute)}

\begin{fulllineitems}
\phantomsection\label{\detokenize{pygpc:pygpc.gpc.gPC.order}}\pysigline{\sphinxbfcode{\sphinxupquote{order}}}
maximum individual expansion order
generates individual polynomials also if maximum expansion order in order\_max is exceeded
\begin{quote}\begin{description}
\item[{Type}] \leavevmode
{[}dim{]} list of int

\end{description}\end{quote}

\end{fulllineitems}

\index{order\_max (pygpc.gpc.gPC attribute)}

\begin{fulllineitems}
\phantomsection\label{\detokenize{pygpc:pygpc.gpc.gPC.order_max}}\pysigline{\sphinxbfcode{\sphinxupquote{order\_max}}}
maximum expansion order (sum of all exponents)
the maximum expansion order considers the sum of the orders of combined polynomials only
\begin{quote}\begin{description}
\item[{Type}] \leavevmode
int

\end{description}\end{quote}

\end{fulllineitems}

\index{interaction\_order (pygpc.gpc.gPC attribute)}

\begin{fulllineitems}
\phantomsection\label{\detokenize{pygpc:pygpc.gpc.gPC.interaction_order}}\pysigline{\sphinxbfcode{\sphinxupquote{interaction\_order}}}
number of random variables, which can interact with each other
all polynomials are ignored, which have an interaction order greater than the specified
\begin{quote}\begin{description}
\item[{Type}] \leavevmode
int

\end{description}\end{quote}

\end{fulllineitems}

\index{grid (pygpc.gpc.gPC attribute)}

\begin{fulllineitems}
\phantomsection\label{\detokenize{pygpc:pygpc.gpc.gPC.grid}}\pysigline{\sphinxbfcode{\sphinxupquote{grid}}}
grid object generated in grid.py including grid.coords and grid.coords\_norm
\begin{quote}\begin{description}
\item[{Type}] \leavevmode
grid object

\end{description}\end{quote}

\end{fulllineitems}

\index{random\_vars (pygpc.gpc.gPC attribute)}

\begin{fulllineitems}
\phantomsection\label{\detokenize{pygpc:pygpc.gpc.gPC.random_vars}}\pysigline{\sphinxbfcode{\sphinxupquote{random\_vars}}}
string labels of the random variables
\begin{quote}\begin{description}
\item[{Type}] \leavevmode
{[}dim{]} list of str

\end{description}\end{quote}

\end{fulllineitems}

\index{sobol (pygpc.gpc.gPC attribute)}

\begin{fulllineitems}
\phantomsection\label{\detokenize{pygpc:pygpc.gpc.gPC.sobol}}\pysigline{\sphinxbfcode{\sphinxupquote{sobol}}}
Sobol indices of N\_out output quantities
\begin{quote}\begin{description}
\item[{Type}] \leavevmode
{[}N\_sobol x N\_out{]} np.ndarray

\end{description}\end{quote}

\end{fulllineitems}

\index{sobol\_idx (pygpc.gpc.gPC attribute)}

\begin{fulllineitems}
\phantomsection\label{\detokenize{pygpc:pygpc.gpc.gPC.sobol_idx}}\pysigline{\sphinxbfcode{\sphinxupquote{sobol\_idx}}}
List of parameter label indices belonging to Sobol indices
\begin{quote}\begin{description}
\item[{Type}] \leavevmode
{[}N\_sobol{]} list of np.ndarray

\end{description}\end{quote}

\end{fulllineitems}

\index{cpu (pygpc.gpc.gPC attribute)}

\begin{fulllineitems}
\phantomsection\label{\detokenize{pygpc:pygpc.gpc.gPC.cpu}}\pysigline{\sphinxbfcode{\sphinxupquote{cpu}}}
flag to execute the calculation on the cpu
\begin{quote}\begin{description}
\item[{Type}] \leavevmode
bool

\end{description}\end{quote}

\end{fulllineitems}

\index{gpu (pygpc.gpc.gPC attribute)}

\begin{fulllineitems}
\phantomsection\label{\detokenize{pygpc:pygpc.gpc.gPC.gpu}}\pysigline{\sphinxbfcode{\sphinxupquote{gpu}}}
flag to execute the calculation on the gpu
\begin{quote}\begin{description}
\item[{Type}] \leavevmode
bool

\end{description}\end{quote}

\end{fulllineitems}

\index{verbose (pygpc.gpc.gPC attribute)}

\begin{fulllineitems}
\phantomsection\label{\detokenize{pygpc:pygpc.gpc.gPC.verbose}}\pysigline{\sphinxbfcode{\sphinxupquote{verbose}}}
boolean value to determine if to print out the progress into the standard output
\begin{quote}\begin{description}
\item[{Type}] \leavevmode
bool

\end{description}\end{quote}

\end{fulllineitems}

\index{gpc\_matrix (pygpc.gpc.gPC attribute)}

\begin{fulllineitems}
\phantomsection\label{\detokenize{pygpc:pygpc.gpc.gPC.gpc_matrix}}\pysigline{\sphinxbfcode{\sphinxupquote{gpc\_matrix}}}
generalized polynomial chaos matrix
\begin{quote}\begin{description}
\item[{Type}] \leavevmode
{[}N\_samples x N\_poly{]} np.ndarray

\end{description}\end{quote}

\end{fulllineitems}

\index{gpc\_matrix\_inv (pygpc.gpc.gPC attribute)}

\begin{fulllineitems}
\phantomsection\label{\detokenize{pygpc:pygpc.gpc.gPC.gpc_matrix_inv}}\pysigline{\sphinxbfcode{\sphinxupquote{gpc\_matrix\_inv}}}
pseudo inverse of the generalized polynomial chaos matrix
\begin{quote}\begin{description}
\item[{Type}] \leavevmode
{[}N\_poly x N\_samples{]} np.ndarray

\end{description}\end{quote}

\end{fulllineitems}

\index{gpc\_coeffs (pygpc.gpc.gPC attribute)}

\begin{fulllineitems}
\phantomsection\label{\detokenize{pygpc:pygpc.gpc.gPC.gpc_coeffs}}\pysigline{\sphinxbfcode{\sphinxupquote{gpc\_coeffs}}}
coefficient matrix of independent regions of interest for every coefficient
\begin{quote}\begin{description}
\item[{Type}] \leavevmode
{[}N\_poly x N\_out{]} np.ndarray

\end{description}\end{quote}

\end{fulllineitems}

\index{poly (pygpc.gpc.gPC attribute)}

\begin{fulllineitems}
\phantomsection\label{\detokenize{pygpc:pygpc.gpc.gPC.poly}}\pysigline{\sphinxbfcode{\sphinxupquote{poly}}}
polynomial objects containing the coefficients that are used to build the gpc matrix
\begin{quote}\begin{description}
\item[{Type}] \leavevmode
{[}dim x order\_span{]} list of list of np.poly1d:

\end{description}\end{quote}

\end{fulllineitems}

\index{poly\_gpu (pygpc.gpc.gPC attribute)}

\begin{fulllineitems}
\phantomsection\label{\detokenize{pygpc:pygpc.gpc.gPC.poly_gpu}}\pysigline{\sphinxbfcode{\sphinxupquote{poly\_gpu}}}
polynomial coefficients stored in a np.ndarray that can be processed on a graphic card
\begin{quote}\begin{description}
\item[{Type}] \leavevmode
np.ndarray

\end{description}\end{quote}

\end{fulllineitems}

\index{poly\_idx (pygpc.gpc.gPC attribute)}

\begin{fulllineitems}
\phantomsection\label{\detokenize{pygpc:pygpc.gpc.gPC.poly_idx}}\pysigline{\sphinxbfcode{\sphinxupquote{poly\_idx}}}
multi indices to determine the degree of the used sub-polynomials
\begin{quote}\begin{description}
\item[{Type}] \leavevmode
{[}N\_poly x dim{]} np.ndarray

\end{description}\end{quote}

\end{fulllineitems}



\begin{fulllineitems}
\pysigline{\sphinxbfcode{\sphinxupquote{poly\_idx\_gpu~{[}N\_poly~x~dim{]}~np.ndarray}}}
multi indices to determine the degree of the used sub-polynomials stored in a np.ndarray that can be processed
on a graphic card

\end{fulllineitems}

\index{poly\_der (pygpc.gpc.gPC attribute)}

\begin{fulllineitems}
\phantomsection\label{\detokenize{pygpc:pygpc.gpc.gPC.poly_der}}\pysigline{\sphinxbfcode{\sphinxupquote{poly\_der}}}
derivative of the polynomial objects containing the coefficients that are used to build the gpc matrix
\begin{quote}\begin{description}
\item[{Type}] \leavevmode
{[}dim x order\_span{]} list of list of np.poly1d:

\end{description}\end{quote}

\end{fulllineitems}

\index{poly\_norm (pygpc.gpc.gPC attribute)}

\begin{fulllineitems}
\phantomsection\label{\detokenize{pygpc:pygpc.gpc.gPC.poly_norm}}\pysigline{\sphinxbfcode{\sphinxupquote{poly\_norm}}}
normalizing scaling factors of the used sub-polynomials
\begin{quote}\begin{description}
\item[{Type}] \leavevmode
{[}order\_span x dim{]} np.ndarray

\end{description}\end{quote}

\end{fulllineitems}

\index{poly\_norm\_basis (pygpc.gpc.gPC attribute)}

\begin{fulllineitems}
\phantomsection\label{\detokenize{pygpc:pygpc.gpc.gPC.poly_norm_basis}}\pysigline{\sphinxbfcode{\sphinxupquote{poly\_norm\_basis}}}
normalizing scaling factors of the polynomial basis functions
\begin{quote}\begin{description}
\item[{Type}] \leavevmode
{[}N\_poly{]} np.ndarray

\end{description}\end{quote}

\end{fulllineitems}

\index{sobol\_idx\_bool (pygpc.gpc.gPC attribute)}

\begin{fulllineitems}
\phantomsection\label{\detokenize{pygpc:pygpc.gpc.gPC.sobol_idx_bool}}\pysigline{\sphinxbfcode{\sphinxupquote{sobol\_idx\_bool}}}
boolean mask that determines which multi indices are unique
\begin{quote}\begin{description}
\item[{Type}] \leavevmode
list of np.ndarray of bool

\end{description}\end{quote}

\end{fulllineitems}

\index{extend\_gpc\_matrix\_samples() (pygpc.gpc.gPC method)}

\begin{fulllineitems}
\phantomsection\label{\detokenize{pygpc:pygpc.gpc.gPC.extend_gpc_matrix_samples}}\pysiglinewithargsret{\sphinxbfcode{\sphinxupquote{extend\_gpc\_matrix\_samples}}}{\emph{samples\_poly\_ratio}, \emph{seed=None}}{}
Add sample points according to input pdfs to grid and extend the gpc matrix such that the ratio of
rows/columns equals samples\_poly\_ratio.

extend\_gpc\_matrix\_samples(samples\_poly\_ratio, seed=None):
\begin{quote}\begin{description}
\item[{Parameters}] \leavevmode\begin{itemize}
\item {} 
\sphinxstyleliteralstrong{\sphinxupquote{samples\_poly\_ratio}} (\sphinxstyleliteralemphasis{\sphinxupquote{float}}) \textendash{} ratio between number of samples and number of polynomials the matrix will be extended until

\item {} 
\sphinxstyleliteralstrong{\sphinxupquote{seed}} (\sphinxstyleliteralemphasis{\sphinxupquote{float}}\sphinxstyleliteralemphasis{\sphinxupquote{, }}\sphinxstyleliteralemphasis{\sphinxupquote{optional}}\sphinxstyleliteralemphasis{\sphinxupquote{, }}\sphinxstyleliteralemphasis{\sphinxupquote{default=None}}) \textendash{} random seeding point

\end{itemize}

\end{description}\end{quote}

\end{fulllineitems}

\index{extend\_polynomial\_basis() (pygpc.gpc.gPC method)}

\begin{fulllineitems}
\phantomsection\label{\detokenize{pygpc:pygpc.gpc.gPC.extend_polynomial_basis}}\pysiglinewithargsret{\sphinxbfcode{\sphinxupquote{extend\_polynomial\_basis}}}{\emph{poly\_idx\_added}}{}
Extend polynomial basis functions and add new columns to gpc matrix.

extend\_polynomial\_basis(poly\_idx\_added)
\begin{quote}\begin{description}
\item[{Parameters}] \leavevmode
\sphinxstyleliteralstrong{\sphinxupquote{poly\_idx\_added}} (\sphinxstyleliteralemphasis{\sphinxupquote{{[}}}\sphinxstyleliteralemphasis{\sphinxupquote{N\_poly\_added x dim}}\sphinxstyleliteralemphasis{\sphinxupquote{{]} }}\sphinxstyleliteralemphasis{\sphinxupquote{np.ndarray}}) \textendash{} array of added polynomials (order)

\end{description}\end{quote}

\end{fulllineitems}

\index{get\_global\_sens() (pygpc.gpc.gPC method)}

\begin{fulllineitems}
\phantomsection\label{\detokenize{pygpc:pygpc.gpc.gPC.get_global_sens}}\pysiglinewithargsret{\sphinxbfcode{\sphinxupquote{get\_global\_sens}}}{\emph{coeffs}}{}
Determine the global derivative based sensitivity coefficients.

Reference:
D. Xiu, Fast Numerical Methods for Stochastic Computations: A Review,
Commun. Comput. Phys., 5 (2009), pp. 242-272 eq. (3.14) page 255

get\_global\_sens = calc\_globalsens(coeffs)
\begin{quote}\begin{description}
\item[{Parameters}] \leavevmode
\sphinxstyleliteralstrong{\sphinxupquote{coeffs}} (\sphinxstyleliteralemphasis{\sphinxupquote{{[}}}\sphinxstyleliteralemphasis{\sphinxupquote{N\_coeffs x N\_out}}\sphinxstyleliteralemphasis{\sphinxupquote{{]} }}\sphinxstyleliteralemphasis{\sphinxupquote{np.ndarray}}) \textendash{} gpc coefficients

\item[{Returns}] \leavevmode
\sphinxstylestrong{get\_global\_sens} \textendash{} global derivative based sensitivity coefficients

\item[{Return type}] \leavevmode
{[}dim x N\_out{]} np.ndarray

\end{description}\end{quote}

\end{fulllineitems}

\index{get\_local\_sens() (pygpc.gpc.gPC method)}

\begin{fulllineitems}
\phantomsection\label{\detokenize{pygpc:pygpc.gpc.gPC.get_local_sens}}\pysiglinewithargsret{\sphinxbfcode{\sphinxupquote{get\_local\_sens}}}{\emph{coeffs}, \emph{xi}}{}
Determine the local derivative based sensitivity coefficients in the point of operation xi
in normalized coordinates.

get\_local\_sens = calc\_localsens(coeffs, xi)
\begin{quote}\begin{description}
\item[{Parameters}] \leavevmode\begin{itemize}
\item {} 
\sphinxstyleliteralstrong{\sphinxupquote{coeffs}} (\sphinxstyleliteralemphasis{\sphinxupquote{{[}}}\sphinxstyleliteralemphasis{\sphinxupquote{N\_coeffs x N\_out}}\sphinxstyleliteralemphasis{\sphinxupquote{{]} }}\sphinxstyleliteralemphasis{\sphinxupquote{np.ndarray}}) \textendash{} gpc coefficients

\item {} 
\sphinxstyleliteralstrong{\sphinxupquote{xi}} (\sphinxstyleliteralemphasis{\sphinxupquote{{[}}}\sphinxstyleliteralemphasis{\sphinxupquote{N\_coeffs x N\_out}}\sphinxstyleliteralemphasis{\sphinxupquote{{]} }}\sphinxstyleliteralemphasis{\sphinxupquote{np.ndarray}}) \textendash{} point in variable space to evaluate local sensitivity in (normalized coordinates!)

\end{itemize}

\item[{Returns}] \leavevmode
\sphinxstylestrong{get\_local\_sens} \textendash{} local sensitivity

\item[{Return type}] \leavevmode
{[}dim x N\_out{]} np.ndarray

\end{description}\end{quote}

\end{fulllineitems}

\index{get\_mean\_random\_vars() (pygpc.gpc.gPC method)}

\begin{fulllineitems}
\phantomsection\label{\detokenize{pygpc:pygpc.gpc.gPC.get_mean_random_vars}}\pysiglinewithargsret{\sphinxbfcode{\sphinxupquote{get\_mean\_random\_vars}}}{}{}
Determine the average values of the input random variables from their pdfs.
\begin{quote}\begin{description}
\item[{Returns}] \leavevmode
\sphinxstylestrong{mean\_random\_vars} \textendash{} average values of the input random variables

\item[{Return type}] \leavevmode
{[}N\_random\_vars{]} np.ndarray

\end{description}\end{quote}

\end{fulllineitems}

\index{get\_mean\_value() (pygpc.gpc.gPC static method)}

\begin{fulllineitems}
\phantomsection\label{\detokenize{pygpc:pygpc.gpc.gPC.get_mean_value}}\pysiglinewithargsret{\sphinxbfcode{\sphinxupquote{static }}\sphinxbfcode{\sphinxupquote{get\_mean\_value}}}{\emph{coeffs}}{}
Calculate the expected mean value.

mean = get\_mean\_value(coeffs)
\begin{quote}\begin{description}
\item[{Parameters}] \leavevmode
\sphinxstyleliteralstrong{\sphinxupquote{coeffs}} (\sphinxstyleliteralemphasis{\sphinxupquote{{[}}}\sphinxstyleliteralemphasis{\sphinxupquote{N\_coeffs x N\_out}}\sphinxstyleliteralemphasis{\sphinxupquote{{]} }}\sphinxstyleliteralemphasis{\sphinxupquote{np.ndarray}}) \textendash{} gpc coefficients

\item[{Returns}] \leavevmode
\sphinxstylestrong{mean} \textendash{} expected mean value

\item[{Return type}] \leavevmode
{[}1 x N\_out{]} np.ndarray

\end{description}\end{quote}

\end{fulllineitems}

\index{get\_pce() (pygpc.gpc.gPC method)}

\begin{fulllineitems}
\phantomsection\label{\detokenize{pygpc:pygpc.gpc.gPC.get_pce}}\pysiglinewithargsret{\sphinxbfcode{\sphinxupquote{get\_pce}}}{\emph{coeffs=None}, \emph{xi=None}, \emph{output\_idx=None}}{}
Calculates the gPC approximation in points with output\_idx and normalized parameters xi (interval: {[}-1, 1{]}).

pce = get\_pce(coeffs=None, xi=None, output\_idx=None)
\begin{quote}\begin{description}
\item[{Parameters}] \leavevmode\begin{itemize}
\item {} 
\sphinxstyleliteralstrong{\sphinxupquote{coeffs}} (\sphinxstyleliteralemphasis{\sphinxupquote{{[}}}\sphinxstyleliteralemphasis{\sphinxupquote{N\_coeffs x N\_out}}\sphinxstyleliteralemphasis{\sphinxupquote{{]} }}\sphinxstyleliteralemphasis{\sphinxupquote{np.ndarray}}\sphinxstyleliteralemphasis{\sphinxupquote{, }}\sphinxstyleliteralemphasis{\sphinxupquote{optional}}\sphinxstyleliteralemphasis{\sphinxupquote{, }}\sphinxstyleliteralemphasis{\sphinxupquote{default=None}}) \textendash{} gpc coefficients

\item {} 
\sphinxstyleliteralstrong{\sphinxupquote{xi}} (\sphinxstyleliteralemphasis{\sphinxupquote{{[}}}\sphinxstyleliteralemphasis{\sphinxupquote{1 x dim}}\sphinxstyleliteralemphasis{\sphinxupquote{{]} }}\sphinxstyleliteralemphasis{\sphinxupquote{np.ndarray}}\sphinxstyleliteralemphasis{\sphinxupquote{, }}\sphinxstyleliteralemphasis{\sphinxupquote{optional}}\sphinxstyleliteralemphasis{\sphinxupquote{, }}\sphinxstyleliteralemphasis{\sphinxupquote{default=None}}) \textendash{} point in variable space to evaluate local sensitivity in normalized coordinates

\item {} 
\sphinxstyleliteralstrong{\sphinxupquote{output\_idx}} (\sphinxstyleliteralemphasis{\sphinxupquote{{[}}}\sphinxstyleliteralemphasis{\sphinxupquote{1 x N\_out}}\sphinxstyleliteralemphasis{\sphinxupquote{{]} }}\sphinxstyleliteralemphasis{\sphinxupquote{np.ndarray}}\sphinxstyleliteralemphasis{\sphinxupquote{, }}\sphinxstyleliteralemphasis{\sphinxupquote{optional}}\sphinxstyleliteralemphasis{\sphinxupquote{, }}\sphinxstyleliteralemphasis{\sphinxupquote{default=None}}) \textendash{} idx of output quantities to consider (Default: all outputs)

\end{itemize}

\item[{Returns}] \leavevmode
\sphinxstylestrong{pce} \textendash{} gpc approximation at normalized coordinates xi

\item[{Return type}] \leavevmode
{[}N\_xi x N\_out{]} np.ndarray

\end{description}\end{quote}
\subsubsection*{Example}

pce = get\_pce({[}{[}xi\_1\_p1 … xi\_dim\_p1{]} ,{[}xi\_1\_p2 … xi\_dim\_p2{]}{]}, np.array({[}{[}0,5,13{]}{]}))

\end{fulllineitems}

\index{get\_pdf() (pygpc.gpc.gPC method)}

\begin{fulllineitems}
\phantomsection\label{\detokenize{pygpc:pygpc.gpc.gPC.get_pdf}}\pysiglinewithargsret{\sphinxbfcode{\sphinxupquote{get\_pdf}}}{\emph{coeffs}, \emph{N\_samples}, \emph{output\_idx=None}}{}
Determine the estimated pdfs of the output quantities

pdf\_x, pdf\_y = get\_pdf(coeffs, N\_samples, output\_idx=None)
\begin{quote}\begin{description}
\item[{Parameters}] \leavevmode\begin{itemize}
\item {} 
\sphinxstyleliteralstrong{\sphinxupquote{coeffs}} (\sphinxstyleliteralemphasis{\sphinxupquote{{[}}}\sphinxstyleliteralemphasis{\sphinxupquote{N\_coeffs x N\_out}}\sphinxstyleliteralemphasis{\sphinxupquote{{]} }}\sphinxstyleliteralemphasis{\sphinxupquote{np.ndarray}}) \textendash{} gpc coefficients

\item {} 
\sphinxstyleliteralstrong{\sphinxupquote{N\_samples}} (\sphinxstyleliteralemphasis{\sphinxupquote{int}}) \textendash{} number of samples used to estimate output pdf

\item {} 
\sphinxstyleliteralstrong{\sphinxupquote{output\_idx}} (\sphinxstyleliteralemphasis{\sphinxupquote{{[}}}\sphinxstyleliteralemphasis{\sphinxupquote{1 x N\_out}}\sphinxstyleliteralemphasis{\sphinxupquote{{]} }}\sphinxstyleliteralemphasis{\sphinxupquote{np.ndarray}}\sphinxstyleliteralemphasis{\sphinxupquote{, }}\sphinxstyleliteralemphasis{\sphinxupquote{optional}}\sphinxstyleliteralemphasis{\sphinxupquote{, }}\sphinxstyleliteralemphasis{\sphinxupquote{default=None}}) \textendash{} idx of output quantities to consider
if output\_idx=None, all output quantities are considered

\end{itemize}

\item[{Returns}] \leavevmode
\begin{itemize}
\item {} 
\sphinxstylestrong{pdf\_x} (\sphinxstyleemphasis{{[}100 x N\_out{]} np.ndarray}) \textendash{} x-coordinates of output pdf (output quantity),

\item {} 
\sphinxstylestrong{pdf\_y} (\sphinxstyleemphasis{{[}100 x N\_out{]} np.ndarray}) \textendash{} y-coordinates of output pdf (probability density of output quantity)

\end{itemize}


\end{description}\end{quote}

\end{fulllineitems}

\index{get\_pdf\_monte\_carlo() (pygpc.gpc.gPC method)}

\begin{fulllineitems}
\phantomsection\label{\detokenize{pygpc:pygpc.gpc.gPC.get_pdf_monte_carlo}}\pysiglinewithargsret{\sphinxbfcode{\sphinxupquote{get\_pdf\_monte\_carlo}}}{\emph{N\_samples}, \emph{coeffs=None}, \emph{output\_idx=None}}{}
Randomly sample the gPC expansion to determine output pdfs in specific points.

xi = get\_pdf\_mc(N\_samples, coeffs=None, output\_idx=None)
\begin{quote}\begin{description}
\item[{Parameters}] \leavevmode\begin{itemize}
\item {} 
\sphinxstyleliteralstrong{\sphinxupquote{N\_samples}} (\sphinxstyleliteralemphasis{\sphinxupquote{int}}) \textendash{} number of random samples drawn from the respective input pdfs

\item {} 
\sphinxstyleliteralstrong{\sphinxupquote{output\_idx}} (\sphinxstyleliteralemphasis{\sphinxupquote{{[}}}\sphinxstyleliteralemphasis{\sphinxupquote{1 x N\_out}}\sphinxstyleliteralemphasis{\sphinxupquote{{]} }}\sphinxstyleliteralemphasis{\sphinxupquote{np.ndarray}}\sphinxstyleliteralemphasis{\sphinxupquote{, }}\sphinxstyleliteralemphasis{\sphinxupquote{optional}}\sphinxstyleliteralemphasis{\sphinxupquote{, }}\sphinxstyleliteralemphasis{\sphinxupquote{default=None}}) \textendash{} idx of output quantities to consider

\item {} 
\sphinxstyleliteralstrong{\sphinxupquote{coeffs}} (\sphinxstyleliteralemphasis{\sphinxupquote{{[}}}\sphinxstyleliteralemphasis{\sphinxupquote{N\_coeffs x N\_out}}\sphinxstyleliteralemphasis{\sphinxupquote{{]} }}\sphinxstyleliteralemphasis{\sphinxupquote{np.ndarray}}\sphinxstyleliteralemphasis{\sphinxupquote{, }}\sphinxstyleliteralemphasis{\sphinxupquote{optional}}\sphinxstyleliteralemphasis{\sphinxupquote{, }}\sphinxstyleliteralemphasis{\sphinxupquote{default=None}}) \textendash{} gPC coefficients

\end{itemize}

\item[{Returns}] \leavevmode
\sphinxstylestrong{xi} \textendash{} generated samples in normalized coordinates

\item[{Return type}] \leavevmode
{[}N\_samples x dim{]} np.ndarray

\end{description}\end{quote}

\end{fulllineitems}

\index{get\_sobol\_indices() (pygpc.gpc.gPC method)}

\begin{fulllineitems}
\phantomsection\label{\detokenize{pygpc:pygpc.gpc.gPC.get_sobol_indices}}\pysiglinewithargsret{\sphinxbfcode{\sphinxupquote{get\_sobol\_indices}}}{\emph{coeffs=None}}{}
Determine the available sobol indices.

sobol, sobol\_idx = get\_sobol\_indices(coeffs=None)
\begin{quote}\begin{description}
\item[{Parameters}] \leavevmode
\sphinxstyleliteralstrong{\sphinxupquote{coeffs}} (\sphinxstyleliteralemphasis{\sphinxupquote{{[}}}\sphinxstyleliteralemphasis{\sphinxupquote{N\_coeffs x N\_out}}\sphinxstyleliteralemphasis{\sphinxupquote{{]} }}\sphinxstyleliteralemphasis{\sphinxupquote{np.ndarray}}\sphinxstyleliteralemphasis{\sphinxupquote{, }}\sphinxstyleliteralemphasis{\sphinxupquote{optional}}\sphinxstyleliteralemphasis{\sphinxupquote{, }}\sphinxstyleliteralemphasis{\sphinxupquote{default=None}}) \textendash{} gpc coefficients

\item[{Returns}] \leavevmode
\begin{itemize}
\item {} 
\sphinxstylestrong{sobol} (\sphinxstyleemphasis{{[}N\_sobol x N\_out{]} np.ndarray}) \textendash{} unnormalized sobol\_indices

\item {} 
\sphinxstylestrong{sobol\_idx} (\sphinxstyleemphasis{list of {[}N\_sobol x dim{]} np.ndarray}) \textendash{} list containing the parameter combinations in rows of sobol

\item {} 
\sphinxstylestrong{sobol\_idx\_bool} (\sphinxstyleemphasis{list of np.ndarray of bool}) \textendash{} boolean mask that determines which multi indices are unique

\end{itemize}


\end{description}\end{quote}

\end{fulllineitems}

\index{get\_sobol\_order() (pygpc.gpc.gPC method)}

\begin{fulllineitems}
\phantomsection\label{\detokenize{pygpc:pygpc.gpc.gPC.get_sobol_order}}\pysiglinewithargsret{\sphinxbfcode{\sphinxupquote{get\_sobol\_order}}}{\emph{sobol=None}, \emph{sobol\_idx=None}, \emph{sobol\_idx\_bool=None}}{}
Evaluate order of determined sobol indices.

sobol, sobol\_idx, sobol\_rel\_order\_mean, sobol\_rel\_order\_std, sobol\_rel\_1st\_order\_mean, sobol\_rel\_1st\_order\_std
= get\_sobol\_order(coeffs=None, sobol=None, sobol\_idx=None, sobol\_idx\_bool=None)
\begin{quote}\begin{description}
\item[{Parameters}] \leavevmode\begin{itemize}
\item {} 
\sphinxstyleliteralstrong{\sphinxupquote{sobol}} (\sphinxstyleliteralemphasis{\sphinxupquote{{[}}}\sphinxstyleliteralemphasis{\sphinxupquote{N\_sobol x N\_out}}\sphinxstyleliteralemphasis{\sphinxupquote{{]} }}\sphinxstyleliteralemphasis{\sphinxupquote{np.ndarray}}) \textendash{} unnormalized sobol\_indices

\item {} 
\sphinxstyleliteralstrong{\sphinxupquote{sobol\_idx}} (\sphinxstyleliteralemphasis{\sphinxupquote{list of}}\sphinxstyleliteralemphasis{\sphinxupquote{ {[}}}\sphinxstyleliteralemphasis{\sphinxupquote{N\_sobol x dim}}\sphinxstyleliteralemphasis{\sphinxupquote{{]} }}\sphinxstyleliteralemphasis{\sphinxupquote{np.ndarray}}) \textendash{} list containing the parameter combinations in rows of sobol

\item {} 
\sphinxstyleliteralstrong{\sphinxupquote{sobol\_idx\_bool}} (\sphinxstyleliteralemphasis{\sphinxupquote{list of np.ndarray of bool}}) \textendash{} boolean mask that determines which multi indices are unique

\end{itemize}

\item[{Returns}] \leavevmode
\begin{itemize}
\item {} 
\sphinxstylestrong{sobol\_rel\_order\_mean} (\sphinxstyleemphasis{np.ndarray}) \textendash{} average proportion of the Sobol indices of the different order to the total variance (1st, 2nd, etc..,)
over all output quantities

\item {} 
\sphinxstylestrong{sobol\_rel\_order\_std} (\sphinxstyleemphasis{np.ndarray}) \textendash{} standard deviation of the proportion of the Sobol indices of the different order to the total variance
(1st, 2nd, etc..,) over all output quantities

\item {} 
\sphinxstylestrong{sobol\_rel\_1st\_order\_mean} (\sphinxstyleemphasis{np.ndarray}) \textendash{} average proportion of the random variables of the 1st order Sobol indices to the total variance over all
output quantities

\item {} 
\sphinxstylestrong{sobol\_rel\_1st\_order\_std} (\sphinxstyleemphasis{np.ndarray}) \textendash{} standard deviation of the proportion of the random variables of the 1st order Sobol indices to the total
variance over all output quantities

\end{itemize}


\end{description}\end{quote}

\end{fulllineitems}

\index{get\_standard\_deviation() (pygpc.gpc.gPC static method)}

\begin{fulllineitems}
\phantomsection\label{\detokenize{pygpc:pygpc.gpc.gPC.get_standard_deviation}}\pysiglinewithargsret{\sphinxbfcode{\sphinxupquote{static }}\sphinxbfcode{\sphinxupquote{get\_standard\_deviation}}}{\emph{coeffs}}{}
Calculate the standard deviation.

std = get\_standard\_deviation(coeffs)
\begin{quote}\begin{description}
\item[{Parameters}] \leavevmode
\sphinxstyleliteralstrong{\sphinxupquote{coeffs}} (\sphinxstyleliteralemphasis{\sphinxupquote{np.array of float}}\sphinxstyleliteralemphasis{\sphinxupquote{ {[}}}\sphinxstyleliteralemphasis{\sphinxupquote{N\_coeffs x N\_out}}\sphinxstyleliteralemphasis{\sphinxupquote{{]}}}) \textendash{} gpc coefficients

\item[{Returns}] \leavevmode
\sphinxstylestrong{std} \textendash{} standard deviation

\item[{Return type}] \leavevmode
{[}1 x N\_out{]} np.ndarray

\end{description}\end{quote}

\end{fulllineitems}

\index{init\_gpc\_matrix() (pygpc.gpc.gPC method)}

\begin{fulllineitems}
\phantomsection\label{\detokenize{pygpc:pygpc.gpc.gPC.init_gpc_matrix}}\pysiglinewithargsret{\sphinxbfcode{\sphinxupquote{init\_gpc\_matrix}}}{}{}
Construct the gPC matrix and  the Moore-Penrose-pseudo-inverse.

init\_gpc\_matrix()

\end{fulllineitems}

\index{init\_polynomial\_basis() (pygpc.gpc.gPC method)}

\begin{fulllineitems}
\phantomsection\label{\detokenize{pygpc:pygpc.gpc.gPC.init_polynomial_basis}}\pysiglinewithargsret{\sphinxbfcode{\sphinxupquote{init\_polynomial\_basis}}}{}{}
Initialize polynomial basis functions for a maximum order expansion.

init\_polynomial\_basis()

\end{fulllineitems}

\index{init\_polynomial\_basis\_gpu() (pygpc.gpc.gPC method)}

\begin{fulllineitems}
\phantomsection\label{\detokenize{pygpc:pygpc.gpc.gPC.init_polynomial_basis_gpu}}\pysiglinewithargsret{\sphinxbfcode{\sphinxupquote{init\_polynomial\_basis\_gpu}}}{}{}
Initialized polynomial basis coefficients for graphic card. Converts list of lists of self.polynomial\_bases
into np.ndarray that can be processed on a graphic card.

init\_polynomial\_basis\_gpu()

\end{fulllineitems}

\index{init\_polynomial\_coeffs() (pygpc.gpc.gPC method)}

\begin{fulllineitems}
\phantomsection\label{\detokenize{pygpc:pygpc.gpc.gPC.init_polynomial_coeffs}}\pysiglinewithargsret{\sphinxbfcode{\sphinxupquote{init\_polynomial\_coeffs}}}{\emph{order\_begin}, \emph{order\_end}}{}
Calculate polynomial basis functions of a given order range and add it to the polynomial lookup tables.
The size, including the polynomials that won’t be used, is {[}max\_individual\_order x dim{]}.
\begin{equation*}
\begin{split}\begin{tabular}{l*{4}{c}}
 Polynomial          & Dimension 1 & Dimension 2 & ... & Dimension M \\
\hline
 Polynomial 1        & [Coefficients] & [Coefficients] & \vdots & [Coefficients] \\
 Polynomial 2        & 0 & [Coefficients] & \vdots & [Coefficients] \\
\vdots              & \vdots & \vdots & \vdots & \vdots \\
 Polynomial N        & [Coefficients] & [Coefficients] & 0 & [Coefficients] \\
\end{tabular}\end{split}
\end{equation*}
init\_polynomial\_coeffs(poly\_idx\_added)
\begin{quote}\begin{description}
\item[{Parameters}] \leavevmode\begin{itemize}
\item {} 
\sphinxstyleliteralstrong{\sphinxupquote{order\_begin}} (\sphinxstyleliteralemphasis{\sphinxupquote{int}}) \textendash{} order of polynomials to begin with

\item {} 
\sphinxstyleliteralstrong{\sphinxupquote{order\_end}} (\sphinxstyleliteralemphasis{\sphinxupquote{int}}) \textendash{} order of polynomials to end with

\end{itemize}

\end{description}\end{quote}

\end{fulllineitems}

\index{init\_polynomial\_index() (pygpc.gpc.gPC method)}

\begin{fulllineitems}
\phantomsection\label{\detokenize{pygpc:pygpc.gpc.gPC.init_polynomial_index}}\pysiglinewithargsret{\sphinxbfcode{\sphinxupquote{init\_polynomial\_index}}}{}{}
Initialize polynomial multi indices. Determine 2D multi-index array (order) of basis functions and
generate multi-index list up to maximum order. The size is {[}No. of basis functions x dim{]}.
\begin{equation*}
\begin{split}\begin{tabular}{l*{4}{c}}
 Polynomial Index    & Dimension 1 & Dimension 2 & ... & Dimension M \\
\hline
 Basis 1             & [Order D1] & [Order D2] & \vdots & [Order M] \\
 Basis 2             & [Order D1] & [Order D2] & \vdots & [Order M] \\
\vdots              & [Order D1] & [Order D2] & \vdots  & [Order M] \\
 Basis N           & [Order D1] & [Order D2] & \vdots & [Order M] \\
\end{tabular}\end{split}
\end{equation*}
init\_polynomial\_index()

\end{fulllineitems}

\index{replace\_gpc\_matrix\_samples() (pygpc.gpc.gPC method)}

\begin{fulllineitems}
\phantomsection\label{\detokenize{pygpc:pygpc.gpc.gPC.replace_gpc_matrix_samples}}\pysiglinewithargsret{\sphinxbfcode{\sphinxupquote{replace\_gpc\_matrix\_samples}}}{\emph{idx}, \emph{seed=None}}{}
Replace distinct sample points from the gpc matrix.

replace\_gpc\_matrix\_samples(idx, seed=None)
\begin{quote}\begin{description}
\item[{Parameters}] \leavevmode\begin{itemize}
\item {} 
\sphinxstyleliteralstrong{\sphinxupquote{idx}} (\sphinxstyleliteralemphasis{\sphinxupquote{np.ndarray}}) \textendash{} array of grid indices of obj.grid.coords{[}idx,:{]} which are going to be replaced
(rows of gPC matrix will be replaced by new ones)

\item {} 
\sphinxstyleliteralstrong{\sphinxupquote{seed}} (\sphinxstyleliteralemphasis{\sphinxupquote{float}}\sphinxstyleliteralemphasis{\sphinxupquote{, }}\sphinxstyleliteralemphasis{\sphinxupquote{optional}}\sphinxstyleliteralemphasis{\sphinxupquote{, }}\sphinxstyleliteralemphasis{\sphinxupquote{default=None}}) \textendash{} random seeding point

\end{itemize}

\end{description}\end{quote}

\end{fulllineitems}

\index{write\_log\_sobol() (pygpc.gpc.gPC method)}

\begin{fulllineitems}
\phantomsection\label{\detokenize{pygpc:pygpc.gpc.gPC.write_log_sobol}}\pysiglinewithargsret{\sphinxbfcode{\sphinxupquote{write\_log\_sobol}}}{\emph{fname}, \emph{sobol\_rel\_order\_mean}, \emph{sobol\_rel\_1st\_order\_mean}, \emph{sobol\_extracted\_idx\_1st}}{}
Write sobol indices into logfile.
\begin{quote}\begin{description}
\item[{Parameters}] \leavevmode\begin{itemize}
\item {} 
\sphinxstyleliteralstrong{\sphinxupquote{fname}} (\sphinxstyleliteralemphasis{\sphinxupquote{str}}) \textendash{} path to output file

\item {} 
\sphinxstyleliteralstrong{\sphinxupquote{sobol\_rel\_order\_mean}} (\sphinxstyleliteralemphasis{\sphinxupquote{np.ndarray}}) \textendash{} average proportion of the Sobol indices of the different order to the total variance (1st, 2nd, etc..,)
over all output quantities

\item {} 
\sphinxstyleliteralstrong{\sphinxupquote{sobol\_rel\_1st\_order\_mean}} (\sphinxstyleliteralemphasis{\sphinxupquote{np.ndarray}}) \textendash{} average proportion of the random variables of the 1st order Sobol indices to the total variance over all
output quantities

\item {} 
\sphinxstyleliteralstrong{\sphinxupquote{\#TODO}} (\sphinxstyleliteralemphasis{\sphinxupquote{add description}}) \textendash{} 

\item {} 
\sphinxstyleliteralstrong{\sphinxupquote{sobol\_extracted\_idx\_1st}} \textendash{} 

\end{itemize}

\end{description}\end{quote}

\end{fulllineitems}


\end{fulllineitems}



\section{pygpc.grid module}
\label{\detokenize{pygpc:module-pygpc.grid}}\label{\detokenize{pygpc:pygpc-grid-module}}\index{pygpc.grid (module)}
Functions and classes that provide data and methods for the generation and processing of numerical grids
\index{RandomGrid (class in pygpc.grid)}

\begin{fulllineitems}
\phantomsection\label{\detokenize{pygpc:pygpc.grid.RandomGrid}}\pysiglinewithargsret{\sphinxbfcode{\sphinxupquote{class }}\sphinxcode{\sphinxupquote{pygpc.grid.}}\sphinxbfcode{\sphinxupquote{RandomGrid}}}{\emph{pdf\_type}, \emph{grid\_shape}, \emph{limits}, \emph{N}, \emph{seed=None}}{}
Generate RandomGrid object instance.

RandomGrid(pdf\_type, grid\_shape, limits, N, seed=None)
\index{pdf\_type (pygpc.grid.RandomGrid attribute)}

\begin{fulllineitems}
\phantomsection\label{\detokenize{pygpc:pygpc.grid.RandomGrid.pdf_type}}\pysigline{\sphinxbfcode{\sphinxupquote{pdf\_type}}}
variable specific type of pdf (“beta”, “normal”)
\begin{quote}\begin{description}
\item[{Type}] \leavevmode
{[}N\_vars{]} list of str

\end{description}\end{quote}

\end{fulllineitems}

\index{grid\_shape (pygpc.grid.RandomGrid attribute)}

\begin{fulllineitems}
\phantomsection\label{\detokenize{pygpc:pygpc.grid.RandomGrid.grid_shape}}\pysigline{\sphinxbfcode{\sphinxupquote{grid\_shape}}}
shape parameters of PDF
beta (jacobi):  {[}alpha, beta{]}
norm (hermite): {[}mean, std{]}
\begin{quote}\begin{description}
\item[{Type}] \leavevmode
{[}2 x N\_vars{]} list of list of float

\end{description}\end{quote}

\end{fulllineitems}

\index{limits (pygpc.grid.RandomGrid attribute)}

\begin{fulllineitems}
\phantomsection\label{\detokenize{pygpc:pygpc.grid.RandomGrid.limits}}\pysigline{\sphinxbfcode{\sphinxupquote{limits}}}
Upper and lower bounds of PDF
beta (jacobi):  {[}min, max{]}
norm (hermite): {[}0, 0{]} (unused)
\begin{quote}\begin{description}
\item[{Type}] \leavevmode
{[}2 x N\_vars{]} list of list of float

\end{description}\end{quote}

\end{fulllineitems}

\index{N (pygpc.grid.RandomGrid attribute)}

\begin{fulllineitems}
\phantomsection\label{\detokenize{pygpc:pygpc.grid.RandomGrid.N}}\pysigline{\sphinxbfcode{\sphinxupquote{N}}}
number of random samples to generate
\begin{quote}\begin{description}
\item[{Type}] \leavevmode
int

\end{description}\end{quote}

\end{fulllineitems}

\index{seed (pygpc.grid.RandomGrid attribute)}

\begin{fulllineitems}
\phantomsection\label{\detokenize{pygpc:pygpc.grid.RandomGrid.seed}}\pysigline{\sphinxbfcode{\sphinxupquote{seed}}}
seeding point to replicate random grids
\begin{quote}\begin{description}
\item[{Type}] \leavevmode
float

\end{description}\end{quote}

\end{fulllineitems}

\index{dim (pygpc.grid.RandomGrid attribute)}

\begin{fulllineitems}
\phantomsection\label{\detokenize{pygpc:pygpc.grid.RandomGrid.dim}}\pysigline{\sphinxbfcode{\sphinxupquote{dim}}}
number of uncertain parameters to process
\begin{quote}\begin{description}
\item[{Type}] \leavevmode
int

\end{description}\end{quote}

\end{fulllineitems}

\index{coords (pygpc.grid.RandomGrid attribute)}

\begin{fulllineitems}
\phantomsection\label{\detokenize{pygpc:pygpc.grid.RandomGrid.coords}}\pysigline{\sphinxbfcode{\sphinxupquote{coords}}}
denormalized coordinates xi
\begin{quote}\begin{description}
\item[{Type}] \leavevmode
{[}N\_samples x dim{]} np.ndarray

\end{description}\end{quote}

\end{fulllineitems}

\index{coords\_norm (pygpc.grid.RandomGrid attribute)}

\begin{fulllineitems}
\phantomsection\label{\detokenize{pygpc:pygpc.grid.RandomGrid.coords_norm}}\pysigline{\sphinxbfcode{\sphinxupquote{coords\_norm}}}
normalized {[}-1, 1{]} coordinates xi
\begin{quote}\begin{description}
\item[{Type}] \leavevmode
{[}N\_samples x dim{]} np.ndarray

\end{description}\end{quote}

\end{fulllineitems}

\begin{quote}\begin{description}
\item[{Parameters}] \leavevmode\begin{itemize}
\item {} 
\sphinxstyleliteralstrong{\sphinxupquote{pdf\_type}} (\sphinxstyleliteralemphasis{\sphinxupquote{{[}}}\sphinxstyleliteralemphasis{\sphinxupquote{N\_vars}}\sphinxstyleliteralemphasis{\sphinxupquote{{]} }}\sphinxstyleliteralemphasis{\sphinxupquote{list of str}}) \textendash{} variable specific type of pdf (“beta”, “normal”)

\item {} 
\sphinxstyleliteralstrong{\sphinxupquote{grid\_shape}} (\sphinxstyleliteralemphasis{\sphinxupquote{{[}}}\sphinxstyleliteralemphasis{\sphinxupquote{2 x N\_vars}}\sphinxstyleliteralemphasis{\sphinxupquote{{]} }}\sphinxstyleliteralemphasis{\sphinxupquote{list of list of float}}) \textendash{} shape parameters of PDF
beta (jacobi):  {[}alpha, beta{]}
norm (hermite): {[}mean, std{]}

\item {} 
\sphinxstyleliteralstrong{\sphinxupquote{limits}} (\sphinxstyleliteralemphasis{\sphinxupquote{{[}}}\sphinxstyleliteralemphasis{\sphinxupquote{2 x N\_vars}}\sphinxstyleliteralemphasis{\sphinxupquote{{]} }}\sphinxstyleliteralemphasis{\sphinxupquote{list of list of float}}) \textendash{} Upper and lower bounds of PDF
beta (jacobi):  {[}min, max{]}
norm (hermite): {[}0, 0{]} (unused)

\item {} 
\sphinxstyleliteralstrong{\sphinxupquote{N}} (\sphinxstyleliteralemphasis{\sphinxupquote{int}}) \textendash{} number of random samples to generate

\item {} 
\sphinxstyleliteralstrong{\sphinxupquote{seed}} (\sphinxstyleliteralemphasis{\sphinxupquote{float}}\sphinxstyleliteralemphasis{\sphinxupquote{, }}\sphinxstyleliteralemphasis{\sphinxupquote{optional}}\sphinxstyleliteralemphasis{\sphinxupquote{, }}\sphinxstyleliteralemphasis{\sphinxupquote{default=None}}) \textendash{} seeding point to replicate random grids

\end{itemize}

\end{description}\end{quote}

\end{fulllineitems}

\index{SparseGrid (class in pygpc.grid)}

\begin{fulllineitems}
\phantomsection\label{\detokenize{pygpc:pygpc.grid.SparseGrid}}\pysiglinewithargsret{\sphinxbfcode{\sphinxupquote{class }}\sphinxcode{\sphinxupquote{pygpc.grid.}}\sphinxbfcode{\sphinxupquote{SparseGrid}}}{\emph{pdf\_type}, \emph{grid\_type}, \emph{grid\_shape}, \emph{limits}, \emph{level}, \emph{level\_max}, \emph{interaction\_order}, \emph{order\_sequence\_type}, \emph{make\_grid=True}, \emph{verbose=True}}{}
Generate SparseGrid object instance.
\begin{description}
\item[{SparseGrid(pdf\_type, grid\_type, grid\_shape, limits, level, level\_max, interaction\_order,}] \leavevmode
order\_sequence\_type, make\_grid=True, verbose=True)

\end{description}
\index{pdf\_type (pygpc.grid.SparseGrid attribute)}

\begin{fulllineitems}
\phantomsection\label{\detokenize{pygpc:pygpc.grid.SparseGrid.pdf_type}}\pysigline{\sphinxbfcode{\sphinxupquote{pdf\_type}}}
variable specific type of PDF (“beta”, “normal”)
\begin{quote}\begin{description}
\item[{Type}] \leavevmode
{[}N\_vars{]} list of str

\end{description}\end{quote}

\end{fulllineitems}

\index{grid\_type (pygpc.grid.SparseGrid attribute)}

\begin{fulllineitems}
\phantomsection\label{\detokenize{pygpc:pygpc.grid.SparseGrid.grid_type}}\pysigline{\sphinxbfcode{\sphinxupquote{grid\_type}}}
specify type of quadrature used to construct sparse grid (‘jacobi’, ‘hermite’, ‘cc’, ‘fejer2’)
\begin{quote}\begin{description}
\item[{Type}] \leavevmode
{[}N\_vars{]} list of str

\end{description}\end{quote}

\end{fulllineitems}

\index{grid\_shape (pygpc.grid.SparseGrid attribute)}

\begin{fulllineitems}
\phantomsection\label{\detokenize{pygpc:pygpc.grid.SparseGrid.grid_shape}}\pysigline{\sphinxbfcode{\sphinxupquote{grid\_shape}}}
shape parameters of PDF
beta (jacobi):  {[}alpha, beta{]}
norm (hermite): {[}mean, std{]}
\begin{quote}\begin{description}
\item[{Type}] \leavevmode
{[}2 x N\_vars{]} list of list of float

\end{description}\end{quote}

\end{fulllineitems}

\index{limits (pygpc.grid.SparseGrid attribute)}

\begin{fulllineitems}
\phantomsection\label{\detokenize{pygpc:pygpc.grid.SparseGrid.limits}}\pysigline{\sphinxbfcode{\sphinxupquote{limits}}}
upper and lower bounds of PDF
beta (jacobi):  {[}min, max{]}
norm (hermite): {[}0, 0{]} (unused)
\begin{quote}\begin{description}
\item[{Type}] \leavevmode
{[}2 x N\_vars{]} list of list of float

\end{description}\end{quote}

\end{fulllineitems}

\index{level (pygpc.grid.SparseGrid attribute)}

\begin{fulllineitems}
\phantomsection\label{\detokenize{pygpc:pygpc.grid.SparseGrid.level}}\pysigline{\sphinxbfcode{\sphinxupquote{level}}}
number of levels in each dimension
\begin{quote}\begin{description}
\item[{Type}] \leavevmode
{[}N\_vars{]} list of int

\end{description}\end{quote}

\end{fulllineitems}

\index{level\_max (pygpc.grid.SparseGrid attribute)}

\begin{fulllineitems}
\phantomsection\label{\detokenize{pygpc:pygpc.grid.SparseGrid.level_max}}\pysigline{\sphinxbfcode{\sphinxupquote{level\_max}}}
global combined level maximum
\begin{quote}\begin{description}
\item[{Type}] \leavevmode
int

\end{description}\end{quote}

\end{fulllineitems}

\index{interaction\_order (pygpc.grid.SparseGrid attribute)}

\begin{fulllineitems}
\phantomsection\label{\detokenize{pygpc:pygpc.grid.SparseGrid.interaction_order}}\pysigline{\sphinxbfcode{\sphinxupquote{interaction\_order}}}
interaction order of parameters and grid, i.e. the grid points are lying between this number of dimensions
\begin{quote}\begin{description}
\item[{Type}] \leavevmode
int

\end{description}\end{quote}

\end{fulllineitems}

\index{order\_sequence\_type (pygpc.grid.SparseGrid attribute)}

\begin{fulllineitems}
\phantomsection\label{\detokenize{pygpc:pygpc.grid.SparseGrid.order_sequence_type}}\pysigline{\sphinxbfcode{\sphinxupquote{order\_sequence\_type}}}
type of order sequence (‘lin’, ‘exp’) common: ‘exp’
\begin{quote}\begin{description}
\item[{Type}] \leavevmode
str

\end{description}\end{quote}

\end{fulllineitems}

\index{make\_grid (pygpc.grid.SparseGrid attribute)}

\begin{fulllineitems}
\phantomsection\label{\detokenize{pygpc:pygpc.grid.SparseGrid.make_grid}}\pysigline{\sphinxbfcode{\sphinxupquote{make\_grid}}}
boolean value to determine if to generate grid during initialization
\begin{quote}\begin{description}
\item[{Type}] \leavevmode
boolean

\end{description}\end{quote}

\end{fulllineitems}

\index{verbose (pygpc.grid.SparseGrid attribute)}

\begin{fulllineitems}
\phantomsection\label{\detokenize{pygpc:pygpc.grid.SparseGrid.verbose}}\pysigline{\sphinxbfcode{\sphinxupquote{verbose}}}
boolean value to determine if to print out the progress into the standard output
\begin{quote}\begin{description}
\item[{Type}] \leavevmode
bool

\end{description}\end{quote}

\end{fulllineitems}

\index{coords\_norm (pygpc.grid.SparseGrid attribute)}

\begin{fulllineitems}
\phantomsection\label{\detokenize{pygpc:pygpc.grid.SparseGrid.coords_norm}}\pysigline{\sphinxbfcode{\sphinxupquote{coords\_norm}}}
normalized {[}-1, 1{]} coordinates xi
\begin{quote}\begin{description}
\item[{Type}] \leavevmode
{[}N\_samples x dim{]} np.ndarray

\end{description}\end{quote}

\end{fulllineitems}

\index{weights (pygpc.grid.SparseGrid attribute)}

\begin{fulllineitems}
\phantomsection\label{\detokenize{pygpc:pygpc.grid.SparseGrid.weights}}\pysigline{\sphinxbfcode{\sphinxupquote{weights}}}
weights of the grid
\begin{quote}\begin{description}
\item[{Type}] \leavevmode
np.ndarray

\end{description}\end{quote}

\end{fulllineitems}

\index{coords (pygpc.grid.SparseGrid attribute)}

\begin{fulllineitems}
\phantomsection\label{\detokenize{pygpc:pygpc.grid.SparseGrid.coords}}\pysigline{\sphinxbfcode{\sphinxupquote{coords}}}
denormalized coordinates xi
\begin{quote}\begin{description}
\item[{Type}] \leavevmode
{[}N\_samples x dim{]} np.ndarray

\end{description}\end{quote}

\end{fulllineitems}

\index{level\_sequence (pygpc.grid.SparseGrid attribute)}

\begin{fulllineitems}
\phantomsection\label{\detokenize{pygpc:pygpc.grid.SparseGrid.level_sequence}}\pysigline{\sphinxbfcode{\sphinxupquote{level\_sequence}}}
list containing the levels
\begin{quote}\begin{description}
\item[{Type}] \leavevmode
list of int

\end{description}\end{quote}

\end{fulllineitems}

\index{order\_sequence (pygpc.grid.SparseGrid attribute)}

\begin{fulllineitems}
\phantomsection\label{\detokenize{pygpc:pygpc.grid.SparseGrid.order_sequence}}\pysigline{\sphinxbfcode{\sphinxupquote{order\_sequence}}}
list containing the polynomial order of the levels
\begin{quote}\begin{description}
\item[{Type}] \leavevmode
list of int

\end{description}\end{quote}

\end{fulllineitems}

\index{dim (pygpc.grid.SparseGrid attribute)}

\begin{fulllineitems}
\phantomsection\label{\detokenize{pygpc:pygpc.grid.SparseGrid.dim}}\pysigline{\sphinxbfcode{\sphinxupquote{dim}}}
number of uncertain parameters to process
\begin{quote}\begin{description}
\item[{Type}] \leavevmode
int

\end{description}\end{quote}

\end{fulllineitems}

\begin{quote}\begin{description}
\item[{Parameters}] \leavevmode\begin{itemize}
\item {} 
\sphinxstyleliteralstrong{\sphinxupquote{pdf\_type}} (\sphinxstyleliteralemphasis{\sphinxupquote{{[}}}\sphinxstyleliteralemphasis{\sphinxupquote{N\_vars}}\sphinxstyleliteralemphasis{\sphinxupquote{{]} }}\sphinxstyleliteralemphasis{\sphinxupquote{list of str}}) \textendash{} variable specific type of PDF (“beta”, “normal”)

\item {} 
\sphinxstyleliteralstrong{\sphinxupquote{grid\_type}} (\sphinxstyleliteralemphasis{\sphinxupquote{{[}}}\sphinxstyleliteralemphasis{\sphinxupquote{N\_vars}}\sphinxstyleliteralemphasis{\sphinxupquote{{]} }}\sphinxstyleliteralemphasis{\sphinxupquote{list of str}}) \textendash{} specify type of quadrature used to construct sparse grid (‘jacobi’, ‘hermite’, ‘cc’, ‘fejer2’)

\item {} 
\sphinxstyleliteralstrong{\sphinxupquote{grid\_shape}} (\sphinxstyleliteralemphasis{\sphinxupquote{{[}}}\sphinxstyleliteralemphasis{\sphinxupquote{2 x N\_vars}}\sphinxstyleliteralemphasis{\sphinxupquote{{]} }}\sphinxstyleliteralemphasis{\sphinxupquote{list of list of float}}) \textendash{} shape parameters of PDF
beta (jacobi):  {[}alpha, beta{]}
norm (hermite): {[}mean, std{]}

\item {} 
\sphinxstyleliteralstrong{\sphinxupquote{limits}} (\sphinxstyleliteralemphasis{\sphinxupquote{{[}}}\sphinxstyleliteralemphasis{\sphinxupquote{2 x N\_vars}}\sphinxstyleliteralemphasis{\sphinxupquote{{]} }}\sphinxstyleliteralemphasis{\sphinxupquote{list of list of float}}) \textendash{} upper and lower bounds of PDF
beta (jacobi):  {[}min, max{]}
norm (hermite): {[}0, 0{]} (unused)

\item {} 
\sphinxstyleliteralstrong{\sphinxupquote{level}} (\sphinxstyleliteralemphasis{\sphinxupquote{{[}}}\sphinxstyleliteralemphasis{\sphinxupquote{N\_vars}}\sphinxstyleliteralemphasis{\sphinxupquote{{]} }}\sphinxstyleliteralemphasis{\sphinxupquote{list of int}}) \textendash{} number of levels in each dimension

\item {} 
\sphinxstyleliteralstrong{\sphinxupquote{level\_max}} (\sphinxstyleliteralemphasis{\sphinxupquote{int}}) \textendash{} global combined level maximum

\item {} 
\sphinxstyleliteralstrong{\sphinxupquote{interaction\_order}} (\sphinxstyleliteralemphasis{\sphinxupquote{int}}) \textendash{} interaction order of parameters and grid, i.e. the grid points are lying between this number of dimensions

\item {} 
\sphinxstyleliteralstrong{\sphinxupquote{order\_sequence\_type}} (\sphinxstyleliteralemphasis{\sphinxupquote{str}}) \textendash{} type of order sequence (‘lin’, ‘exp’) common: ‘exp’

\item {} 
\sphinxstyleliteralstrong{\sphinxupquote{make\_grid}} (\sphinxstyleliteralemphasis{\sphinxupquote{boolean}}\sphinxstyleliteralemphasis{\sphinxupquote{, }}\sphinxstyleliteralemphasis{\sphinxupquote{optional}}\sphinxstyleliteralemphasis{\sphinxupquote{, }}\sphinxstyleliteralemphasis{\sphinxupquote{default=True}}) \textendash{} boolean value to determine if to generate grid during initialization

\item {} 
\sphinxstyleliteralstrong{\sphinxupquote{verbose}} (\sphinxstyleliteralemphasis{\sphinxupquote{bool}}\sphinxstyleliteralemphasis{\sphinxupquote{, }}\sphinxstyleliteralemphasis{\sphinxupquote{optional}}\sphinxstyleliteralemphasis{\sphinxupquote{, }}\sphinxstyleliteralemphasis{\sphinxupquote{default=True}}) \textendash{} boolean value to determine if to print out the progress into the standard output

\end{itemize}

\end{description}\end{quote}
\index{calc\_coords\_weights() (pygpc.grid.SparseGrid method)}

\begin{fulllineitems}
\phantomsection\label{\detokenize{pygpc:pygpc.grid.SparseGrid.calc_coords_weights}}\pysiglinewithargsret{\sphinxbfcode{\sphinxupquote{calc\_coords\_weights}}}{}{}
Find similar points in grid and formulate calculate a list containing these points.

\end{fulllineitems}

\index{calc\_grid() (pygpc.grid.SparseGrid method)}

\begin{fulllineitems}
\phantomsection\label{\detokenize{pygpc:pygpc.grid.SparseGrid.calc_grid}}\pysiglinewithargsret{\sphinxbfcode{\sphinxupquote{calc\_grid}}}{}{}
Calculate a cubature lookup table for knots and weights.

dl\_k, dl\_w = calc\_grid()
\begin{quote}\begin{description}
\item[{Returns}] \leavevmode
\begin{itemize}
\item {} 
\sphinxstylestrong{dl\_k} (\sphinxstyleemphasis{list of list of float}) \textendash{} cubature lookup table for knots

\item {} 
\sphinxstylestrong{dl\_w} (\sphinxstyleemphasis{list of list of float}) \textendash{} cubature lookup table for weights

\end{itemize}


\end{description}\end{quote}

\end{fulllineitems}

\index{calc\_l\_level() (pygpc.grid.SparseGrid method)}

\begin{fulllineitems}
\phantomsection\label{\detokenize{pygpc:pygpc.grid.SparseGrid.calc_l_level}}\pysiglinewithargsret{\sphinxbfcode{\sphinxupquote{calc\_l\_level}}}{}{}
Calculate the l-level needed for the Fejer grid type 2.

l\_level = calc\_l\_level()
\begin{quote}\begin{description}
\item[{Returns}] \leavevmode
\sphinxstylestrong{l\_level} \textendash{} multi indices filtered by level capacity and interaction order

\item[{Return type}] \leavevmode
np.ndarray

\end{description}\end{quote}

\end{fulllineitems}

\index{calc\_multi\_index\_lst() (pygpc.grid.SparseGrid method)}

\begin{fulllineitems}
\phantomsection\label{\detokenize{pygpc:pygpc.grid.SparseGrid.calc_multi_index_lst}}\pysiglinewithargsret{\sphinxbfcode{\sphinxupquote{calc\_multi\_index\_lst}}}{}{}
Calculate the multi index list needed for the calculation of the SparseGrid.

\end{fulllineitems}

\index{calc\_tensor\_products() (pygpc.grid.SparseGrid method)}

\begin{fulllineitems}
\phantomsection\label{\detokenize{pygpc:pygpc.grid.SparseGrid.calc_tensor_products}}\pysiglinewithargsret{\sphinxbfcode{\sphinxupquote{calc\_tensor\_products}}}{}{}
Calculate the tensor products of the knots and the weights.

dL\_k, dL\_w = calc\_tensor\_products()
\begin{quote}\begin{description}
\item[{Returns}] \leavevmode
\begin{itemize}
\item {} 
\sphinxstylestrong{dL\_k} (\sphinxstyleemphasis{np.ndarray}) \textendash{} tensor product of knots

\item {} 
\sphinxstylestrong{dL\_w} (\sphinxstyleemphasis{np.ndarray}) \textendash{} tensor product of weights

\end{itemize}


\end{description}\end{quote}

\end{fulllineitems}


\end{fulllineitems}

\index{TensorGrid (class in pygpc.grid)}

\begin{fulllineitems}
\phantomsection\label{\detokenize{pygpc:pygpc.grid.TensorGrid}}\pysiglinewithargsret{\sphinxbfcode{\sphinxupquote{class }}\sphinxcode{\sphinxupquote{pygpc.grid.}}\sphinxbfcode{\sphinxupquote{TensorGrid}}}{\emph{pdf\_type}, \emph{grid\_type}, \emph{grid\_shape}, \emph{limits}, \emph{N}}{}
Generate TensorGrid object instance.

TensorGrid(pdf\_type, grid\_type, grid\_shape, limits, N):
\index{pdf\_type (pygpc.grid.TensorGrid attribute)}

\begin{fulllineitems}
\phantomsection\label{\detokenize{pygpc:pygpc.grid.TensorGrid.pdf_type}}\pysigline{\sphinxbfcode{\sphinxupquote{pdf\_type}}}
variable specific type of PDF (“beta”, “normal”)
\begin{quote}\begin{description}
\item[{Type}] \leavevmode
{[}N\_vars{]} list of str

\end{description}\end{quote}

\end{fulllineitems}

\index{grid\_type (pygpc.grid.TensorGrid attribute)}

\begin{fulllineitems}
\phantomsection\label{\detokenize{pygpc:pygpc.grid.TensorGrid.grid_type}}\pysigline{\sphinxbfcode{\sphinxupquote{grid\_type}}}
specify type of quadrature used to construct sparse grid (‘jacobi’, ‘hermite’, ‘cc’, ‘fejer2’)
\begin{quote}\begin{description}
\item[{Type}] \leavevmode
{[}N\_vars{]} list of str

\end{description}\end{quote}

\end{fulllineitems}

\index{grid\_shape (pygpc.grid.TensorGrid attribute)}

\begin{fulllineitems}
\phantomsection\label{\detokenize{pygpc:pygpc.grid.TensorGrid.grid_shape}}\pysigline{\sphinxbfcode{\sphinxupquote{grid\_shape}}}
shape parameters of PDF
beta (jacobi):  {[}alpha, beta{]}
norm (hermite): {[}mean, std{]}
\begin{quote}\begin{description}
\item[{Type}] \leavevmode
{[}2 x N\_vars{]} list of list of float

\end{description}\end{quote}

\end{fulllineitems}

\index{limits (pygpc.grid.TensorGrid attribute)}

\begin{fulllineitems}
\phantomsection\label{\detokenize{pygpc:pygpc.grid.TensorGrid.limits}}\pysigline{\sphinxbfcode{\sphinxupquote{limits}}}
upper and lower bounds of PDF
beta (jacobi):  {[}min, max{]}
norm (hermite): {[}0, 0{]} (unused)
\begin{quote}\begin{description}
\item[{Type}] \leavevmode
{[}2 x N\_vars{]} list of list of float

\end{description}\end{quote}

\end{fulllineitems}

\index{N (pygpc.grid.TensorGrid attribute)}

\begin{fulllineitems}
\phantomsection\label{\detokenize{pygpc:pygpc.grid.TensorGrid.N}}\pysigline{\sphinxbfcode{\sphinxupquote{N}}}
number of nodes in each dimension
\begin{quote}\begin{description}
\item[{Type}] \leavevmode
{[}N\_vars{]} list of int

\end{description}\end{quote}

\end{fulllineitems}

\index{dim (pygpc.grid.TensorGrid attribute)}

\begin{fulllineitems}
\phantomsection\label{\detokenize{pygpc:pygpc.grid.TensorGrid.dim}}\pysigline{\sphinxbfcode{\sphinxupquote{dim}}}
number of uncertain parameters to process
\begin{quote}\begin{description}
\item[{Type}] \leavevmode
int

\end{description}\end{quote}

\end{fulllineitems}

\index{knots\_dim\_list (pygpc.grid.TensorGrid attribute)}

\begin{fulllineitems}
\phantomsection\label{\detokenize{pygpc:pygpc.grid.TensorGrid.knots_dim_list}}\pysigline{\sphinxbfcode{\sphinxupquote{knots\_dim\_list}}}
knots of grid in each dimension
\begin{quote}\begin{description}
\item[{Type}] \leavevmode
{[}dim{]} list of np.ndarray

\end{description}\end{quote}

\end{fulllineitems}

\index{weights\_dim\_list (pygpc.grid.TensorGrid attribute)}

\begin{fulllineitems}
\phantomsection\label{\detokenize{pygpc:pygpc.grid.TensorGrid.weights_dim_list}}\pysigline{\sphinxbfcode{\sphinxupquote{weights\_dim\_list}}}
weights of grid in each dimension
\begin{quote}\begin{description}
\item[{Type}] \leavevmode
{[}dim{]} list of np.ndarray

\end{description}\end{quote}

\end{fulllineitems}

\index{coords\_norm (pygpc.grid.TensorGrid attribute)}

\begin{fulllineitems}
\phantomsection\label{\detokenize{pygpc:pygpc.grid.TensorGrid.coords_norm}}\pysigline{\sphinxbfcode{\sphinxupquote{coords\_norm}}}
normalized {[}-1, 1{]} coordinates xi
\begin{quote}\begin{description}
\item[{Type}] \leavevmode
{[}N\_samples x dim{]} np.ndarray

\end{description}\end{quote}

\end{fulllineitems}

\index{weights (pygpc.grid.TensorGrid attribute)}

\begin{fulllineitems}
\phantomsection\label{\detokenize{pygpc:pygpc.grid.TensorGrid.weights}}\pysigline{\sphinxbfcode{\sphinxupquote{weights}}}
weights of the grid
\begin{quote}\begin{description}
\item[{Type}] \leavevmode
np.ndarray

\end{description}\end{quote}

\end{fulllineitems}

\index{coords (pygpc.grid.TensorGrid attribute)}

\begin{fulllineitems}
\phantomsection\label{\detokenize{pygpc:pygpc.grid.TensorGrid.coords}}\pysigline{\sphinxbfcode{\sphinxupquote{coords}}}
denormalized coordinates xi
\begin{quote}\begin{description}
\item[{Type}] \leavevmode
{[}N\_samples x dim{]} np.ndarray

\end{description}\end{quote}

\end{fulllineitems}

\begin{quote}\begin{description}
\item[{Parameters}] \leavevmode\begin{itemize}
\item {} 
\sphinxstyleliteralstrong{\sphinxupquote{pdf\_type}} (\sphinxstyleliteralemphasis{\sphinxupquote{{[}}}\sphinxstyleliteralemphasis{\sphinxupquote{N\_vars}}\sphinxstyleliteralemphasis{\sphinxupquote{{]} }}\sphinxstyleliteralemphasis{\sphinxupquote{list of str}}) \textendash{} variable specific type of PDF (“beta”, “normal”)

\item {} 
\sphinxstyleliteralstrong{\sphinxupquote{grid\_type}} (\sphinxstyleliteralemphasis{\sphinxupquote{{[}}}\sphinxstyleliteralemphasis{\sphinxupquote{N\_vars}}\sphinxstyleliteralemphasis{\sphinxupquote{{]} }}\sphinxstyleliteralemphasis{\sphinxupquote{list of str}}) \textendash{} specify type of quadrature used to construct sparse grid (‘jacobi’, ‘hermite’, ‘cc’, ‘fejer2’)

\item {} 
\sphinxstyleliteralstrong{\sphinxupquote{grid\_shape}} (\sphinxstyleliteralemphasis{\sphinxupquote{{[}}}\sphinxstyleliteralemphasis{\sphinxupquote{2 x N\_vars}}\sphinxstyleliteralemphasis{\sphinxupquote{{]} }}\sphinxstyleliteralemphasis{\sphinxupquote{list of list of float}}) \textendash{} shape parameters of PDF
beta (jacobi):  {[}alpha, beta{]}
norm (hermite): {[}mean, std{]}

\item {} 
\sphinxstyleliteralstrong{\sphinxupquote{limits}} (\sphinxstyleliteralemphasis{\sphinxupquote{{[}}}\sphinxstyleliteralemphasis{\sphinxupquote{2 x N\_vars}}\sphinxstyleliteralemphasis{\sphinxupquote{{]} }}\sphinxstyleliteralemphasis{\sphinxupquote{list of list of float}}) \textendash{} upper and lower bounds of PDF
beta (jacobi):  {[}min, max{]}
norm (hermite): {[}0, 0{]} (unused)

\item {} 
\sphinxstyleliteralstrong{\sphinxupquote{N}} (\sphinxstyleliteralemphasis{\sphinxupquote{{[}}}\sphinxstyleliteralemphasis{\sphinxupquote{N\_vars}}\sphinxstyleliteralemphasis{\sphinxupquote{{]} }}\sphinxstyleliteralemphasis{\sphinxupquote{list of int}}) \textendash{} number of nodes in each dimension

\end{itemize}

\end{description}\end{quote}

\end{fulllineitems}

\index{get\_denormalized\_coordinates() (in module pygpc.grid)}

\begin{fulllineitems}
\phantomsection\label{\detokenize{pygpc:pygpc.grid.get_denormalized_coordinates}}\pysiglinewithargsret{\sphinxcode{\sphinxupquote{pygpc.grid.}}\sphinxbfcode{\sphinxupquote{get\_denormalized\_coordinates}}}{\emph{coords\_norm}, \emph{pdf\_type}, \emph{grid\_shape}, \emph{limits}}{}
Denormalize grid from standardized ({[}-1, 1{]} except hermite) to original parameter space for simulations.

coords = get\_denormalized\_coordinates(coords\_norm, pdf\_type, grid\_shape, limits)
\begin{quote}\begin{description}
\item[{Parameters}] \leavevmode\begin{itemize}
\item {} 
\sphinxstyleliteralstrong{\sphinxupquote{pdf\_type}} (\sphinxstyleliteralemphasis{\sphinxupquote{{[}}}\sphinxstyleliteralemphasis{\sphinxupquote{dim}}\sphinxstyleliteralemphasis{\sphinxupquote{{]} }}\sphinxstyleliteralemphasis{\sphinxupquote{list of str}}) \textendash{} type of pdf ‘beta’ or ‘norm’

\item {} 
\sphinxstyleliteralstrong{\sphinxupquote{grid\_shape}} (\sphinxstyleliteralemphasis{\sphinxupquote{{[}}}\sphinxstyleliteralemphasis{\sphinxupquote{2 x N\_vars}}\sphinxstyleliteralemphasis{\sphinxupquote{{]} }}\sphinxstyleliteralemphasis{\sphinxupquote{list of list of float}}) \textendash{} shape parameters of PDF
beta (jacobi):  {[}alpha, beta{]}
norm (hermite): {[}mean, std{]}

\item {} 
\sphinxstyleliteralstrong{\sphinxupquote{limits}} (\sphinxstyleliteralemphasis{\sphinxupquote{{[}}}\sphinxstyleliteralemphasis{\sphinxupquote{2 x N\_vars}}\sphinxstyleliteralemphasis{\sphinxupquote{{]} }}\sphinxstyleliteralemphasis{\sphinxupquote{list of list of float}}) \textendash{} upper and lower bounds of PDF
beta (jacobi):  {[}min, max{]}
norm (hermite): {[}0, 0{]} (unused)

\item {} 
\sphinxstyleliteralstrong{\sphinxupquote{coords\_norm}} (\sphinxstyleliteralemphasis{\sphinxupquote{{[}}}\sphinxstyleliteralemphasis{\sphinxupquote{N\_samples x dim}}\sphinxstyleliteralemphasis{\sphinxupquote{{]} }}\sphinxstyleliteralemphasis{\sphinxupquote{np.ndarray}}) \textendash{} normalized {[}-1, 1{]} coordinates xi

\end{itemize}

\item[{Returns}] \leavevmode
\sphinxstylestrong{coords} \textendash{} denormalized coordinates xi

\item[{Return type}] \leavevmode
{[}N\_samples x dim{]} np.ndarray

\end{description}\end{quote}

\end{fulllineitems}

\index{get\_normalized\_coordinates() (in module pygpc.grid)}

\begin{fulllineitems}
\phantomsection\label{\detokenize{pygpc:pygpc.grid.get_normalized_coordinates}}\pysiglinewithargsret{\sphinxcode{\sphinxupquote{pygpc.grid.}}\sphinxbfcode{\sphinxupquote{get\_normalized\_coordinates}}}{\emph{coords}, \emph{pdf\_type}, \emph{grid\_shape}, \emph{limits}}{}
Normalize grid from original parameter (except hermite) to standardized ({[}-1, 1{]} space for simulations.

coords\_norm = get\_normalized\_coordinates(coords, pdf\_type, grid\_shape, limits)
\begin{quote}\begin{description}
\item[{Parameters}] \leavevmode\begin{itemize}
\item {} 
\sphinxstyleliteralstrong{\sphinxupquote{pdf\_type}} (\sphinxstyleliteralemphasis{\sphinxupquote{{[}}}\sphinxstyleliteralemphasis{\sphinxupquote{dim}}\sphinxstyleliteralemphasis{\sphinxupquote{{]} }}\sphinxstyleliteralemphasis{\sphinxupquote{list of str}}) \textendash{} type of pdf ‘beta’ or ‘norm’

\item {} 
\sphinxstyleliteralstrong{\sphinxupquote{grid\_shape}} (\sphinxstyleliteralemphasis{\sphinxupquote{{[}}}\sphinxstyleliteralemphasis{\sphinxupquote{2 x N\_vars}}\sphinxstyleliteralemphasis{\sphinxupquote{{]} }}\sphinxstyleliteralemphasis{\sphinxupquote{list of list of float}}) \textendash{} shape parameters of PDF
beta (jacobi):  {[}alpha, beta{]}
norm (hermite): {[}mean, std{]}

\item {} 
\sphinxstyleliteralstrong{\sphinxupquote{limits}} (\sphinxstyleliteralemphasis{\sphinxupquote{{[}}}\sphinxstyleliteralemphasis{\sphinxupquote{2 x N\_vars}}\sphinxstyleliteralemphasis{\sphinxupquote{{]} }}\sphinxstyleliteralemphasis{\sphinxupquote{list of list of float}}) \textendash{} upper and lower bounds of PDF
beta (jacobi):  {[}min, max{]}
norm (hermite): {[}0, 0{]} (unused)

\item {} 
\sphinxstyleliteralstrong{\sphinxupquote{coords}} (\sphinxstyleliteralemphasis{\sphinxupquote{{[}}}\sphinxstyleliteralemphasis{\sphinxupquote{N\_samples x dim}}\sphinxstyleliteralemphasis{\sphinxupquote{{]} }}\sphinxstyleliteralemphasis{\sphinxupquote{np.ndarray}}) \textendash{} denormalized coordinates xi

\end{itemize}

\item[{Returns}] \leavevmode
\sphinxstylestrong{coords\_norm} \textendash{} normalized {[}-1, 1{]} coordinates xi

\item[{Return type}] \leavevmode
{[}N\_samples x dim{]} np.ndarray

\end{description}\end{quote}

\end{fulllineitems}

\index{get\_quadrature\_clenshaw\_curtis\_1d() (in module pygpc.grid)}

\begin{fulllineitems}
\phantomsection\label{\detokenize{pygpc:pygpc.grid.get_quadrature_clenshaw_curtis_1d}}\pysiglinewithargsret{\sphinxcode{\sphinxupquote{pygpc.grid.}}\sphinxbfcode{\sphinxupquote{get\_quadrature\_clenshaw\_curtis\_1d}}}{\emph{N}}{}
Get the Clenshaw Curtis nodes and weights.

knots, weights = get\_quadrature\_clenshaw\_curtis\_1d(N)
\begin{quote}\begin{description}
\item[{Parameters}] \leavevmode
\sphinxstyleliteralstrong{\sphinxupquote{N}} (\sphinxstyleliteralemphasis{\sphinxupquote{int}}) \textendash{} number of knots

\item[{Returns}] \leavevmode
\begin{itemize}
\item {} 
\sphinxstylestrong{knots} (\sphinxstyleemphasis{np.ndarray}) \textendash{} knots of the grid

\item {} 
\sphinxstylestrong{weights} (\sphinxstyleemphasis{np.ndarray}) \textendash{} weights of the grid

\end{itemize}


\end{description}\end{quote}

\end{fulllineitems}

\index{get\_quadrature\_fejer1\_1d() (in module pygpc.grid)}

\begin{fulllineitems}
\phantomsection\label{\detokenize{pygpc:pygpc.grid.get_quadrature_fejer1_1d}}\pysiglinewithargsret{\sphinxcode{\sphinxupquote{pygpc.grid.}}\sphinxbfcode{\sphinxupquote{get\_quadrature\_fejer1\_1d}}}{\emph{N}}{}
Computes the Fejer type 1 nodes and weights.

This method uses a direct approach. The paper by Waldvogel
exhibits a more efficient approach using Fourier transforms.

Reference:
Philip Davis, Philip Rabinowitz,
Methods of Numerical Integration,
Second Edition,
Dover, 2007,
ISBN: 0486453391 Titel anhand dieser ISBN in Citavi-Projekt übernehmen,
LC: QA299.3.D28.

Walter Gautschi,
Numerical Quadrature in the Presence of a Singularity,
SIAM Journal on Numerical Analysis,
Volume 4, Number 3, 1967, pages 357-362.

Joerg Waldvogel,
Fast Construction of the Fejer and Clenshaw-Curtis Quadrature Rules,
BIT Numerical Mathematics,
Volume 43, Number 1, 2003, pages 1-18.

knots, weights = get\_quadrature\_fejer1\_1d(N)
\begin{quote}\begin{description}
\item[{Parameters}] \leavevmode
\sphinxstyleliteralstrong{\sphinxupquote{N}} (\sphinxstyleliteralemphasis{\sphinxupquote{int}}) \textendash{} number of knots

\item[{Returns}] \leavevmode
\begin{itemize}
\item {} 
\sphinxstylestrong{knots} (\sphinxstyleemphasis{np.ndarray}) \textendash{} knots of the grid

\item {} 
\sphinxstylestrong{weights} (\sphinxstyleemphasis{np.ndarray}) \textendash{} weights of the grid

\end{itemize}


\end{description}\end{quote}

\end{fulllineitems}

\index{get\_quadrature\_fejer2\_1d() (in module pygpc.grid)}

\begin{fulllineitems}
\phantomsection\label{\detokenize{pygpc:pygpc.grid.get_quadrature_fejer2_1d}}\pysiglinewithargsret{\sphinxcode{\sphinxupquote{pygpc.grid.}}\sphinxbfcode{\sphinxupquote{get\_quadrature\_fejer2\_1d}}}{\emph{N}}{}
Computes the Fejer type 2 nodes and weights (Clenshaw Curtis without boundary nodes).

This method uses a direct approach. The paper by Waldvogel
exhibits a more efficient approach using Fourier transforms.

Reference:
Philip Davis, Philip Rabinowitz,
Methods of Numerical Integration,
Second Edition,
Dover, 2007,
ISBN: 0486453391 Titel anhand dieser ISBN in Citavi-Projekt übernehmen,
LC: QA299.3.D28.

Walter Gautschi,
Numerical Quadrature in the Presence of a Singularity,
SIAM Journal on Numerical Analysis,
Volume 4, Number 3, 1967, pages 357-362.

Joerg Waldvogel,
Fast Construction of the Fejer and Clenshaw-Curtis Quadrature Rules,
BIT Numerical Mathematics,
Volume 43, Number 1, 2003, pages 1-18.

knots, weights = get\_quadrature\_fejer2\_1d(N)
\begin{quote}\begin{description}
\item[{Parameters}] \leavevmode
\sphinxstyleliteralstrong{\sphinxupquote{N}} (\sphinxstyleliteralemphasis{\sphinxupquote{int}}) \textendash{} number of knots

\item[{Returns}] \leavevmode
\begin{itemize}
\item {} 
\sphinxstylestrong{knots} (\sphinxstyleemphasis{np.ndarray}) \textendash{} knots of the grid

\item {} 
\sphinxstylestrong{weights} (\sphinxstyleemphasis{np.ndarray}) \textendash{} weights of the grid

\end{itemize}


\end{description}\end{quote}

\end{fulllineitems}

\index{get\_quadrature\_hermite\_1d() (in module pygpc.grid)}

\begin{fulllineitems}
\phantomsection\label{\detokenize{pygpc:pygpc.grid.get_quadrature_hermite_1d}}\pysiglinewithargsret{\sphinxcode{\sphinxupquote{pygpc.grid.}}\sphinxbfcode{\sphinxupquote{get\_quadrature\_hermite\_1d}}}{\emph{N}}{}
Get knots and weights of Hermite polynomials (normal distribution).

knots, weights = get\_quadrature\_hermite\_1d(N)
\begin{quote}\begin{description}
\item[{Parameters}] \leavevmode
\sphinxstyleliteralstrong{\sphinxupquote{N}} (\sphinxstyleliteralemphasis{\sphinxupquote{int}}) \textendash{} number of knots

\item[{Returns}] \leavevmode
\begin{itemize}
\item {} 
\sphinxstylestrong{knots} (\sphinxstyleemphasis{np.ndarray}) \textendash{} knots of the grid

\item {} 
\sphinxstylestrong{weights} (\sphinxstyleemphasis{np.ndarray}) \textendash{} weights of the grid

\end{itemize}


\end{description}\end{quote}

\end{fulllineitems}

\index{get\_quadrature\_jacobi\_1d() (in module pygpc.grid)}

\begin{fulllineitems}
\phantomsection\label{\detokenize{pygpc:pygpc.grid.get_quadrature_jacobi_1d}}\pysiglinewithargsret{\sphinxcode{\sphinxupquote{pygpc.grid.}}\sphinxbfcode{\sphinxupquote{get\_quadrature\_jacobi\_1d}}}{\emph{N}, \emph{b}, \emph{a}}{}
Get knots and weights of Jacobi polynomials (beta distribution).

knots, weights = get\_quadrature\_jacobi\_1d(N, b, a)
\begin{quote}\begin{description}
\item[{Parameters}] \leavevmode\begin{itemize}
\item {} 
\sphinxstyleliteralstrong{\sphinxupquote{N}} (\sphinxstyleliteralemphasis{\sphinxupquote{int}}) \textendash{} number of knots

\item {} 
\sphinxstyleliteralstrong{\sphinxupquote{a}} (\sphinxstyleliteralemphasis{\sphinxupquote{float}}) \textendash{} lower limit of quadrature coefficients

\item {} 
\sphinxstyleliteralstrong{\sphinxupquote{b}} (\sphinxstyleliteralemphasis{\sphinxupquote{float}}) \textendash{} upper limit of quadrature coefficients

\end{itemize}

\item[{Returns}] \leavevmode
\begin{itemize}
\item {} 
\sphinxstylestrong{knots} (\sphinxstyleemphasis{np.ndarray}) \textendash{} knots of the grid

\item {} 
\sphinxstylestrong{weights} (\sphinxstyleemphasis{np.ndarray}) \textendash{} weights of the grid

\end{itemize}


\end{description}\end{quote}

\end{fulllineitems}

\index{get\_quadrature\_patterson\_1d() (in module pygpc.grid)}

\begin{fulllineitems}
\phantomsection\label{\detokenize{pygpc:pygpc.grid.get_quadrature_patterson_1d}}\pysiglinewithargsret{\sphinxcode{\sphinxupquote{pygpc.grid.}}\sphinxbfcode{\sphinxupquote{get\_quadrature\_patterson\_1d}}}{\emph{N}}{}
Computes the nested Gauss-Patterson nodes and weights for N = 1,3,7,15,31.

knots, weights = get\_quadrature\_patterson\_1d(N)
\begin{quote}\begin{description}
\item[{Parameters}] \leavevmode
\sphinxstyleliteralstrong{\sphinxupquote{N}} (\sphinxstyleliteralemphasis{\sphinxupquote{int}}) \textendash{} number of knots
possible values: 1, 3, 7, 15, 31

\item[{Returns}] \leavevmode
\begin{itemize}
\item {} 
\sphinxstylestrong{knots} (\sphinxstyleemphasis{np.ndarray}) \textendash{} knots of the grid

\item {} 
\sphinxstylestrong{weights} (\sphinxstyleemphasis{np.ndarray}) \textendash{} weights of the grid

\end{itemize}


\end{description}\end{quote}

\end{fulllineitems}



\section{pygpc.misc module}
\label{\detokenize{pygpc:module-pygpc.misc}}\label{\detokenize{pygpc:pygpc-misc-module}}\index{pygpc.misc (module)}
Functions and classes that provide data and methods with general usage in the pygpc package
\index{NoDaemonProcess (class in pygpc.misc)}

\begin{fulllineitems}
\phantomsection\label{\detokenize{pygpc:pygpc.misc.NoDaemonProcess}}\pysiglinewithargsret{\sphinxbfcode{\sphinxupquote{class }}\sphinxcode{\sphinxupquote{pygpc.misc.}}\sphinxbfcode{\sphinxupquote{NoDaemonProcess}}}{\emph{group=None}, \emph{target=None}, \emph{name=None}, \emph{args=()}, \emph{kwargs=\{\}}}{}
Bases: \sphinxcode{\sphinxupquote{multiprocessing.process.Process}}

Helper class to create a non daemonic process.
From \sphinxurl{https://stackoverflow.com/questions/6974695/python-process-pool-non-daemonic}
make ‘daemon’ attribute always return False
\index{daemon (pygpc.misc.NoDaemonProcess attribute)}

\begin{fulllineitems}
\phantomsection\label{\detokenize{pygpc:pygpc.misc.NoDaemonProcess.daemon}}\pysigline{\sphinxbfcode{\sphinxupquote{daemon}}}
\end{fulllineitems}


\end{fulllineitems}

\index{NonDaemonicPool (class in pygpc.misc)}

\begin{fulllineitems}
\phantomsection\label{\detokenize{pygpc:pygpc.misc.NonDaemonicPool}}\pysiglinewithargsret{\sphinxbfcode{\sphinxupquote{class }}\sphinxcode{\sphinxupquote{pygpc.misc.}}\sphinxbfcode{\sphinxupquote{NonDaemonicPool}}}{\emph{processes=None}, \emph{initializer=None}, \emph{initargs=()}, \emph{maxtasksperchild=None}}{}
Bases: \sphinxcode{\sphinxupquote{multiprocessing.pool.Pool}}

Helper class to create a non daemonic pool.
We sub-class multiprocessing.pool.Pool instead of multiprocessing.Pool
because the latter is only a wrapper function, not a proper class.
\index{Process (pygpc.misc.NonDaemonicPool attribute)}

\begin{fulllineitems}
\phantomsection\label{\detokenize{pygpc:pygpc.misc.NonDaemonicPool.Process}}\pysigline{\sphinxbfcode{\sphinxupquote{Process}}}
alias of {\hyperref[\detokenize{pygpc:pygpc.misc.NoDaemonProcess}]{\sphinxcrossref{\sphinxcode{\sphinxupquote{NoDaemonProcess}}}}}

\end{fulllineitems}


\end{fulllineitems}

\index{display\_fancy\_bar() (in module pygpc.misc)}

\begin{fulllineitems}
\phantomsection\label{\detokenize{pygpc:pygpc.misc.display_fancy_bar}}\pysiglinewithargsret{\sphinxcode{\sphinxupquote{pygpc.misc.}}\sphinxbfcode{\sphinxupquote{display\_fancy\_bar}}}{\emph{text}, \emph{i}, \emph{n\_i}, \emph{more\_text=None}}{}
Display a simple progess bar.
Call for each iteration and start with i=1.
\begin{quote}\begin{description}
\item[{Parameters}] \leavevmode\begin{itemize}
\item {} 
\sphinxstyleliteralstrong{\sphinxupquote{text}} (\sphinxstyleliteralemphasis{\sphinxupquote{str}}) \textendash{} text to display in front of actual iteration

\item {} 
\sphinxstyleliteralstrong{\sphinxupquote{i}} (\sphinxstyleliteralemphasis{\sphinxupquote{str}}\sphinxstyleliteralemphasis{\sphinxupquote{ or }}\sphinxstyleliteralemphasis{\sphinxupquote{int}}) \textendash{} actual iteration

\item {} 
\sphinxstyleliteralstrong{\sphinxupquote{n\_i}} (\sphinxstyleliteralemphasis{\sphinxupquote{int}}) \textendash{} number of iterations

\item {} 
\sphinxstyleliteralstrong{\sphinxupquote{more\_text}} (\sphinxstyleliteralemphasis{\sphinxupquote{str}}\sphinxstyleliteralemphasis{\sphinxupquote{, }}\sphinxstyleliteralemphasis{\sphinxupquote{optional}}\sphinxstyleliteralemphasis{\sphinxupquote{, }}\sphinxstyleliteralemphasis{\sphinxupquote{default=None}}) \textendash{} text that displayed at an extra line.

\end{itemize}

\end{description}\end{quote}
\subsubsection*{Examples}

fancy\_bar(‘Run’,7,10):
Run 07 from 10 {[}================================        {]} 70\%

fancy\_bar(Run,9,10,’Some more text’):
Some more text
Run 09 from 10 {[}======================================= {]} 90\%

\end{fulllineitems}

\index{get\_array\_unique\_rows() (in module pygpc.misc)}

\begin{fulllineitems}
\phantomsection\label{\detokenize{pygpc:pygpc.misc.get_array_unique_rows}}\pysiglinewithargsret{\sphinxcode{\sphinxupquote{pygpc.misc.}}\sphinxbfcode{\sphinxupquote{get\_array\_unique\_rows}}}{\emph{array}}{}
Compute unique rows of array and delete rows that are linearly dependent.

unique = get\_array\_unique\_rows(array)
\begin{quote}\begin{description}
\item[{Parameters}] \leavevmode
\sphinxstyleliteralstrong{\sphinxupquote{array}} (\sphinxstyleliteralemphasis{\sphinxupquote{np.ndarray}}) \textendash{} matrix with k linearly dependent rows

\item[{Returns}] \leavevmode
\sphinxstylestrong{unique} \textendash{} matrix without k linearly dependent rows

\item[{Return type}] \leavevmode
np.ndarray

\end{description}\end{quote}

\end{fulllineitems}

\index{get\_betapdf\_fit() (in module pygpc.misc)}

\begin{fulllineitems}
\phantomsection\label{\detokenize{pygpc:pygpc.misc.get_betapdf_fit}}\pysiglinewithargsret{\sphinxcode{\sphinxupquote{pygpc.misc.}}\sphinxbfcode{\sphinxupquote{get\_betapdf\_fit}}}{\emph{data}, \emph{beta\_tolerance=0}, \emph{uni\_intervall=0}}{}
Fit data to a beta distribution in the interval {[}a, b{]}.

beta\_parameters, moments, p\_value, uni\_parameters = get\_betapdf\_fit(data, beta\_tolerance=0, uni\_intervall=0)
\begin{quote}\begin{description}
\item[{Parameters}] \leavevmode\begin{itemize}
\item {} 
\sphinxstyleliteralstrong{\sphinxupquote{data}} (\sphinxstyleliteralemphasis{\sphinxupquote{np.ndarray}}) \textendash{} data to fit

\item {} 
\sphinxstyleliteralstrong{\sphinxupquote{beta\_tolerance}} (\sphinxstyleliteralemphasis{\sphinxupquote{float}}\sphinxstyleliteralemphasis{\sphinxupquote{, }}\sphinxstyleliteralemphasis{\sphinxupquote{optional}}\sphinxstyleliteralemphasis{\sphinxupquote{, }}\sphinxstyleliteralemphasis{\sphinxupquote{default=0}}) \textendash{} tolerance interval to calculate the bounds of beta distribution
from observed data, e.g. 0.2 (+-20\% tolerance)

\item {} 
\sphinxstyleliteralstrong{\sphinxupquote{uni\_intervall}} (\sphinxstyleliteralemphasis{\sphinxupquote{float}}\sphinxstyleliteralemphasis{\sphinxupquote{, }}\sphinxstyleliteralemphasis{\sphinxupquote{optional}}\sphinxstyleliteralemphasis{\sphinxupquote{, }}\sphinxstyleliteralemphasis{\sphinxupquote{default=0}}) \textendash{} uniform distribution interval defined as fraction of
beta distribution interval
range: {[}0…1{]}, e.g. 0.90 (90\%)

\end{itemize}

\item[{Returns}] \leavevmode
\begin{itemize}
\item {} 
\sphinxstylestrong{beta\_parameters} (\sphinxstyleemphasis{{[}4{]} list of float}) \textendash{} 2 shape parameters and limits
{[}p, q, a, b{]}

\item {} 
\sphinxstylestrong{moments} (\sphinxstyleemphasis{{[}4{]} list of float}) \textendash{} {[}data\_mean, data\_std, beta\_mean, beta\_std{]}

\item {} 
\sphinxstylestrong{p\_value} (\sphinxstyleemphasis{float}) \textendash{} p-value of the Kolmogorov Smirnov test

\item {} 
\sphinxstylestrong{uni\_parameters} (\sphinxstyleemphasis{{[}2{]} list of float}) \textendash{} limits a and b
{[}a, b{]}

\end{itemize}


\end{description}\end{quote}

\end{fulllineitems}

\index{get\_cartesian\_product() (in module pygpc.misc)}

\begin{fulllineitems}
\phantomsection\label{\detokenize{pygpc:pygpc.misc.get_cartesian_product}}\pysiglinewithargsret{\sphinxcode{\sphinxupquote{pygpc.misc.}}\sphinxbfcode{\sphinxupquote{get\_cartesian\_product}}}{\emph{array\_list}, \emph{cartesian\_product=None}}{}
Generate a cartesian product of input arrays.

cartesian\_product = get\_cartesian\_product(array\_list, cartesian\_product=None)
\begin{quote}\begin{description}
\item[{Parameters}] \leavevmode\begin{itemize}
\item {} 
\sphinxstyleliteralstrong{\sphinxupquote{array\_list}} (\sphinxstyleliteralemphasis{\sphinxupquote{list of np.ndarray}}) \textendash{} arrays to form the cartesian product with

\item {} 
\sphinxstyleliteralstrong{\sphinxupquote{cartesian\_product}} (\sphinxstyleliteralemphasis{\sphinxupquote{np.ndarray}}) \textendash{} array to write the cartesian product

\end{itemize}

\item[{Returns}] \leavevmode
\sphinxstylestrong{cartesian\_product} \textendash{} array to write the cartesian product
(M, len(arrays))

\item[{Return type}] \leavevmode
np.ndarray

\end{description}\end{quote}
\subsubsection*{Examples}

cartesian(({[}1, 2, 3{]}, {[}4, 5{]}, {[}6, 7{]})) =
\begin{description}
\item[{array({[}{[}1, 4, 6{]},}] \leavevmode
{[}1, 4, 7{]},
{[}1, 5, 6{]},
{[}1, 5, 7{]},
{[}2, 4, 6{]},
{[}2, 4, 7{]},
{[}2, 5, 6{]},
{[}2, 5, 7{]},
{[}3, 4, 6{]},
{[}3, 4, 7{]},
{[}3, 5, 6{]},
{[}3, 5, 7{]}{]})

\end{description}

\end{fulllineitems}

\index{get\_list\_multi\_delete() (in module pygpc.misc)}

\begin{fulllineitems}
\phantomsection\label{\detokenize{pygpc:pygpc.misc.get_list_multi_delete}}\pysiglinewithargsret{\sphinxcode{\sphinxupquote{pygpc.misc.}}\sphinxbfcode{\sphinxupquote{get\_list\_multi\_delete}}}{\emph{input\_list}, \emph{index}}{}
Delete multiple entries from list.

input\_list = get\_list\_multi\_delete(input\_list, index)
\begin{quote}\begin{description}
\item[{Parameters}] \leavevmode\begin{itemize}
\item {} 
\sphinxstyleliteralstrong{\sphinxupquote{input\_list}} (\sphinxstyleliteralemphasis{\sphinxupquote{list}}) \textendash{} simple list

\item {} 
\sphinxstyleliteralstrong{\sphinxupquote{index}} (\sphinxstyleliteralemphasis{\sphinxupquote{list of integer}}) \textendash{} list of indices to delete

\end{itemize}

\item[{Returns}] \leavevmode
\sphinxstylestrong{input\_list} \textendash{} input list without entries specified in index

\item[{Return type}] \leavevmode
list

\end{description}\end{quote}

\end{fulllineitems}

\index{get\_multi\_indices() (in module pygpc.misc)}

\begin{fulllineitems}
\phantomsection\label{\detokenize{pygpc:pygpc.misc.get_multi_indices}}\pysiglinewithargsret{\sphinxcode{\sphinxupquote{pygpc.misc.}}\sphinxbfcode{\sphinxupquote{get\_multi\_indices}}}{\emph{length}, \emph{max\_order}}{}
Computes all multi-indices with a maximum overall order of max\_order.

multi\_indices = get\_multi\_indices(length, max\_order)
\begin{quote}\begin{description}
\item[{Parameters}] \leavevmode\begin{itemize}
\item {} 
\sphinxstyleliteralstrong{\sphinxupquote{length}} (\sphinxstyleliteralemphasis{\sphinxupquote{int}}) \textendash{} length of multi-index tuples

\item {} 
\sphinxstyleliteralstrong{\sphinxupquote{max\_order}} (\sphinxstyleliteralemphasis{\sphinxupquote{int}}) \textendash{} maximum overall interaction order

\end{itemize}

\item[{Returns}] \leavevmode
\sphinxstylestrong{multi\_indices} \textendash{} matrix of multi-indices

\item[{Return type}] \leavevmode
np.ndarray

\end{description}\end{quote}

\end{fulllineitems}

\index{get\_normalized\_rms() (in module pygpc.misc)}

\begin{fulllineitems}
\phantomsection\label{\detokenize{pygpc:pygpc.misc.get_normalized_rms}}\pysiglinewithargsret{\sphinxcode{\sphinxupquote{pygpc.misc.}}\sphinxbfcode{\sphinxupquote{get\_normalized\_rms}}}{\emph{array}, \emph{ref}}{}
Determine the normalized root mean square deviation between input data and reference data in {[}\%{]}.

normalized\_rms = get\_normalized\_rms(array, ref)
\begin{quote}\begin{description}
\item[{Parameters}] \leavevmode\begin{itemize}
\item {} 
\sphinxstyleliteralstrong{\sphinxupquote{array}} (\sphinxstyleliteralemphasis{\sphinxupquote{np.ndarray}}) \textendash{} input data {[} (x), y0, y1, y2 … {]}

\item {} 
\sphinxstyleliteralstrong{\sphinxupquote{ref}} (\sphinxstyleliteralemphasis{\sphinxupquote{np.ndarray}}) \textendash{} reference data {[} (xref), yref {]}
if ref is 1D, all sizes have to match

\end{itemize}

\item[{Returns}] \leavevmode
\sphinxstylestrong{normalized\_rms} \textendash{} normalized root mean square deviation

\item[{Return type}] \leavevmode
float

\end{description}\end{quote}

\end{fulllineitems}

\index{get\_num\_coeffs() (in module pygpc.misc)}

\begin{fulllineitems}
\phantomsection\label{\detokenize{pygpc:pygpc.misc.get_num_coeffs}}\pysiglinewithargsret{\sphinxcode{\sphinxupquote{pygpc.misc.}}\sphinxbfcode{\sphinxupquote{get\_num\_coeffs}}}{\emph{order}, \emph{dim}}{}
Calculate the number of PCE coefficients by the used order and dimension.

num\_coeffs = (order+dim)! / (order! * dim!)

num\_coeffs = get\_num\_coeffs(order , dim)
\begin{quote}\begin{description}
\item[{Parameters}] \leavevmode\begin{itemize}
\item {} 
\sphinxstyleliteralstrong{\sphinxupquote{order}} (\sphinxstyleliteralemphasis{\sphinxupquote{int}}) \textendash{} global order of expansion

\item {} 
\sphinxstyleliteralstrong{\sphinxupquote{dim}} (\sphinxstyleliteralemphasis{\sphinxupquote{int}}) \textendash{} number of random variables

\end{itemize}

\item[{Returns}] \leavevmode
\sphinxstylestrong{num\_coeffs} \textendash{} number of coefficients and polynomials

\item[{Return type}] \leavevmode
int

\end{description}\end{quote}

\end{fulllineitems}

\index{get\_num\_coeffs\_sparse() (in module pygpc.misc)}

\begin{fulllineitems}
\phantomsection\label{\detokenize{pygpc:pygpc.misc.get_num_coeffs_sparse}}\pysiglinewithargsret{\sphinxcode{\sphinxupquote{pygpc.misc.}}\sphinxbfcode{\sphinxupquote{get\_num\_coeffs\_sparse}}}{\emph{order\_dim\_max}, \emph{order\_glob\_max}, \emph{order\_inter\_max}, \emph{dim}}{}
Calculate the number of PCE coefficients for a specific maximum order in each dimension order\_dim\_max,
maximum order of interacting polynomials order\_glob\_max and the interaction order order\_inter\_max.

num\_coeffs\_sparse = get\_num\_coeffs\_sparse(order\_dim\_max, order\_glob\_max, order\_inter\_max, dim)
\begin{quote}\begin{description}
\item[{Parameters}] \leavevmode\begin{itemize}
\item {} 
\sphinxstyleliteralstrong{\sphinxupquote{order\_dim\_max}} (\sphinxstyleliteralemphasis{\sphinxupquote{int}}\sphinxstyleliteralemphasis{\sphinxupquote{ or }}\sphinxstyleliteralemphasis{\sphinxupquote{np.ndarray}}) \textendash{} maximum order in each dimension

\item {} 
\sphinxstyleliteralstrong{\sphinxupquote{order\_glob\_max}} (\sphinxstyleliteralemphasis{\sphinxupquote{int}}) \textendash{} maximum global order of interacting polynomials

\item {} 
\sphinxstyleliteralstrong{\sphinxupquote{order\_inter\_max}} (\sphinxstyleliteralemphasis{\sphinxupquote{int}}) \textendash{} interaction order

\item {} 
\sphinxstyleliteralstrong{\sphinxupquote{dim}} (\sphinxstyleliteralemphasis{\sphinxupquote{int}}) \textendash{} number of random variables

\end{itemize}

\item[{Returns}] \leavevmode
\sphinxstylestrong{num\_coeffs\_sparse} \textendash{} number of coefficients and polynomials

\item[{Return type}] \leavevmode
int

\end{description}\end{quote}

\end{fulllineitems}

\index{get\_pdf\_beta() (in module pygpc.misc)}

\begin{fulllineitems}
\phantomsection\label{\detokenize{pygpc:pygpc.misc.get_pdf_beta}}\pysiglinewithargsret{\sphinxcode{\sphinxupquote{pygpc.misc.}}\sphinxbfcode{\sphinxupquote{get\_pdf\_beta}}}{\emph{x}, \emph{p}, \emph{q}, \emph{a}, \emph{b}}{}
Calculate the probability density function of the beta distribution in the interval {[}a,b{]}.
\begin{description}
\item[{pdf = (gamma(p)*gamma(q)/gamma(p+q).*(b-a)**(p+q-1))**(-1) *}] \leavevmode
(x-a)**(p-1) * (b-x)**(q-1);

\end{description}

pdf = get\_pdf\_beta(x, p, q, a, b)
\begin{quote}\begin{description}
\item[{Parameters}] \leavevmode\begin{itemize}
\item {} 
\sphinxstyleliteralstrong{\sphinxupquote{x}} (\sphinxstyleliteralemphasis{\sphinxupquote{np.ndarray}}) \textendash{} values of random variable

\item {} 
\sphinxstyleliteralstrong{\sphinxupquote{a}} (\sphinxstyleliteralemphasis{\sphinxupquote{float}}) \textendash{} min boundary

\item {} 
\sphinxstyleliteralstrong{\sphinxupquote{b}} (\sphinxstyleliteralemphasis{\sphinxupquote{float}}) \textendash{} max boundary

\item {} 
\sphinxstyleliteralstrong{\sphinxupquote{p}} (\sphinxstyleliteralemphasis{\sphinxupquote{float}}) \textendash{} parameter defining the distribution shape

\item {} 
\sphinxstyleliteralstrong{\sphinxupquote{q}} (\sphinxstyleliteralemphasis{\sphinxupquote{float}}) \textendash{} parameter defining the distribution shape

\end{itemize}

\item[{Returns}] \leavevmode
\sphinxstylestrong{pdf} \textendash{} probability density

\item[{Return type}] \leavevmode
np.ndarray

\end{description}\end{quote}

\end{fulllineitems}

\index{get\_rotation\_matrix() (in module pygpc.misc)}

\begin{fulllineitems}
\phantomsection\label{\detokenize{pygpc:pygpc.misc.get_rotation_matrix}}\pysiglinewithargsret{\sphinxcode{\sphinxupquote{pygpc.misc.}}\sphinxbfcode{\sphinxupquote{get\_rotation\_matrix}}}{\emph{theta}}{}
Generate rotation matrix from euler angles.

rotation\_matrix = get\_rotation\_matrix(theta)
\begin{quote}\begin{description}
\item[{Parameters}] \leavevmode
\sphinxstyleliteralstrong{\sphinxupquote{theta}} (\sphinxstyleliteralemphasis{\sphinxupquote{list of float}}) \textendash{} list of euler angles

\item[{Returns}] \leavevmode
\sphinxstylestrong{rotation\_matrix} \textendash{} rotation matrix computed from euler angles

\item[{Return type}] \leavevmode
{[}3,3{]} np.ndarray

\end{description}\end{quote}

\end{fulllineitems}

\index{get\_set\_combinations() (in module pygpc.misc)}

\begin{fulllineitems}
\phantomsection\label{\detokenize{pygpc:pygpc.misc.get_set_combinations}}\pysiglinewithargsret{\sphinxcode{\sphinxupquote{pygpc.misc.}}\sphinxbfcode{\sphinxupquote{get\_set\_combinations}}}{\emph{array}, \emph{number\_elements}}{}
Computes all k-tuples (e\_1, e\_2, …, e\_k) of combinations of the set of elements of the first row of the
input matrix where e\_n+1 \textgreater{} e\_n

combination\_vectors = get\_set\_combinations(array, number\_elements)
\begin{quote}\begin{description}
\item[{Parameters}] \leavevmode\begin{itemize}
\item {} 
\sphinxstyleliteralstrong{\sphinxupquote{array}} (\sphinxstyleliteralemphasis{\sphinxupquote{np.ndarray}}) \textendash{} matrix containing a first row of input elements

\item {} 
\sphinxstyleliteralstrong{\sphinxupquote{number\_elements}} (\sphinxstyleliteralemphasis{\sphinxupquote{int}}) \textendash{} number of elements in tuple

\end{itemize}

\item[{Returns}] \leavevmode
\sphinxstylestrong{combination\_vectors} \textendash{} matrix of combination vectors

\item[{Return type}] \leavevmode
np.ndarray

\end{description}\end{quote}

\end{fulllineitems}

\index{mutcoh() (in module pygpc.misc)}

\begin{fulllineitems}
\phantomsection\label{\detokenize{pygpc:pygpc.misc.mutcoh}}\pysiglinewithargsret{\sphinxcode{\sphinxupquote{pygpc.misc.}}\sphinxbfcode{\sphinxupquote{mutcoh}}}{\emph{array}}{}
Calculate the mutual coherence of a matrix A. It can also be referred as the cosine
of the smallest angle between two columns.

mutual\_coherence = mutcoh(array)
\begin{quote}\begin{description}
\item[{Parameters}] \leavevmode
\sphinxstyleliteralstrong{\sphinxupquote{array}} (\sphinxstyleliteralemphasis{\sphinxupquote{np.ndarray}}) \textendash{} input matrix

\item[{Returns}] \leavevmode
\sphinxstylestrong{mutual\_coherence}

\item[{Return type}] \leavevmode
float

\end{description}\end{quote}

\end{fulllineitems}

\index{vprint() (in module pygpc.misc)}

\begin{fulllineitems}
\phantomsection\label{\detokenize{pygpc:pygpc.misc.vprint}}\pysiglinewithargsret{\sphinxcode{\sphinxupquote{pygpc.misc.}}\sphinxbfcode{\sphinxupquote{vprint}}}{\emph{message}, \emph{verbose=True}}{}
Function that prints out a message if verbose argument is true.

vprint(message, verbose=True)
\begin{quote}\begin{description}
\item[{Parameters}] \leavevmode\begin{itemize}
\item {} 
\sphinxstyleliteralstrong{\sphinxupquote{message}} (\sphinxstyleliteralemphasis{\sphinxupquote{string}}) \textendash{} string to print in standard output

\item {} 
\sphinxstyleliteralstrong{\sphinxupquote{verbose}} (\sphinxstyleliteralemphasis{\sphinxupquote{bool}}\sphinxstyleliteralemphasis{\sphinxupquote{, }}\sphinxstyleliteralemphasis{\sphinxupquote{optional}}\sphinxstyleliteralemphasis{\sphinxupquote{, }}\sphinxstyleliteralemphasis{\sphinxupquote{default=True}}) \textendash{} determines if string is printed out

\end{itemize}

\end{description}\end{quote}

\end{fulllineitems}

\index{wrap\_function() (in module pygpc.misc)}

\begin{fulllineitems}
\phantomsection\label{\detokenize{pygpc:pygpc.misc.wrap_function}}\pysiglinewithargsret{\sphinxcode{\sphinxupquote{pygpc.misc.}}\sphinxbfcode{\sphinxupquote{wrap\_function}}}{\emph{fn}, \emph{x}, \emph{args}}{}
Function wrapper to call anonymous function with variable number of arguments (tuple).

wrap\_function(fn, x, args)
\begin{quote}\begin{description}
\item[{Parameters}] \leavevmode\begin{itemize}
\item {} 
\sphinxstyleliteralstrong{\sphinxupquote{fn}} (\sphinxstyleliteralemphasis{\sphinxupquote{function}}) \textendash{} anonymous function to call

\item {} 
\sphinxstyleliteralstrong{\sphinxupquote{x}} (\sphinxstyleliteralemphasis{\sphinxupquote{tuple}}) \textendash{} parameters of function

\item {} 
\sphinxstyleliteralstrong{\sphinxupquote{args}} (\sphinxstyleliteralemphasis{\sphinxupquote{tuple}}) \textendash{} arguments of function

\end{itemize}

\item[{Returns}] \leavevmode
\sphinxstylestrong{function\_wrapper} \textendash{} wrapped function

\item[{Return type}] \leavevmode
function

\end{description}\end{quote}

\end{fulllineitems}



\section{pygpc.ni module}
\label{\detokenize{pygpc:module-pygpc.ni}}\label{\detokenize{pygpc:pygpc-ni-module}}\index{pygpc.ni (module)}
Functions that provide adaptive regression approches to perform uncertainty analysis on dynamic systems.
\index{run\_reg\_adaptive2() (in module pygpc.ni)}

\begin{fulllineitems}
\phantomsection\label{\detokenize{pygpc:pygpc.ni.run_reg_adaptive2}}\pysiglinewithargsret{\sphinxcode{\sphinxupquote{pygpc.ni.}}\sphinxbfcode{\sphinxupquote{run\_reg\_adaptive2}}}{\emph{random\_vars}, \emph{pdf\_type}, \emph{pdf\_shape}, \emph{limits}, \emph{func}, \emph{args=()}, \emph{order\_start=0}, \emph{order\_end=10}, \emph{interaction\_order\_max=None}, \emph{eps=0.001}, \emph{print\_out=False}, \emph{seed=None}, \emph{save\_res\_fn=''}}{}
Perform adaptive regression approach based on leave one out cross validation error estimation.
\begin{quote}\begin{description}
\item[{Parameters}] \leavevmode\begin{itemize}
\item {} 
\sphinxstyleliteralstrong{\sphinxupquote{random\_vars}} (\sphinxstyleliteralemphasis{\sphinxupquote{list of str}}) \textendash{} string labels of the random variables

\item {} 
\sphinxstyleliteralstrong{\sphinxupquote{pdf\_type}} (\sphinxstyleliteralemphasis{\sphinxupquote{list}}) \textendash{} type of probability density functions of input parameters,
i.e. {[}“beta”, “norm”,…{]}

\item {} 
\sphinxstyleliteralstrong{\sphinxupquote{pdf\_shape}} (\sphinxstyleliteralemphasis{\sphinxupquote{list of lists}}) \textendash{} shape parameters of probability density functions
s1={[}…{]} “beta”: p, “norm”: mean
s2={[}…{]} “beta”: q, “norm”: std
pdf\_shape = {[}s1,s2{]}

\item {} 
\sphinxstyleliteralstrong{\sphinxupquote{limits}} (\sphinxstyleliteralemphasis{\sphinxupquote{list of lists}}) \textendash{} upper and lower bounds of random variables (only “beta”)
a={[}…{]} “beta”: lower bound, “norm”: n/a define 0
b={[}…{]} “beta”: upper bound, “norm”: n/a define 0
limits = {[}a,b{]}

\item {} 
\sphinxstyleliteralstrong{\sphinxupquote{func}} (\sphinxstyleliteralemphasis{\sphinxupquote{function}}) \textendash{} the objective function to be minimized
func(x,*args)

\item {} 
\sphinxstyleliteralstrong{\sphinxupquote{args}} (\sphinxstyleliteralemphasis{\sphinxupquote{tuple}}\sphinxstyleliteralemphasis{\sphinxupquote{, }}\sphinxstyleliteralemphasis{\sphinxupquote{optional}}\sphinxstyleliteralemphasis{\sphinxupquote{, }}\sphinxstyleliteralemphasis{\sphinxupquote{default=}}\sphinxstyleliteralemphasis{\sphinxupquote{(}}\sphinxstyleliteralemphasis{\sphinxupquote{)}}) \textendash{} extra arguments passed to function
i.e. f(x,*args)

\item {} 
\sphinxstyleliteralstrong{\sphinxupquote{order\_start}} (\sphinxstyleliteralemphasis{\sphinxupquote{int}}\sphinxstyleliteralemphasis{\sphinxupquote{, }}\sphinxstyleliteralemphasis{\sphinxupquote{optional}}\sphinxstyleliteralemphasis{\sphinxupquote{, }}\sphinxstyleliteralemphasis{\sphinxupquote{default=0}}) \textendash{} initial gpc expansion order

\item {} 
\sphinxstyleliteralstrong{\sphinxupquote{order\_end}} (\sphinxstyleliteralemphasis{\sphinxupquote{int}}\sphinxstyleliteralemphasis{\sphinxupquote{, }}\sphinxstyleliteralemphasis{\sphinxupquote{optional}}\sphinxstyleliteralemphasis{\sphinxupquote{, }}\sphinxstyleliteralemphasis{\sphinxupquote{default=10}}) \textendash{} maximum gpc expansion order

\item {} 
\sphinxstyleliteralstrong{\sphinxupquote{interaction\_order\_max}} (\sphinxstyleliteralemphasis{\sphinxupquote{int}}\sphinxstyleliteralemphasis{\sphinxupquote{, }}\sphinxstyleliteralemphasis{\sphinxupquote{optional}}\sphinxstyleliteralemphasis{\sphinxupquote{, }}\sphinxstyleliteralemphasis{\sphinxupquote{defailt=None}}) \textendash{} define maximum interaction order of parameters
if None, perform all interactions

\item {} 
\sphinxstyleliteralstrong{\sphinxupquote{eps}} (\sphinxstyleliteralemphasis{\sphinxupquote{float}}\sphinxstyleliteralemphasis{\sphinxupquote{, }}\sphinxstyleliteralemphasis{\sphinxupquote{optional}}\sphinxstyleliteralemphasis{\sphinxupquote{, }}\sphinxstyleliteralemphasis{\sphinxupquote{default=1E-3}}) \textendash{} relative mean error bound of leave one out cross validation

\item {} 
\sphinxstyleliteralstrong{\sphinxupquote{print\_out}} (\sphinxstyleliteralemphasis{\sphinxupquote{boolean}}\sphinxstyleliteralemphasis{\sphinxupquote{, }}\sphinxstyleliteralemphasis{\sphinxupquote{optional}}\sphinxstyleliteralemphasis{\sphinxupquote{, }}\sphinxstyleliteralemphasis{\sphinxupquote{default=False}}) \textendash{} boolean value that determines if to print output the iterations and subiterations

\item {} 
\sphinxstyleliteralstrong{\sphinxupquote{seed}} (\sphinxstyleliteralemphasis{\sphinxupquote{int}}\sphinxstyleliteralemphasis{\sphinxupquote{, }}\sphinxstyleliteralemphasis{\sphinxupquote{optional}}\sphinxstyleliteralemphasis{\sphinxupquote{, }}\sphinxstyleliteralemphasis{\sphinxupquote{default=None}}) \textendash{} seeding point to replicate random grids

\item {} 
\sphinxstyleliteralstrong{\sphinxupquote{save\_res\_fn}} (\sphinxstyleliteralemphasis{\sphinxupquote{str}}\sphinxstyleliteralemphasis{\sphinxupquote{, }}\sphinxstyleliteralemphasis{\sphinxupquote{optional}}\sphinxstyleliteralemphasis{\sphinxupquote{, }}\sphinxstyleliteralemphasis{\sphinxupquote{default}}) \textendash{} hdf5 filename where the output data should be saved

\end{itemize}

\item[{Returns}] \leavevmode
\begin{itemize}
\item {} 
\sphinxstylestrong{gobj} (\sphinxstyleemphasis{gpc object}) \textendash{} gpc object

\item {} 
\sphinxstylestrong{res} (\sphinxstyleemphasis{{[}N\_grid x N\_out{]} np.ndarray}) \textendash{} function values at grid points of the N\_out output variables

\end{itemize}


\end{description}\end{quote}

\end{fulllineitems}

\index{run\_reg\_adaptive\_E\_gPC() (in module pygpc.ni)}

\begin{fulllineitems}
\phantomsection\label{\detokenize{pygpc:pygpc.ni.run_reg_adaptive_E_gPC}}\pysiglinewithargsret{\sphinxcode{\sphinxupquote{pygpc.ni.}}\sphinxbfcode{\sphinxupquote{run\_reg\_adaptive\_E\_gPC}}}{\emph{pdf\_type}, \emph{pdf\_shape}, \emph{limits}, \emph{func}, \emph{args=()}, \emph{fname=None}, \emph{order\_start=0}, \emph{order\_end=10}, \emph{interaction\_order\_max=None}, \emph{eps=0.001}, \emph{print\_out=False}, \emph{seed=None}, \emph{do\_mp=False}, \emph{n\_cpu=4}, \emph{dispy=False}, \emph{dispy\_sched\_host='localhost'}, \emph{random\_vars=''}, \emph{hdf5\_geo\_fn=''}}{}
Perform adaptive regression approach based on leave one out cross validation error estimation.
\begin{quote}\begin{description}
\item[{Parameters}] \leavevmode\begin{itemize}
\item {} 
\sphinxstyleliteralstrong{\sphinxupquote{random\_vars}} (\sphinxstyleliteralemphasis{\sphinxupquote{list of str}}) \textendash{} string labels of the random variables

\item {} 
\sphinxstyleliteralstrong{\sphinxupquote{pdf\_type}} (\sphinxstyleliteralemphasis{\sphinxupquote{list}}) \textendash{} type of probability density functions of input parameters,
i.e. {[}“beta”, “norm”,…{]}

\item {} 
\sphinxstyleliteralstrong{\sphinxupquote{pdf\_shape}} (\sphinxstyleliteralemphasis{\sphinxupquote{list of lists}}) \textendash{} shape parameters of probability density functions
s1={[}…{]} “beta”: p, “norm”: mean
s2={[}…{]} “beta”: q, “norm”: std
pdf\_shape = {[}s1,s2{]}

\item {} 
\sphinxstyleliteralstrong{\sphinxupquote{limits}} (\sphinxstyleliteralemphasis{\sphinxupquote{list of lists}}) \textendash{} upper and lower bounds of random variables (only “beta”)
a={[}…{]} “beta”: lower bound, “norm”: n/a define 0
b={[}…{]} “beta”: upper bound, “norm”: n/a define 0
limits = {[}a,b{]}

\item {} 
\sphinxstyleliteralstrong{\sphinxupquote{func}} (\sphinxstyleliteralemphasis{\sphinxupquote{function}}) \textendash{} the objective function to be minimized
func(x,*args)

\item {} 
\sphinxstyleliteralstrong{\sphinxupquote{args}} (\sphinxstyleliteralemphasis{\sphinxupquote{tuple}}\sphinxstyleliteralemphasis{\sphinxupquote{, }}\sphinxstyleliteralemphasis{\sphinxupquote{optional}}\sphinxstyleliteralemphasis{\sphinxupquote{, }}\sphinxstyleliteralemphasis{\sphinxupquote{default=}}\sphinxstyleliteralemphasis{\sphinxupquote{(}}\sphinxstyleliteralemphasis{\sphinxupquote{)}}) \textendash{} extra arguments passed to function
i.e. f(x,*args)

\item {} 
\sphinxstyleliteralstrong{\sphinxupquote{fname}} (\sphinxstyleliteralemphasis{\sphinxupquote{str}}\sphinxstyleliteralemphasis{\sphinxupquote{, }}\sphinxstyleliteralemphasis{\sphinxupquote{optional}}\sphinxstyleliteralemphasis{\sphinxupquote{, }}\sphinxstyleliteralemphasis{\sphinxupquote{default=None}}) \textendash{} if fname exists, reg\_obj will be created from it
if not exist, it will be created

\item {} 
\sphinxstyleliteralstrong{\sphinxupquote{order\_start}} (\sphinxstyleliteralemphasis{\sphinxupquote{int}}\sphinxstyleliteralemphasis{\sphinxupquote{, }}\sphinxstyleliteralemphasis{\sphinxupquote{optional}}\sphinxstyleliteralemphasis{\sphinxupquote{, }}\sphinxstyleliteralemphasis{\sphinxupquote{default=0}}) \textendash{} initial gpc expansion order

\item {} 
\sphinxstyleliteralstrong{\sphinxupquote{order\_end}} (\sphinxstyleliteralemphasis{\sphinxupquote{int}}\sphinxstyleliteralemphasis{\sphinxupquote{, }}\sphinxstyleliteralemphasis{\sphinxupquote{optional}}\sphinxstyleliteralemphasis{\sphinxupquote{, }}\sphinxstyleliteralemphasis{\sphinxupquote{default=10}}) \textendash{} maximum gpc expansion order

\item {} 
\sphinxstyleliteralstrong{\sphinxupquote{interaction\_order\_max}} (\sphinxstyleliteralemphasis{\sphinxupquote{int}}\sphinxstyleliteralemphasis{\sphinxupquote{, }}\sphinxstyleliteralemphasis{\sphinxupquote{optional}}\sphinxstyleliteralemphasis{\sphinxupquote{, }}\sphinxstyleliteralemphasis{\sphinxupquote{defailt=None}}) \textendash{} define maximum interaction order of parameters
if None, perform all interactions

\item {} 
\sphinxstyleliteralstrong{\sphinxupquote{eps}} (\sphinxstyleliteralemphasis{\sphinxupquote{float}}\sphinxstyleliteralemphasis{\sphinxupquote{, }}\sphinxstyleliteralemphasis{\sphinxupquote{optional}}\sphinxstyleliteralemphasis{\sphinxupquote{, }}\sphinxstyleliteralemphasis{\sphinxupquote{default=1E-3}}) \textendash{} relative mean error bound of leave one out cross validation

\item {} 
\sphinxstyleliteralstrong{\sphinxupquote{print\_out}} (\sphinxstyleliteralemphasis{\sphinxupquote{boolean}}\sphinxstyleliteralemphasis{\sphinxupquote{, }}\sphinxstyleliteralemphasis{\sphinxupquote{optional}}\sphinxstyleliteralemphasis{\sphinxupquote{, }}\sphinxstyleliteralemphasis{\sphinxupquote{default=False}}) \textendash{} boolean value that determines if to print output the iterations and subiterations

\item {} 
\sphinxstyleliteralstrong{\sphinxupquote{seed}} (\sphinxstyleliteralemphasis{\sphinxupquote{int}}\sphinxstyleliteralemphasis{\sphinxupquote{, }}\sphinxstyleliteralemphasis{\sphinxupquote{optional}}\sphinxstyleliteralemphasis{\sphinxupquote{, }}\sphinxstyleliteralemphasis{\sphinxupquote{default=None}}) \textendash{} seeding point to replicate random grids

\item {} 
\sphinxstyleliteralstrong{\sphinxupquote{do\_mp}} (\sphinxstyleliteralemphasis{\sphinxupquote{boolean}}\sphinxstyleliteralemphasis{\sphinxupquote{, }}\sphinxstyleliteralemphasis{\sphinxupquote{optional}}\sphinxstyleliteralemphasis{\sphinxupquote{, }}\sphinxstyleliteralemphasis{\sphinxupquote{default=False}}) \textendash{} boolean value that determines if to do each func(x,*args) in each iteration with parmap.starmap(func)

\item {} 
\sphinxstyleliteralstrong{\sphinxupquote{n\_cpu}} (\sphinxstyleliteralemphasis{\sphinxupquote{int}}\sphinxstyleliteralemphasis{\sphinxupquote{, }}\sphinxstyleliteralemphasis{\sphinxupquote{optional}}\sphinxstyleliteralemphasis{\sphinxupquote{, }}\sphinxstyleliteralemphasis{\sphinxupquote{default=4}}) \textendash{} if multiprocessing is enabled, utilize n\_cpu cores

\item {} 
\sphinxstyleliteralstrong{\sphinxupquote{dispy}} (\sphinxstyleliteralemphasis{\sphinxupquote{boolean}}\sphinxstyleliteralemphasis{\sphinxupquote{, }}\sphinxstyleliteralemphasis{\sphinxupquote{optional}}\sphinxstyleliteralemphasis{\sphinxupquote{, }}\sphinxstyleliteralemphasis{\sphinxupquote{default=False}}) \textendash{} boolean value that determines if to compute function with dispy cluster

\item {} 
\sphinxstyleliteralstrong{\sphinxupquote{dispy\_sched\_host}} (\sphinxstyleliteralemphasis{\sphinxupquote{str}}\sphinxstyleliteralemphasis{\sphinxupquote{, }}\sphinxstyleliteralemphasis{\sphinxupquote{optional}}\sphinxstyleliteralemphasis{\sphinxupquote{, }}\sphinxstyleliteralemphasis{\sphinxupquote{default='localhost'}}) \textendash{} host name where dispyscheduler will be running

\item {} 
\sphinxstyleliteralstrong{\sphinxupquote{hdf5\_geo\_fn}} (\sphinxstyleliteralemphasis{\sphinxupquote{str}}\sphinxstyleliteralemphasis{\sphinxupquote{, }}\sphinxstyleliteralemphasis{\sphinxupquote{optional}}\sphinxstyleliteralemphasis{\sphinxupquote{, }}\sphinxstyleliteralemphasis{\sphinxupquote{default='{'}}}) \textendash{} hdf5 filename with spatial information: /mesh/elm/*

\end{itemize}

\item[{Returns}] \leavevmode
\begin{itemize}
\item {} 
\sphinxstylestrong{gobj} (\sphinxstyleemphasis{gpc object}) \textendash{} gpc object

\item {} 
\sphinxstylestrong{res} (\sphinxstyleemphasis{{[}N\_grid x N\_out{]} np.ndarray}) \textendash{} function values at grid points of the N\_out output variables

\end{itemize}


\end{description}\end{quote}

\end{fulllineitems}



\section{pygpc.postproc module}
\label{\detokenize{pygpc:module-pygpc.postproc}}\label{\detokenize{pygpc:pygpc-postproc-module}}\index{pygpc.postproc (module)}
Functions that provide postprocessing implementations
\index{get\_extracted\_sobol\_order() (in module pygpc.postproc)}

\begin{fulllineitems}
\phantomsection\label{\detokenize{pygpc:pygpc.postproc.get_extracted_sobol_order}}\pysiglinewithargsret{\sphinxcode{\sphinxupquote{pygpc.postproc.}}\sphinxbfcode{\sphinxupquote{get\_extracted\_sobol\_order}}}{\emph{sobol}, \emph{sobol\_idx}, \emph{order=1}}{}
Extract Sobol indices with specified order from Sobol data.

sobol\_1st, sobol\_idx\_1st = extract\_sobol\_order(sobol, sobol\_idx, order=1)
\begin{quote}\begin{description}
\item[{Parameters}] \leavevmode\begin{itemize}
\item {} 
\sphinxstyleliteralstrong{\sphinxupquote{sobol}} (\sphinxstyleliteralemphasis{\sphinxupquote{{[}}}\sphinxstyleliteralemphasis{\sphinxupquote{N\_sobol x N\_out}}\sphinxstyleliteralemphasis{\sphinxupquote{{]} }}\sphinxstyleliteralemphasis{\sphinxupquote{np.ndarray}}) \textendash{} Sobol indices of N\_out output quantities

\item {} 
\sphinxstyleliteralstrong{\sphinxupquote{sobol\_idx}} (\sphinxstyleliteralemphasis{\sphinxupquote{{[}}}\sphinxstyleliteralemphasis{\sphinxupquote{N\_sobol}}\sphinxstyleliteralemphasis{\sphinxupquote{{]} }}\sphinxstyleliteralemphasis{\sphinxupquote{list}}\sphinxstyleliteralemphasis{\sphinxupquote{ or }}\sphinxstyleliteralemphasis{\sphinxupquote{np.ndarray of int}}) \textendash{} list of parameter label indices belonging to Sobol indices

\item {} 
\sphinxstyleliteralstrong{\sphinxupquote{order}} (\sphinxstyleliteralemphasis{\sphinxupquote{int}}\sphinxstyleliteralemphasis{\sphinxupquote{, }}\sphinxstyleliteralemphasis{\sphinxupquote{optional}}\sphinxstyleliteralemphasis{\sphinxupquote{, }}\sphinxstyleliteralemphasis{\sphinxupquote{default=1}}) \textendash{} Sobol index order to extract

\end{itemize}

\item[{Returns}] \leavevmode
\begin{itemize}
\item {} 
\sphinxstylestrong{sobol\_n\_order} (\sphinxstyleemphasis{np.ndarray}) \textendash{} n-th order Sobol indices of N\_out output quantities

\item {} 
\sphinxstylestrong{sobol\_idx\_n\_order} (\sphinxstyleemphasis{np.ndarray}) \textendash{} List of parameter label indices belonging to n-th order Sobol indices

\end{itemize}


\end{description}\end{quote}

\end{fulllineitems}



\section{pygpc.quad module}
\label{\detokenize{pygpc:module-pygpc.quad}}\label{\detokenize{pygpc:pygpc-quad-module}}\index{pygpc.quad (module)}
Class that provides polynomial chaos quadratur methods
\index{Quad (class in pygpc.quad)}

\begin{fulllineitems}
\phantomsection\label{\detokenize{pygpc:pygpc.quad.Quad}}\pysiglinewithargsret{\sphinxbfcode{\sphinxupquote{class }}\sphinxcode{\sphinxupquote{pygpc.quad.}}\sphinxbfcode{\sphinxupquote{Quad}}}{\emph{pdf\_type}, \emph{pdf\_shape}, \emph{limits}, \emph{order}, \emph{order\_max}, \emph{interaction\_order}, \emph{grid}, \emph{random\_vars=None}}{}
Bases: {\hyperref[\detokenize{pygpc:pygpc.gpc.gPC}]{\sphinxcrossref{\sphinxcode{\sphinxupquote{pygpc.gpc.gPC}}}}}

Quadratur gPC subclass

Quad(pdf\_type, pdf\_shape, limits, order, order\_max, interaction\_order, grid, random\_vars=None)
\index{N\_grid (pygpc.quad.Quad attribute)}

\begin{fulllineitems}
\phantomsection\label{\detokenize{pygpc:pygpc.quad.Quad.N_grid}}\pysigline{\sphinxbfcode{\sphinxupquote{N\_grid}}}
number of grid points
\begin{quote}\begin{description}
\item[{Type}] \leavevmode
int

\end{description}\end{quote}

\end{fulllineitems}

\index{dim (pygpc.quad.Quad attribute)}

\begin{fulllineitems}
\phantomsection\label{\detokenize{pygpc:pygpc.quad.Quad.dim}}\pysigline{\sphinxbfcode{\sphinxupquote{dim}}}
number of uncertain parameters to process
\begin{quote}\begin{description}
\item[{Type}] \leavevmode
int

\end{description}\end{quote}

\end{fulllineitems}

\index{pdf\_type (pygpc.quad.Quad attribute)}

\begin{fulllineitems}
\phantomsection\label{\detokenize{pygpc:pygpc.quad.Quad.pdf_type}}\pysigline{\sphinxbfcode{\sphinxupquote{pdf\_type}}}
type of pdf ‘beta’ or ‘norm’
\begin{quote}\begin{description}
\item[{Type}] \leavevmode
{[}dim{]} list of str

\end{description}\end{quote}

\end{fulllineitems}

\index{pdf\_shape (pygpc.quad.Quad attribute)}

\begin{fulllineitems}
\phantomsection\label{\detokenize{pygpc:pygpc.quad.Quad.pdf_shape}}\pysigline{\sphinxbfcode{\sphinxupquote{pdf\_shape}}}
shape parameters of pdfs
beta-dist:   {[}{[}alpha{]}, {[}beta{]}    {]}
normal-dist: {[}{[}mean{]},  {[}variance{]}{]}
\begin{quote}\begin{description}
\item[{Type}] \leavevmode
list of list of float

\end{description}\end{quote}

\end{fulllineitems}

\index{limits (pygpc.quad.Quad attribute)}

\begin{fulllineitems}
\phantomsection\label{\detokenize{pygpc:pygpc.quad.Quad.limits}}\pysigline{\sphinxbfcode{\sphinxupquote{limits}}}
upper and lower bounds of random variables
beta-dist:   {[}{[}a1 …{]}, {[}b1 …{]}{]}
normal-dist: {[}{[}0 … {]}, {[}0 … {]}{]} (not used)
\begin{quote}\begin{description}
\item[{Type}] \leavevmode
list of list of float

\end{description}\end{quote}

\end{fulllineitems}

\index{order (pygpc.quad.Quad attribute)}

\begin{fulllineitems}
\phantomsection\label{\detokenize{pygpc:pygpc.quad.Quad.order}}\pysigline{\sphinxbfcode{\sphinxupquote{order}}}
maximum individual expansion order
generates individual polynomials also if maximum expansion order in order\_max is exceeded
\begin{quote}\begin{description}
\item[{Type}] \leavevmode
{[}dim{]} list of int

\end{description}\end{quote}

\end{fulllineitems}

\index{order\_max (pygpc.quad.Quad attribute)}

\begin{fulllineitems}
\phantomsection\label{\detokenize{pygpc:pygpc.quad.Quad.order_max}}\pysigline{\sphinxbfcode{\sphinxupquote{order\_max}}}
maximum expansion order (sum of all exponents)
the maximum expansion order considers the sum of the orders of combined polynomials only
\begin{quote}\begin{description}
\item[{Type}] \leavevmode
int

\end{description}\end{quote}

\end{fulllineitems}

\index{interaction\_order (pygpc.quad.Quad attribute)}

\begin{fulllineitems}
\phantomsection\label{\detokenize{pygpc:pygpc.quad.Quad.interaction_order}}\pysigline{\sphinxbfcode{\sphinxupquote{interaction\_order}}}
number of random variables, which can interact with each other
all polynomials are ignored, which have an interaction order greater than the specified
\begin{quote}\begin{description}
\item[{Type}] \leavevmode
int

\end{description}\end{quote}

\end{fulllineitems}

\index{grid (pygpc.quad.Quad attribute)}

\begin{fulllineitems}
\phantomsection\label{\detokenize{pygpc:pygpc.quad.Quad.grid}}\pysigline{\sphinxbfcode{\sphinxupquote{grid}}}
grid object generated in grid.py including grid.coords and grid.coords\_norm
\begin{quote}\begin{description}
\item[{Type}] \leavevmode
grid object

\end{description}\end{quote}

\end{fulllineitems}

\index{random\_vars (pygpc.quad.Quad attribute)}

\begin{fulllineitems}
\phantomsection\label{\detokenize{pygpc:pygpc.quad.Quad.random_vars}}\pysigline{\sphinxbfcode{\sphinxupquote{random\_vars}}}
string labels of the random variables
\begin{quote}\begin{description}
\item[{Type}] \leavevmode
{[}dim{]} list of str

\end{description}\end{quote}

\end{fulllineitems}

\begin{quote}\begin{description}
\item[{Parameters}] \leavevmode\begin{itemize}
\item {} 
\sphinxstyleliteralstrong{\sphinxupquote{pdf\_type}} (\sphinxstyleliteralemphasis{\sphinxupquote{{[}}}\sphinxstyleliteralemphasis{\sphinxupquote{dim}}\sphinxstyleliteralemphasis{\sphinxupquote{{]} }}\sphinxstyleliteralemphasis{\sphinxupquote{list of str}}) \textendash{} type of pdf ‘beta’ or ‘norm’

\item {} 
\sphinxstyleliteralstrong{\sphinxupquote{pdf\_shape}} (\sphinxstyleliteralemphasis{\sphinxupquote{list of list of float}}) \textendash{} shape parameters of pdfs
beta-dist:   {[}{[}alpha{]}, {[}beta{]}    {]}
normal-dist: {[}{[}mean{]},  {[}variance{]}{]}

\item {} 
\sphinxstyleliteralstrong{\sphinxupquote{limits}} (\sphinxstyleliteralemphasis{\sphinxupquote{list of list of float}}) \textendash{} upper and lower bounds of random variables
beta-dist:   {[}{[}a1 …{]}, {[}b1 …{]}{]}
normal-dist: {[}{[}0 … {]}, {[}0 … {]}{]} (not used)

\item {} 
\sphinxstyleliteralstrong{\sphinxupquote{order}} (\sphinxstyleliteralemphasis{\sphinxupquote{{[}}}\sphinxstyleliteralemphasis{\sphinxupquote{dim}}\sphinxstyleliteralemphasis{\sphinxupquote{{]} }}\sphinxstyleliteralemphasis{\sphinxupquote{list of int}}) \textendash{} maximum individual expansion order
generates individual polynomials also if maximum expansion order in order\_max is exceeded

\item {} 
\sphinxstyleliteralstrong{\sphinxupquote{order\_max}} (\sphinxstyleliteralemphasis{\sphinxupquote{int}}) \textendash{} maximum expansion order (sum of all exponents)
the maximum expansion order considers the sum of the orders of combined polynomials only

\item {} 
\sphinxstyleliteralstrong{\sphinxupquote{interaction\_order}} (\sphinxstyleliteralemphasis{\sphinxupquote{int}}) \textendash{} number of random variables, which can interact with each other
all polynomials are ignored, which have an interaction order greater than the specified

\item {} 
\sphinxstyleliteralstrong{\sphinxupquote{grid}} (\sphinxstyleliteralemphasis{\sphinxupquote{grid object}}) \textendash{} grid object generated in grid.py including grid.coords and grid.coords\_norm

\item {} 
\sphinxstyleliteralstrong{\sphinxupquote{random\_vars}} (\sphinxstyleliteralemphasis{\sphinxupquote{{[}}}\sphinxstyleliteralemphasis{\sphinxupquote{dim}}\sphinxstyleliteralemphasis{\sphinxupquote{{]} }}\sphinxstyleliteralemphasis{\sphinxupquote{list of str}}\sphinxstyleliteralemphasis{\sphinxupquote{, }}\sphinxstyleliteralemphasis{\sphinxupquote{optional}}\sphinxstyleliteralemphasis{\sphinxupquote{, }}\sphinxstyleliteralemphasis{\sphinxupquote{default=None}}) \textendash{} string labels of the random variables

\end{itemize}

\end{description}\end{quote}
\index{get\_coeffs\_expand() (pygpc.quad.Quad method)}

\begin{fulllineitems}
\phantomsection\label{\detokenize{pygpc:pygpc.quad.Quad.get_coeffs_expand}}\pysiglinewithargsret{\sphinxbfcode{\sphinxupquote{get\_coeffs\_expand}}}{\emph{sim\_results}}{}
Determine the gPC coefficients by the quadrature method

coeffs = get\_coeffs\_expand(self, sim\_results)
\begin{quote}\begin{description}
\item[{Parameters}] \leavevmode
\sphinxstyleliteralstrong{\sphinxupquote{sim\_results}} (\sphinxstyleliteralemphasis{\sphinxupquote{{[}}}\sphinxstyleliteralemphasis{\sphinxupquote{N\_grid x N\_out}}\sphinxstyleliteralemphasis{\sphinxupquote{{]} }}\sphinxstyleliteralemphasis{\sphinxupquote{np.ndarray of float}}) \textendash{} results from simulations with N\_out output quantities

\item[{Returns}] \leavevmode
\sphinxstylestrong{coeffs} \textendash{} gPC coefficients

\item[{Return type}] \leavevmode
{[}N\_coeffs x N\_out{]} np.ndarray of float

\end{description}\end{quote}

\end{fulllineitems}


\end{fulllineitems}



\section{pygpc.reg module}
\label{\detokenize{pygpc:module-pygpc.reg}}\label{\detokenize{pygpc:pygpc-reg-module}}\index{pygpc.reg (module)}
Class that provides polynomial chaos regression methods
\index{Reg (class in pygpc.reg)}

\begin{fulllineitems}
\phantomsection\label{\detokenize{pygpc:pygpc.reg.Reg}}\pysiglinewithargsret{\sphinxbfcode{\sphinxupquote{class }}\sphinxcode{\sphinxupquote{pygpc.reg.}}\sphinxbfcode{\sphinxupquote{Reg}}}{\emph{pdf\_type}, \emph{pdf\_shape}, \emph{limits}, \emph{order}, \emph{order\_max}, \emph{interaction\_order}, \emph{grid}, \emph{random\_vars=None}}{}
Bases: {\hyperref[\detokenize{pygpc:pygpc.gpc.gPC}]{\sphinxcrossref{\sphinxcode{\sphinxupquote{pygpc.gpc.gPC}}}}}

Regression gPC subclass

Reg(pdf\_type, pdf\_shape, limits, order, order\_max, interaction\_order, grid, random\_vars=None)
\index{N\_grid (pygpc.reg.Reg attribute)}

\begin{fulllineitems}
\phantomsection\label{\detokenize{pygpc:pygpc.reg.Reg.N_grid}}\pysigline{\sphinxbfcode{\sphinxupquote{N\_grid}}}
number of grid points
\begin{quote}\begin{description}
\item[{Type}] \leavevmode
int

\end{description}\end{quote}

\end{fulllineitems}

\index{dim (pygpc.reg.Reg attribute)}

\begin{fulllineitems}
\phantomsection\label{\detokenize{pygpc:pygpc.reg.Reg.dim}}\pysigline{\sphinxbfcode{\sphinxupquote{dim}}}
number of uncertain parameters to process
\begin{quote}\begin{description}
\item[{Type}] \leavevmode
int

\end{description}\end{quote}

\end{fulllineitems}

\index{pdf\_type (pygpc.reg.Reg attribute)}

\begin{fulllineitems}
\phantomsection\label{\detokenize{pygpc:pygpc.reg.Reg.pdf_type}}\pysigline{\sphinxbfcode{\sphinxupquote{pdf\_type}}}
type of pdf ‘beta’ or ‘norm’
\begin{quote}\begin{description}
\item[{Type}] \leavevmode
{[}dim{]} list of str

\end{description}\end{quote}

\end{fulllineitems}

\index{pdf\_shape (pygpc.reg.Reg attribute)}

\begin{fulllineitems}
\phantomsection\label{\detokenize{pygpc:pygpc.reg.Reg.pdf_shape}}\pysigline{\sphinxbfcode{\sphinxupquote{pdf\_shape}}}
shape parameters of pdfs
beta-dist:   {[}{[}alpha{]}, {[}beta{]}    {]}
normal-dist: {[}{[}mean{]},  {[}variance{]}{]}
\begin{quote}\begin{description}
\item[{Type}] \leavevmode
list of list of float

\end{description}\end{quote}

\end{fulllineitems}

\index{limits (pygpc.reg.Reg attribute)}

\begin{fulllineitems}
\phantomsection\label{\detokenize{pygpc:pygpc.reg.Reg.limits}}\pysigline{\sphinxbfcode{\sphinxupquote{limits}}}
upper and lower bounds of random variables
beta-dist:   {[}{[}a1 …{]}, {[}b1 …{]}{]}
normal-dist: {[}{[}0 … {]}, {[}0 … {]}{]} (not used)
\begin{quote}\begin{description}
\item[{Type}] \leavevmode
list of list of float

\end{description}\end{quote}

\end{fulllineitems}

\index{order (pygpc.reg.Reg attribute)}

\begin{fulllineitems}
\phantomsection\label{\detokenize{pygpc:pygpc.reg.Reg.order}}\pysigline{\sphinxbfcode{\sphinxupquote{order}}}
maximum individual expansion order
generates individual polynomials also if maximum expansion order in order\_max is exceeded
\begin{quote}\begin{description}
\item[{Type}] \leavevmode
{[}dim{]} list of int

\end{description}\end{quote}

\end{fulllineitems}

\index{order\_max (pygpc.reg.Reg attribute)}

\begin{fulllineitems}
\phantomsection\label{\detokenize{pygpc:pygpc.reg.Reg.order_max}}\pysigline{\sphinxbfcode{\sphinxupquote{order\_max}}}
maximum expansion order (sum of all exponents)
the maximum expansion order considers the sum of the orders of combined polynomials only
\begin{quote}\begin{description}
\item[{Type}] \leavevmode
int

\end{description}\end{quote}

\end{fulllineitems}

\index{interaction\_order (pygpc.reg.Reg attribute)}

\begin{fulllineitems}
\phantomsection\label{\detokenize{pygpc:pygpc.reg.Reg.interaction_order}}\pysigline{\sphinxbfcode{\sphinxupquote{interaction\_order}}}
number of random variables, which can interact with each other
all polynomials are ignored, which have an interaction order greater than the specified
\begin{quote}\begin{description}
\item[{Type}] \leavevmode
int

\end{description}\end{quote}

\end{fulllineitems}

\index{grid (pygpc.reg.Reg attribute)}

\begin{fulllineitems}
\phantomsection\label{\detokenize{pygpc:pygpc.reg.Reg.grid}}\pysigline{\sphinxbfcode{\sphinxupquote{grid}}}
grid object generated in grid.py including grid.coords and grid.coords\_norm
\begin{quote}\begin{description}
\item[{Type}] \leavevmode
grid object

\end{description}\end{quote}

\end{fulllineitems}

\index{random\_vars (pygpc.reg.Reg attribute)}

\begin{fulllineitems}
\phantomsection\label{\detokenize{pygpc:pygpc.reg.Reg.random_vars}}\pysigline{\sphinxbfcode{\sphinxupquote{random\_vars}}}
string labels of the random variables
\begin{quote}\begin{description}
\item[{Type}] \leavevmode
{[}dim{]} list of str

\end{description}\end{quote}

\end{fulllineitems}

\index{relative\_error\_loocv (pygpc.reg.Reg attribute)}

\begin{fulllineitems}
\phantomsection\label{\detokenize{pygpc:pygpc.reg.Reg.relative_error_loocv}}\pysigline{\sphinxbfcode{\sphinxupquote{relative\_error\_loocv}}}
relative error of the leave-one-out-cross-validation
\begin{quote}\begin{description}
\item[{Type}] \leavevmode
list of float

\end{description}\end{quote}

\end{fulllineitems}

\index{nan\_elm (pygpc.reg.Reg attribute)}

\begin{fulllineitems}
\phantomsection\label{\detokenize{pygpc:pygpc.reg.Reg.nan_elm}}\pysigline{\sphinxbfcode{\sphinxupquote{nan\_elm}}}
which elements were dropped due to NaN
\begin{quote}\begin{description}
\item[{Type}] \leavevmode
list of float

\end{description}\end{quote}

\end{fulllineitems}

\begin{quote}\begin{description}
\item[{Parameters}] \leavevmode\begin{itemize}
\item {} 
\sphinxstyleliteralstrong{\sphinxupquote{pdf\_type}} (\sphinxstyleliteralemphasis{\sphinxupquote{{[}}}\sphinxstyleliteralemphasis{\sphinxupquote{dim}}\sphinxstyleliteralemphasis{\sphinxupquote{{]} }}\sphinxstyleliteralemphasis{\sphinxupquote{list of str}}) \textendash{} type of pdf ‘beta’ or ‘norm’

\item {} 
\sphinxstyleliteralstrong{\sphinxupquote{pdf\_shape}} (\sphinxstyleliteralemphasis{\sphinxupquote{list of list of float}}) \textendash{} shape parameters of pdfs
beta-dist:   {[}{[}alpha{]}, {[}beta{]}    {]}
normal-dist: {[}{[}mean{]},  {[}variance{]}{]}

\item {} 
\sphinxstyleliteralstrong{\sphinxupquote{limits}} (\sphinxstyleliteralemphasis{\sphinxupquote{list of list of float}}) \textendash{} upper and lower bounds of random variables
beta-dist:   {[}{[}a1 …{]}, {[}b1 …{]}{]}
normal-dist: {[}{[}0 … {]}, {[}0 … {]}{]} (not used)

\item {} 
\sphinxstyleliteralstrong{\sphinxupquote{order}} (\sphinxstyleliteralemphasis{\sphinxupquote{{[}}}\sphinxstyleliteralemphasis{\sphinxupquote{dim}}\sphinxstyleliteralemphasis{\sphinxupquote{{]} }}\sphinxstyleliteralemphasis{\sphinxupquote{list of int}}) \textendash{} maximum individual expansion order
generates individual polynomials also if maximum expansion order in order\_max is exceeded

\item {} 
\sphinxstyleliteralstrong{\sphinxupquote{order\_max}} (\sphinxstyleliteralemphasis{\sphinxupquote{int}}) \textendash{} maximum expansion order (sum of all exponents)
the maximum expansion order considers the sum of the orders of combined polynomials only

\item {} 
\sphinxstyleliteralstrong{\sphinxupquote{interaction\_order}} (\sphinxstyleliteralemphasis{\sphinxupquote{int}}) \textendash{} number of random variables, which can interact with each other
all polynomials are ignored, which have an interaction order greater than the specified

\item {} 
\sphinxstyleliteralstrong{\sphinxupquote{grid}} (\sphinxstyleliteralemphasis{\sphinxupquote{grid object}}) \textendash{} grid object generated in grid.py including grid.coords and grid.coords\_norm

\item {} 
\sphinxstyleliteralstrong{\sphinxupquote{random\_vars}} (\sphinxstyleliteralemphasis{\sphinxupquote{{[}}}\sphinxstyleliteralemphasis{\sphinxupquote{dim}}\sphinxstyleliteralemphasis{\sphinxupquote{{]} }}\sphinxstyleliteralemphasis{\sphinxupquote{list of str}}\sphinxstyleliteralemphasis{\sphinxupquote{, }}\sphinxstyleliteralemphasis{\sphinxupquote{optional}}\sphinxstyleliteralemphasis{\sphinxupquote{, }}\sphinxstyleliteralemphasis{\sphinxupquote{default=None}}) \textendash{} string labels of the random variables

\end{itemize}

\end{description}\end{quote}
\index{get\_coeffs\_expand() (pygpc.reg.Reg method)}

\begin{fulllineitems}
\phantomsection\label{\detokenize{pygpc:pygpc.reg.Reg.get_coeffs_expand}}\pysiglinewithargsret{\sphinxbfcode{\sphinxupquote{get\_coeffs\_expand}}}{\emph{sim\_results}}{}
Determine the gPC coefficients by the regression method.

coeffs = get\_coeffs\_expand(sim\_results)
\begin{quote}\begin{description}
\item[{Parameters}] \leavevmode
\sphinxstyleliteralstrong{\sphinxupquote{sim\_results}} (\sphinxstyleliteralemphasis{\sphinxupquote{{[}}}\sphinxstyleliteralemphasis{\sphinxupquote{N\_grid x N\_out}}\sphinxstyleliteralemphasis{\sphinxupquote{{]} }}\sphinxstyleliteralemphasis{\sphinxupquote{np.ndarray of float}}) \textendash{} results from simulations with N\_out output quantities,

\item[{Returns}] \leavevmode
\sphinxstylestrong{coeffs} \textendash{} gPC coefficients

\item[{Return type}] \leavevmode
{[}N\_coeffs x N\_out{]} np.ndarray of float

\end{description}\end{quote}

\end{fulllineitems}

\index{get\_loocv() (pygpc.reg.Reg method)}

\begin{fulllineitems}
\phantomsection\label{\detokenize{pygpc:pygpc.reg.Reg.get_loocv}}\pysiglinewithargsret{\sphinxbfcode{\sphinxupquote{get\_loocv}}}{\emph{sim\_results}}{}
Perform leave one out cross validation of gPC with maximal 100 points
and add result to self.relative\_error\_loocv.

relative\_error\_loocv = get\_loocv(sim\_results)
\begin{quote}\begin{description}
\item[{Parameters}] \leavevmode
\sphinxstyleliteralstrong{\sphinxupquote{sim\_results}} (\sphinxstyleliteralemphasis{\sphinxupquote{{[}}}\sphinxstyleliteralemphasis{\sphinxupquote{N\_grid x N\_out}}\sphinxstyleliteralemphasis{\sphinxupquote{{]} }}\sphinxstyleliteralemphasis{\sphinxupquote{np.ndarray}}) \textendash{} Results from N\_grid simulations with N\_out output quantities

\item[{Returns}] \leavevmode
\sphinxstylestrong{relative\_error\_loocv} \textendash{} relative mean error of leave one out cross validation

\item[{Return type}] \leavevmode
float

\end{description}\end{quote}

\end{fulllineitems}


\end{fulllineitems}



\section{pygpc.rw module}
\label{\detokenize{pygpc:module-pygpc.rw}}\label{\detokenize{pygpc:pygpc-rw-module}}\index{pygpc.rw (module)}
Functions that provide input and output functionality
\index{read\_data\_hdf5() (in module pygpc.rw)}

\begin{fulllineitems}
\phantomsection\label{\detokenize{pygpc:pygpc.rw.read_data_hdf5}}\pysiglinewithargsret{\sphinxcode{\sphinxupquote{pygpc.rw.}}\sphinxbfcode{\sphinxupquote{read\_data\_hdf5}}}{\emph{fname}, \emph{loc}}{}
Read data from .hdf5 file (e.g. coeffs, mean, std, …).

load\_data\_hdf5(fname, loc)
\begin{quote}\begin{description}
\item[{Parameters}] \leavevmode\begin{itemize}
\item {} 
\sphinxstyleliteralstrong{\sphinxupquote{fname}} (\sphinxstyleliteralemphasis{\sphinxupquote{str}}) \textendash{} path to input file

\item {} 
\sphinxstyleliteralstrong{\sphinxupquote{loc}} (\sphinxstyleliteralemphasis{\sphinxupquote{str}}) \textendash{} location (folder and name) in hdf5 file (e.g. data/phi)

\end{itemize}

\end{description}\end{quote}

\end{fulllineitems}

\index{read\_gpc\_pkl() (in module pygpc.rw)}

\begin{fulllineitems}
\phantomsection\label{\detokenize{pygpc:pygpc.rw.read_gpc_pkl}}\pysiglinewithargsret{\sphinxcode{\sphinxupquote{pygpc.rw.}}\sphinxbfcode{\sphinxupquote{read\_gpc\_pkl}}}{\emph{fname}}{}
Read gPC object including infos about input pdfs, polynomials, grid etc.

object = read\_gpc\_obj(fname)
\begin{quote}\begin{description}
\item[{Parameters}] \leavevmode
\sphinxstyleliteralstrong{\sphinxupquote{fname}} (\sphinxstyleliteralemphasis{\sphinxupquote{str}}) \textendash{} path to input file

\end{description}\end{quote}

\end{fulllineitems}

\index{read\_gpc\_yml() (in module pygpc.rw)}

\begin{fulllineitems}
\phantomsection\label{\detokenize{pygpc:pygpc.rw.read_gpc_yml}}\pysiglinewithargsret{\sphinxcode{\sphinxupquote{pygpc.rw.}}\sphinxbfcode{\sphinxupquote{read\_gpc\_yml}}}{\emph{fname}}{}
Read gPC infos about input pdfs, polynomials, grid etc. as .yml file and initialize gpc object.

obj = read\_gpc\_yml(fname)
\begin{quote}\begin{description}
\item[{Parameters}] \leavevmode
\sphinxstyleliteralstrong{\sphinxupquote{fname}} (\sphinxstyleliteralemphasis{\sphinxupquote{str}}) \textendash{} path to input file

\end{description}\end{quote}

\end{fulllineitems}

\index{read\_sobol\_idx\_txt() (in module pygpc.rw)}

\begin{fulllineitems}
\phantomsection\label{\detokenize{pygpc:pygpc.rw.read_sobol_idx_txt}}\pysiglinewithargsret{\sphinxcode{\sphinxupquote{pygpc.rw.}}\sphinxbfcode{\sphinxupquote{read\_sobol\_idx\_txt}}}{\emph{fname}}{}
Read sobol\_idx list from file.

read\_sobol\_idx\_txt(fname)
\begin{quote}\begin{description}
\item[{Parameters}] \leavevmode
\sphinxstyleliteralstrong{\sphinxupquote{fname}} (\sphinxstyleliteralemphasis{\sphinxupquote{str}}) \textendash{} path to input file

\item[{Returns}] \leavevmode
\sphinxstylestrong{sobol\_idx} \textendash{} list of parameter label indices belonging to Sobol indices

\item[{Return type}] \leavevmode
{[}N\_sobol{]} list of np.array

\end{description}\end{quote}

\end{fulllineitems}

\index{write\_data\_hdf5() (in module pygpc.rw)}

\begin{fulllineitems}
\phantomsection\label{\detokenize{pygpc:pygpc.rw.write_data_hdf5}}\pysiglinewithargsret{\sphinxcode{\sphinxupquote{pygpc.rw.}}\sphinxbfcode{\sphinxupquote{write\_data\_hdf5}}}{\emph{data}, \emph{fname}, \emph{loc}}{}
Write quantity of interest in .hdf5 file (e.g. coeffs, mean, std, …).

write\_data\_hdf5(data, fname, loc)
\begin{quote}\begin{description}
\item[{Parameters}] \leavevmode\begin{itemize}
\item {} 
\sphinxstyleliteralstrong{\sphinxupquote{data}} (\sphinxstyleliteralemphasis{\sphinxupquote{np.ndarray}}) \textendash{} data to save

\item {} 
\sphinxstyleliteralstrong{\sphinxupquote{fname}} (\sphinxstyleliteralemphasis{\sphinxupquote{str}}) \textendash{} path to output file

\item {} 
\sphinxstyleliteralstrong{\sphinxupquote{loc}} (\sphinxstyleliteralemphasis{\sphinxupquote{str}}) \textendash{} location (folder and name) in hdf5 file (e.g. data/phi)

\end{itemize}

\end{description}\end{quote}

\end{fulllineitems}

\index{write\_data\_txt() (in module pygpc.rw)}

\begin{fulllineitems}
\phantomsection\label{\detokenize{pygpc:pygpc.rw.write_data_txt}}\pysiglinewithargsret{\sphinxcode{\sphinxupquote{pygpc.rw.}}\sphinxbfcode{\sphinxupquote{write\_data\_txt}}}{\emph{data}, \emph{fname}}{}
Write data (quantity of interest) in .txt file (e.g. coeffs, mean, std, …).

write\_data\_txt(data, fname)
\begin{quote}\begin{description}
\item[{Parameters}] \leavevmode\begin{itemize}
\item {} 
\sphinxstyleliteralstrong{\sphinxupquote{data}} (\sphinxstyleliteralemphasis{\sphinxupquote{np.ndarray}}) \textendash{} data to save

\item {} 
\sphinxstyleliteralstrong{\sphinxupquote{fname}} (\sphinxstyleliteralemphasis{\sphinxupquote{str}}) \textendash{} path to output file

\end{itemize}

\end{description}\end{quote}

\end{fulllineitems}

\index{write\_gpc\_pkl() (in module pygpc.rw)}

\begin{fulllineitems}
\phantomsection\label{\detokenize{pygpc:pygpc.rw.write_gpc_pkl}}\pysiglinewithargsret{\sphinxcode{\sphinxupquote{pygpc.rw.}}\sphinxbfcode{\sphinxupquote{write\_gpc\_pkl}}}{\emph{obj}, \emph{fname}}{}
Write gPC object including infos about input pdfs, polynomials, grid etc. as pickle file.

write\_gpc\_obj(obj, fname)
\begin{quote}\begin{description}
\item[{Parameters}] \leavevmode\begin{itemize}
\item {} 
\sphinxstyleliteralstrong{\sphinxupquote{obj}} ({\hyperref[\detokenize{pygpc:pygpc.gpc.gPC}]{\sphinxcrossref{\sphinxstyleliteralemphasis{\sphinxupquote{gPC}}}}}\sphinxstyleliteralemphasis{\sphinxupquote{ or }}\sphinxstyleliteralemphasis{\sphinxupquote{derived class}}) \textendash{} class instance containing gpc data

\item {} 
\sphinxstyleliteralstrong{\sphinxupquote{fname}} (\sphinxstyleliteralemphasis{\sphinxupquote{str}}) \textendash{} path to output file

\end{itemize}

\end{description}\end{quote}

\end{fulllineitems}

\index{write\_gpc\_yml() (in module pygpc.rw)}

\begin{fulllineitems}
\phantomsection\label{\detokenize{pygpc:pygpc.rw.write_gpc_yml}}\pysiglinewithargsret{\sphinxcode{\sphinxupquote{pygpc.rw.}}\sphinxbfcode{\sphinxupquote{write\_gpc\_yml}}}{\emph{obj}, \emph{fname}}{}
Write gPC infos about input pdfs, polynomials, grid etc. as .yml file.

write\_gpc\_yml(obj, fname)
\begin{quote}\begin{description}
\item[{Parameters}] \leavevmode\begin{itemize}
\item {} 
\sphinxstyleliteralstrong{\sphinxupquote{obj}} ({\hyperref[\detokenize{pygpc:pygpc.gpc.gPC}]{\sphinxcrossref{\sphinxstyleliteralemphasis{\sphinxupquote{gPC}}}}}\sphinxstyleliteralemphasis{\sphinxupquote{ or }}\sphinxstyleliteralemphasis{\sphinxupquote{derived class}}) \textendash{} class instance containing gpc data

\item {} 
\sphinxstyleliteralstrong{\sphinxupquote{fname}} (\sphinxstyleliteralemphasis{\sphinxupquote{str}}) \textendash{} path to output file

\end{itemize}

\end{description}\end{quote}

\end{fulllineitems}

\index{write\_sobol\_idx\_txt() (in module pygpc.rw)}

\begin{fulllineitems}
\phantomsection\label{\detokenize{pygpc:pygpc.rw.write_sobol_idx_txt}}\pysiglinewithargsret{\sphinxcode{\sphinxupquote{pygpc.rw.}}\sphinxbfcode{\sphinxupquote{write\_sobol\_idx\_txt}}}{\emph{sobol\_idx}, \emph{fname}}{}
Write sobol\_idx list in file.

write\_sobol\_idx\_txt(sobol\_idx, filename)
\begin{quote}\begin{description}
\item[{Parameters}] \leavevmode\begin{itemize}
\item {} 
\sphinxstyleliteralstrong{\sphinxupquote{sobol\_idx}} (\sphinxstyleliteralemphasis{\sphinxupquote{{[}}}\sphinxstyleliteralemphasis{\sphinxupquote{N\_sobol}}\sphinxstyleliteralemphasis{\sphinxupquote{{]} }}\sphinxstyleliteralemphasis{\sphinxupquote{list of np.ndarray}}) \textendash{} list of parameter label indices belonging to Sobol indices

\item {} 
\sphinxstyleliteralstrong{\sphinxupquote{fname}} (\sphinxstyleliteralemphasis{\sphinxupquote{str}}) \textendash{} path to output file

\end{itemize}

\end{description}\end{quote}

\end{fulllineitems}



\section{pygpc.testfun module}
\label{\detokenize{pygpc:module-pygpc.testfun}}\label{\detokenize{pygpc:pygpc-testfun-module}}\index{pygpc.testfun (module)}
Testfunctions and electromagnetic field calculations
\index{calc\_B\_field\_outside\_sphere() (in module pygpc.testfun)}

\begin{fulllineitems}
\phantomsection\label{\detokenize{pygpc:pygpc.testfun.calc_B_field_outside_sphere}}\pysiglinewithargsret{\sphinxcode{\sphinxupquote{pygpc.testfun.}}\sphinxbfcode{\sphinxupquote{calc\_B\_field\_outside\_sphere}}}{\emph{sphere\_radius}, \emph{dipole\_pos}, \emph{dipole\_moment}, \emph{detector\_positions}}{}
Calculate the B field outside a sphere, does not depend on conductivity.
Dipole in SI units, positions in mm

J.Savras - Basic mathematical and electromagnetic concepts of the biomagnetic inverse problem

B = calc\_B\_field\_outside\_sphere(sphere\_radius, dipole\_pos, dipole\_moment, detector\_positions)
\begin{quote}\begin{description}
\item[{Parameters}] \leavevmode\begin{itemize}
\item {} 
\sphinxstyleliteralstrong{\sphinxupquote{sphere\_radius}} (\sphinxstyleliteralemphasis{\sphinxupquote{float}}) \textendash{} radius of sphere

\item {} 
\sphinxstyleliteralstrong{\sphinxupquote{dipole\_pos}} (\sphinxstyleliteralemphasis{\sphinxupquote{{[}}}\sphinxstyleliteralemphasis{\sphinxupquote{3 x 1}}\sphinxstyleliteralemphasis{\sphinxupquote{{]} }}\sphinxstyleliteralemphasis{\sphinxupquote{np.ndarray}}) \textendash{} position of dipole

\item {} 
\sphinxstyleliteralstrong{\sphinxupquote{dipole\_moment}} (\sphinxstyleliteralemphasis{\sphinxupquote{{[}}}\sphinxstyleliteralemphasis{\sphinxupquote{3 x 1}}\sphinxstyleliteralemphasis{\sphinxupquote{{]} }}\sphinxstyleliteralemphasis{\sphinxupquote{np.ndarray}}) \textendash{} moment of dipole

\item {} 
\sphinxstyleliteralstrong{\sphinxupquote{detector\_positions}} (\sphinxstyleliteralemphasis{\sphinxupquote{{[}}}\sphinxstyleliteralemphasis{\sphinxupquote{n x 3}}\sphinxstyleliteralemphasis{\sphinxupquote{{]} }}\sphinxstyleliteralemphasis{\sphinxupquote{np.ndarray}}) \textendash{} position of detectors, must lie outside sphere

\end{itemize}

\item[{Returns}] \leavevmode
\sphinxstylestrong{B} \textendash{} array with B fields in detector positions

\item[{Return type}] \leavevmode
{[}N x 3{]} np.ndarray

\end{description}\end{quote}

\end{fulllineitems}

\index{calc\_fibonacci\_sphere() (in module pygpc.testfun)}

\begin{fulllineitems}
\phantomsection\label{\detokenize{pygpc:pygpc.testfun.calc_fibonacci_sphere}}\pysiglinewithargsret{\sphinxcode{\sphinxupquote{pygpc.testfun.}}\sphinxbfcode{\sphinxupquote{calc\_fibonacci\_sphere}}}{\emph{nr\_points}, \emph{R=1}}{}
Creates N points around evenly spread through a unit sphere.

points = calc\_fibonacci\_sphere(nr\_points, R=1)
\begin{quote}\begin{description}
\item[{Parameters}] \leavevmode\begin{itemize}
\item {} 
\sphinxstyleliteralstrong{\sphinxupquote{nr\_points}} (\sphinxstyleliteralemphasis{\sphinxupquote{int}}) \textendash{} number of points to be spread, must be odd

\item {} 
\sphinxstyleliteralstrong{\sphinxupquote{R}} (\sphinxstyleliteralemphasis{\sphinxupquote{float}}\sphinxstyleliteralemphasis{\sphinxupquote{, }}\sphinxstyleliteralemphasis{\sphinxupquote{optional}}\sphinxstyleliteralemphasis{\sphinxupquote{, }}\sphinxstyleliteralemphasis{\sphinxupquote{default=1}}) \textendash{} radius of sphere

\end{itemize}

\item[{Returns}] \leavevmode
\sphinxstylestrong{points} \textendash{} evenly spread points through a unit sphere

\item[{Return type}] \leavevmode
{[}N x 3{]} np.ndarray

\end{description}\end{quote}

\end{fulllineitems}

\index{calc\_potential\_dipole\_3layers() (in module pygpc.testfun)}

\begin{fulllineitems}
\phantomsection\label{\detokenize{pygpc:pygpc.testfun.calc_potential_dipole_3layers}}\pysiglinewithargsret{\sphinxcode{\sphinxupquote{pygpc.testfun.}}\sphinxbfcode{\sphinxupquote{calc\_potential\_dipole\_3layers}}}{\emph{radii}, \emph{cond\_brain\_scalp}, \emph{cond\_skull}, \emph{dipole\_pos}, \emph{dipole\_moment}, \emph{surface\_points}, \emph{nbr\_polynomials=100}}{}
Calculates the electric potential in a 3-layered sphere caused by a dipole
Calculations assumes dimensions in SI units

Ary, James P., Stanley A. Klein, and Derek H. Fender.
“Location of sources of evoked scalp potentials: corrections for skull and scalp thicknesses.”
Biomedical Engineering 28.6 (1981).
eq. 2 and 2a
\begin{quote}\begin{description}
\item[{Parameters}] \leavevmode\begin{itemize}
\item {} 
\sphinxstyleliteralstrong{\sphinxupquote{radii}} (\sphinxstyleliteralemphasis{\sphinxupquote{{[}}}\sphinxstyleliteralemphasis{\sphinxupquote{3}}\sphinxstyleliteralemphasis{\sphinxupquote{{]}}}\sphinxstyleliteralemphasis{\sphinxupquote{list}}) \textendash{} radius of each of the 3 layers (innermost to outermost), in mm

\item {} 
\sphinxstyleliteralstrong{\sphinxupquote{cond\_brain\_scalp}} (\sphinxstyleliteralemphasis{\sphinxupquote{float}}) \textendash{} conductivity of the brain and scalp layers, in S/m

\item {} 
\sphinxstyleliteralstrong{\sphinxupquote{cond\_skull}} (\sphinxstyleliteralemphasis{\sphinxupquote{float}}) \textendash{} conductivity of the skull layer, in S/m

\item {} 
\sphinxstyleliteralstrong{\sphinxupquote{dipole\_pos}} (\sphinxstyleliteralemphasis{\sphinxupquote{{[}}}\sphinxstyleliteralemphasis{\sphinxupquote{3 x 1}}\sphinxstyleliteralemphasis{\sphinxupquote{{]} }}\sphinxstyleliteralemphasis{\sphinxupquote{np.ndarray}}) \textendash{} position of the dipole, in mm

\item {} 
\sphinxstyleliteralstrong{\sphinxupquote{dipole\_moment}} (\sphinxstyleliteralemphasis{\sphinxupquote{{[}}}\sphinxstyleliteralemphasis{\sphinxupquote{3 x 1}}\sphinxstyleliteralemphasis{\sphinxupquote{{]} }}\sphinxstyleliteralemphasis{\sphinxupquote{np.ndarray}}) \textendash{} moment of dipole, in C x m

\item {} 
\sphinxstyleliteralstrong{\sphinxupquote{surface\_points}} (\sphinxstyleliteralemphasis{\sphinxupquote{{[}}}\sphinxstyleliteralemphasis{\sphinxupquote{N x 3}}\sphinxstyleliteralemphasis{\sphinxupquote{{]} }}\sphinxstyleliteralemphasis{\sphinxupquote{np.ndarray}}) \textendash{} list of positions where the poteitial should be calculated, in mm

\item {} 
\sphinxstyleliteralstrong{\sphinxupquote{nbr\_polynomials}} (\sphinxstyleliteralemphasis{\sphinxupquote{int}}) \textendash{} number of of legendre polynomials to use (default = 100)

\end{itemize}

\item[{Returns}] \leavevmode
\sphinxstylestrong{potential} \textendash{} values of the electric potential, in V

\item[{Return type}] \leavevmode
{[}N x 1{]} np.ndarray

\end{description}\end{quote}

\end{fulllineitems}

\index{calc\_potential\_homogeneous\_dipole() (in module pygpc.testfun)}

\begin{fulllineitems}
\phantomsection\label{\detokenize{pygpc:pygpc.testfun.calc_potential_homogeneous_dipole}}\pysiglinewithargsret{\sphinxcode{\sphinxupquote{pygpc.testfun.}}\sphinxbfcode{\sphinxupquote{calc\_potential\_homogeneous\_dipole}}}{\emph{sphere\_radius}, \emph{conductivity}, \emph{dipole\_pos}, \emph{dipole\_moment}, \emph{detector\_positions}}{}
Calculate the surface potential generated by a dipole inside a homogeneous conducting sphere.

Dezhong Yao, Electric Potential Produced by a Dipole in a Homogeneous
Conducting Sphere
\begin{description}
\item[{potential = calc\_potential\_homogeneous\_dipole(sphere\_radius, conductivity, dipole\_pos, dipole\_moment,}] \leavevmode
detector\_positions):

\end{description}
\begin{quote}\begin{description}
\item[{Parameters}] \leavevmode\begin{itemize}
\item {} 
\sphinxstyleliteralstrong{\sphinxupquote{sphere\_radius}} (\sphinxstyleliteralemphasis{\sphinxupquote{float}}) \textendash{} radius of sphere, in mm

\item {} 
\sphinxstyleliteralstrong{\sphinxupquote{conductivity}} (\sphinxstyleliteralemphasis{\sphinxupquote{float}}) \textendash{} conductivity of medium, in S/m

\item {} 
\sphinxstyleliteralstrong{\sphinxupquote{dipole\_pos}} (\sphinxstyleliteralemphasis{\sphinxupquote{{[}}}\sphinxstyleliteralemphasis{\sphinxupquote{3 x 1}}\sphinxstyleliteralemphasis{\sphinxupquote{{]} }}\sphinxstyleliteralemphasis{\sphinxupquote{np.ndarray}}) \textendash{} position of dipole, in mm

\item {} 
\sphinxstyleliteralstrong{\sphinxupquote{dipole\_moment}} (\sphinxstyleliteralemphasis{\sphinxupquote{{[}}}\sphinxstyleliteralemphasis{\sphinxupquote{3 x 1}}\sphinxstyleliteralemphasis{\sphinxupquote{{]} }}\sphinxstyleliteralemphasis{\sphinxupquote{np.ndarray}}) \textendash{} moment of dipole, in C.m

\item {} 
\sphinxstyleliteralstrong{\sphinxupquote{detector\_positions}} (\sphinxstyleliteralemphasis{\sphinxupquote{{[}}}\sphinxstyleliteralemphasis{\sphinxupquote{n x 3}}\sphinxstyleliteralemphasis{\sphinxupquote{{]} }}\sphinxstyleliteralemphasis{\sphinxupquote{np.ndarray}}) \textendash{} position of detectors, will be projected into the sphere surface, in mm

\end{itemize}

\item[{Returns}] \leavevmode
\sphinxstylestrong{potential} \textendash{} potential at the points

\item[{Return type}] \leavevmode
{[}n x 1{]} np.ndarray

\end{description}\end{quote}

\end{fulllineitems}

\index{calc\_potentials\_3layers\_surface\_electrodes() (in module pygpc.testfun)}

\begin{fulllineitems}
\phantomsection\label{\detokenize{pygpc:pygpc.testfun.calc_potentials_3layers_surface_electrodes}}\pysiglinewithargsret{\sphinxcode{\sphinxupquote{pygpc.testfun.}}\sphinxbfcode{\sphinxupquote{calc\_potentials\_3layers\_surface\_electrodes}}}{\emph{conductivities}, \emph{radii}, \emph{anode\_pos}, \emph{cathode\_pos}, \emph{p}, \emph{nbr\_polynomials=50}}{}
Calculate the electric potential in a 3-layered sphere caused by point-like electrodes.

S.Rush, D.Driscol EEG electrode sensitivity\textendash{}an application of reciprocity
\begin{description}
\item[{potential = calc\_potentials\_3layers\_surface\_electrodes(conductivities, radii, anode\_pos, cathode\_pos, p,}] \leavevmode
nbr\_polynomials=50):

\end{description}
\begin{quote}\begin{description}
\item[{Parameters}] \leavevmode\begin{itemize}
\item {} 
\sphinxstyleliteralstrong{\sphinxupquote{conductivities}} (\sphinxstyleliteralemphasis{\sphinxupquote{{[}}}\sphinxstyleliteralemphasis{\sphinxupquote{3}}\sphinxstyleliteralemphasis{\sphinxupquote{{]} }}\sphinxstyleliteralemphasis{\sphinxupquote{list}}) \textendash{} conductivity of the 3 layers (innermost to outermost), in S/m

\item {} 
\sphinxstyleliteralstrong{\sphinxupquote{radii}} (\sphinxstyleliteralemphasis{\sphinxupquote{{[}}}\sphinxstyleliteralemphasis{\sphinxupquote{3}}\sphinxstyleliteralemphasis{\sphinxupquote{{]} }}\sphinxstyleliteralemphasis{\sphinxupquote{list}}) \textendash{} radius of each of the 3 layers (innermost to outermost), in mm

\item {} 
\sphinxstyleliteralstrong{\sphinxupquote{anode\_pos}} (\sphinxstyleliteralemphasis{\sphinxupquote{{[}}}\sphinxstyleliteralemphasis{\sphinxupquote{3 x 1}}\sphinxstyleliteralemphasis{\sphinxupquote{{]} }}\sphinxstyleliteralemphasis{\sphinxupquote{np.ndarray}}) \textendash{} position of the anode\_pos, in mm

\item {} 
\sphinxstyleliteralstrong{\sphinxupquote{cathode\_pos}} (\sphinxstyleliteralemphasis{\sphinxupquote{{[}}}\sphinxstyleliteralemphasis{\sphinxupquote{3 x 1}}\sphinxstyleliteralemphasis{\sphinxupquote{{]} }}\sphinxstyleliteralemphasis{\sphinxupquote{np.ndarray}}) \textendash{} position of cathode\_pos, in mm

\item {} 
\sphinxstyleliteralstrong{\sphinxupquote{p}} (\sphinxstyleliteralemphasis{\sphinxupquote{{[}}}\sphinxstyleliteralemphasis{\sphinxupquote{N x 3}}\sphinxstyleliteralemphasis{\sphinxupquote{{]} }}\sphinxstyleliteralemphasis{\sphinxupquote{np.ndarray}}) \textendash{} list of positions where the poteitial should be calculated, in mm

\item {} 
\sphinxstyleliteralstrong{\sphinxupquote{nbr\_polynomials}} (\sphinxstyleliteralemphasis{\sphinxupquote{int}}\sphinxstyleliteralemphasis{\sphinxupquote{, }}\sphinxstyleliteralemphasis{\sphinxupquote{optional}}\sphinxstyleliteralemphasis{\sphinxupquote{, }}\sphinxstyleliteralemphasis{\sphinxupquote{default=50}}) \textendash{} number of of legendre polynomials to use

\end{itemize}

\item[{Returns}] \leavevmode
\sphinxstylestrong{potential} \textendash{} values of the electric potential, in V

\item[{Return type}] \leavevmode
{[}N x 1{]} np.ndarray

\end{description}\end{quote}

\end{fulllineitems}

\index{calc\_tms\_E\_field() (in module pygpc.testfun)}

\begin{fulllineitems}
\phantomsection\label{\detokenize{pygpc:pygpc.testfun.calc_tms_E_field}}\pysiglinewithargsret{\sphinxcode{\sphinxupquote{pygpc.testfun.}}\sphinxbfcode{\sphinxupquote{calc\_tms\_E\_field}}}{\emph{dipole\_pos}, \emph{dipole\_moment}, \emph{didt}, \emph{positions}}{}
Calculate the E field in a sphere caused by external magnetic dipoles.
Dipole in SI units, positions in mm
Everything should be in SI
Independent of conductivity, see references
\begin{enumerate}
\def\theenumi{\Alph{enumi}}
\def\labelenumi{\theenumi .}
\makeatletter\def\p@enumii{\p@enumi \theenumi .}\makeatother
\setcounter{enumi}{11}
\item {} 
Heller and D. van Hulsteyn, Brain stimulation using electromagnetic sources: theoretical aspects

\end{enumerate}

E = calc\_tms\_E\_field(dipole\_pos, dipole\_moment, didt, positions)
\begin{quote}\begin{description}
\item[{Parameters}] \leavevmode\begin{itemize}
\item {} 
\sphinxstyleliteralstrong{\sphinxupquote{dipole\_pos}} (\sphinxstyleliteralemphasis{\sphinxupquote{{[}}}\sphinxstyleliteralemphasis{\sphinxupquote{M x 3}}\sphinxstyleliteralemphasis{\sphinxupquote{{]} }}\sphinxstyleliteralemphasis{\sphinxupquote{np.ndarray}}) \textendash{} position of dipoles, must be outside sphere

\item {} 
\sphinxstyleliteralstrong{\sphinxupquote{dipole\_moment}} (\sphinxstyleliteralemphasis{\sphinxupquote{{[}}}\sphinxstyleliteralemphasis{\sphinxupquote{m x 3}}\sphinxstyleliteralemphasis{\sphinxupquote{{]} }}\sphinxstyleliteralemphasis{\sphinxupquote{np.ndarray}}) \textendash{} moment of dipoles

\item {} 
\sphinxstyleliteralstrong{\sphinxupquote{didt}} (\sphinxstyleliteralemphasis{\sphinxupquote{float}}) \textendash{} variation rate of current in the coil

\item {} 
\sphinxstyleliteralstrong{\sphinxupquote{positions}} (\sphinxstyleliteralemphasis{\sphinxupquote{{[}}}\sphinxstyleliteralemphasis{\sphinxupquote{N x 3}}\sphinxstyleliteralemphasis{\sphinxupquote{{]} }}\sphinxstyleliteralemphasis{\sphinxupquote{np.ndarray}}) \textendash{} position where fields should be calculated, must lie inside sphere

\end{itemize}

\item[{Returns}] \leavevmode
\sphinxstylestrong{E} \textendash{} array with E-fields at detector positions

\item[{Return type}] \leavevmode
{[}N x 3{]} np.ndarray

\end{description}\end{quote}

\end{fulllineitems}

\index{g\_function() (in module pygpc.testfun)}

\begin{fulllineitems}
\phantomsection\label{\detokenize{pygpc:pygpc.testfun.g_function}}\pysiglinewithargsret{\sphinxcode{\sphinxupquote{pygpc.testfun.}}\sphinxbfcode{\sphinxupquote{g\_function}}}{\emph{x}, \emph{a}}{}
N-dimensional g-function used by Saltelli and Sobol

this test function is used as an integrand for various numerical
estimation methods, including sensitivity analysis methods, because it
is fairly complex, and its sensitivity indices can be expressed
analytically. The exact value of the integral with this function as an
integrand is 1.

Saltelli, Andrea; Sobol, I. M. (1995): Sensitivity analysis for nonlinear
mathematical models: numerical experience. In: Mathematical models and
computer experiment 7 (11), S. 16-28.

y = g\_function(x, a)
\begin{quote}\begin{description}
\item[{Parameters}] \leavevmode\begin{itemize}
\item {} 
\sphinxstyleliteralstrong{\sphinxupquote{x}} (\sphinxstyleliteralemphasis{\sphinxupquote{{[}}}\sphinxstyleliteralemphasis{\sphinxupquote{N\_input x N\_dims}}\sphinxstyleliteralemphasis{\sphinxupquote{{]} }}\sphinxstyleliteralemphasis{\sphinxupquote{np.ndarray}}) \textendash{} input data

\item {} 
\sphinxstyleliteralstrong{\sphinxupquote{a}} (\sphinxstyleliteralemphasis{\sphinxupquote{{[}}}\sphinxstyleliteralemphasis{\sphinxupquote{N\_dims}}\sphinxstyleliteralemphasis{\sphinxupquote{{]} }}\sphinxstyleliteralemphasis{\sphinxupquote{np.ndarray}}) \textendash{} importance factor of dimensions

\end{itemize}

\item[{Returns}] \leavevmode
\sphinxstylestrong{y} \textendash{} output data

\item[{Return type}] \leavevmode
{[}N\_input x 1{]} np.ndarray

\end{description}\end{quote}

\end{fulllineitems}

\index{ishigami() (in module pygpc.testfun)}

\begin{fulllineitems}
\phantomsection\label{\detokenize{pygpc:pygpc.testfun.ishigami}}\pysiglinewithargsret{\sphinxcode{\sphinxupquote{pygpc.testfun.}}\sphinxbfcode{\sphinxupquote{ishigami}}}{\emph{x}, \emph{a}, \emph{b}}{}
Three-dimensional test function of Ishigami

The Ishigami function of Ishigami \& Homma (1990) is used as an example
for uncertainty and sensitivity analysis methods, because it exhibits
strong nonlinearity and nonmonotonicity. It also has a peculiar
dependence on x3, as described by Sobol’ \& Levitan (1999).

Ishigami, T., \& Homma, T. (1990, December). An importance quantification
technique in uncertainty analysis for computer models. In Uncertainty
Modeling and Analysis, 1990. Proceedings., First International Symposium
on (pp. 398-403). IEEE.

Sobol’, I. M., \& Levitan, Y. L. (1999). On the use of variance reducing
multipliers in Monte Carlo computations of a global sensitivity index.
Computer Physics Communications, 117(1), 52-61.

f(x) = sin(x1) + a*sin(x2)\textasciicircum{}2 + b*x3\textasciicircum{}4*sin(x1)

y = ishigami(x,a,b)
\begin{quote}\begin{description}
\item[{Parameters}] \leavevmode\begin{itemize}
\item {} 
\sphinxstyleliteralstrong{\sphinxupquote{x}} (\sphinxstyleliteralemphasis{\sphinxupquote{{[}}}\sphinxstyleliteralemphasis{\sphinxupquote{N x 3}}\sphinxstyleliteralemphasis{\sphinxupquote{{]} }}\sphinxstyleliteralemphasis{\sphinxupquote{np.ndarray}}) \textendash{} input data
xi \textasciitilde{} Uniform{[}-pi, pi{]}, for all i = 1, 2, 3

\item {} 
\sphinxstyleliteralstrong{\sphinxupquote{a}} (\sphinxstyleliteralemphasis{\sphinxupquote{float}}) \textendash{} shape parameter

\item {} 
\sphinxstyleliteralstrong{\sphinxupquote{b}} (\sphinxstyleliteralemphasis{\sphinxupquote{float}}) \textendash{} shape parameter

\end{itemize}

\item[{Returns}] \leavevmode
\sphinxstylestrong{y} \textendash{} output data

\item[{Return type}] \leavevmode
{[}N x 1{]} np.ndarray

\end{description}\end{quote}

\end{fulllineitems}

\index{lim\_2002() (in module pygpc.testfun)}

\begin{fulllineitems}
\phantomsection\label{\detokenize{pygpc:pygpc.testfun.lim_2002}}\pysiglinewithargsret{\sphinxcode{\sphinxupquote{pygpc.testfun.}}\sphinxbfcode{\sphinxupquote{lim\_2002}}}{\emph{x}}{}
Two-dimensional test function of Lim et al

This function is a polynomial in two dimensions, with terms up to degree
5. It is nonlinear, and it is smooth despite being complex, which is
common for computer experiment functions (Lim et al., 2002).

Lim, Y. B., Sacks, J., Studden, W. J., \& Welch, W. J. (2002). Design
and analysis of computer experiments when the output is highly correlated
over the input space. Canadian Journal of Statistics, 30(1), 109-126.
\begin{description}
\item[{f(x) = 9 + 5/2*x1 - 35/2*x2 + 5/2*x1*x2 + 19*x2\textasciicircum{}2 - 15/2*x1\textasciicircum{}3}] \leavevmode\begin{itemize}
\item {} 
5/2*x1*x2\textasciicircum{}2 - 11/2*x2\textasciicircum{}4 + x1\textasciicircum{}3*x2\textasciicircum{}2

\end{itemize}

\end{description}

y = lim\_2002(x)
\begin{quote}\begin{description}
\item[{Parameters}] \leavevmode
\sphinxstyleliteralstrong{\sphinxupquote{x}} (\sphinxstyleliteralemphasis{\sphinxupquote{{[}}}\sphinxstyleliteralemphasis{\sphinxupquote{N x 2}}\sphinxstyleliteralemphasis{\sphinxupquote{{]} }}\sphinxstyleliteralemphasis{\sphinxupquote{np.ndarray}}) \textendash{} input data
xi is element of {[}0, 1{]}, for all i = 1, 2

\item[{Returns}] \leavevmode
\sphinxstylestrong{y} \textendash{} output data

\item[{Return type}] \leavevmode
{[}N x 1{]} np.ndarray

\end{description}\end{quote}

\end{fulllineitems}

\index{oakley\_ohagan\_2004() (in module pygpc.testfun)}

\begin{fulllineitems}
\phantomsection\label{\detokenize{pygpc:pygpc.testfun.oakley_ohagan_2004}}\pysiglinewithargsret{\sphinxcode{\sphinxupquote{pygpc.testfun.}}\sphinxbfcode{\sphinxupquote{oakley\_ohagan\_2004}}}{\emph{x}}{}
15-dimensional test function of OAKLEY \& O’HAGAN (2004)

This function’s a-coefficients are chosen so that 5 of the input
variables contribute significantly to the output variance, 5 have a
much smaller effect, and the remaining 5 have almost no effect on the
output variance.

Oakley, J. E., \& O’Hagan, A. (2004). Probabilistic sensitivity analysis
of complex models: a Bayesian approach. Journal of the Royal Statistical
Society: Series B (Statistical Methodology), 66(3), 751-769.

y = oakley\_ohagan\_2004(x)
\begin{quote}\begin{description}
\item[{Parameters}] \leavevmode
\sphinxstyleliteralstrong{\sphinxupquote{x}} (\sphinxstyleliteralemphasis{\sphinxupquote{{[}}}\sphinxstyleliteralemphasis{\sphinxupquote{N\_input x 15}}\sphinxstyleliteralemphasis{\sphinxupquote{{]} }}\sphinxstyleliteralemphasis{\sphinxupquote{np.ndarray}}) \textendash{} input data
xi \textasciitilde{} N(mu=0, sigma=1), for all i = 1, 2,…, 15.

\item[{Returns}] \leavevmode
\sphinxstylestrong{y} \textendash{} output data

\item[{Return type}] \leavevmode
{[}N\_input x 1{]} np.ndarray

\end{description}\end{quote}

\end{fulllineitems}

\index{peaks() (in module pygpc.testfun)}

\begin{fulllineitems}
\phantomsection\label{\detokenize{pygpc:pygpc.testfun.peaks}}\pysiglinewithargsret{\sphinxcode{\sphinxupquote{pygpc.testfun.}}\sphinxbfcode{\sphinxupquote{peaks}}}{\emph{x}}{}
Two-dimensional peaks function.

y = peaks(x)
\begin{quote}\begin{description}
\item[{Parameters}] \leavevmode
\sphinxstyleliteralstrong{\sphinxupquote{x}} (\sphinxstyleliteralemphasis{\sphinxupquote{{[}}}\sphinxstyleliteralemphasis{\sphinxupquote{N x 2}}\sphinxstyleliteralemphasis{\sphinxupquote{{]} }}\sphinxstyleliteralemphasis{\sphinxupquote{np.ndarray}}) \textendash{} input data

\item[{Returns}] \leavevmode
\sphinxstylestrong{y} \textendash{} output data

\item[{Return type}] \leavevmode
{[}N x 1{]} np.ndarray

\end{description}\end{quote}

\end{fulllineitems}

\index{sphere() (in module pygpc.testfun)}

\begin{fulllineitems}
\phantomsection\label{\detokenize{pygpc:pygpc.testfun.sphere}}\pysiglinewithargsret{\sphinxcode{\sphinxupquote{pygpc.testfun.}}\sphinxbfcode{\sphinxupquote{sphere}}}{\emph{x}}{}
N-dimensional sphere function.

y = sphere(x)
\begin{quote}\begin{description}
\item[{Parameters}] \leavevmode
\sphinxstyleliteralstrong{\sphinxupquote{x}} (\sphinxstyleliteralemphasis{\sphinxupquote{{[}}}\sphinxstyleliteralemphasis{\sphinxupquote{N\_input x N\_dims}}\sphinxstyleliteralemphasis{\sphinxupquote{{]} }}\sphinxstyleliteralemphasis{\sphinxupquote{np.ndarray}}) \textendash{} input data

\item[{Returns}] \leavevmode
output data

\item[{Return type}] \leavevmode
y {[}N\_input x 1{]} np.ndarray

\end{description}\end{quote}

\end{fulllineitems}

\index{sphere\_zero\_mean() (in module pygpc.testfun)}

\begin{fulllineitems}
\phantomsection\label{\detokenize{pygpc:pygpc.testfun.sphere_zero_mean}}\pysiglinewithargsret{\sphinxcode{\sphinxupquote{pygpc.testfun.}}\sphinxbfcode{\sphinxupquote{sphere\_zero\_mean}}}{\emph{x}, \emph{a}, \emph{b}}{}
N-dimensional sphere function with zero mean.

y = sphere\_zero\_mean(x,a,b)
\begin{quote}\begin{description}
\item[{Parameters}] \leavevmode\begin{itemize}
\item {} 
\sphinxstyleliteralstrong{\sphinxupquote{x}} (\sphinxstyleliteralemphasis{\sphinxupquote{{[}}}\sphinxstyleliteralemphasis{\sphinxupquote{N\_input x N\_dims}}\sphinxstyleliteralemphasis{\sphinxupquote{{]} }}\sphinxstyleliteralemphasis{\sphinxupquote{np.ndarray}}) \textendash{} input data

\item {} 
\sphinxstyleliteralstrong{\sphinxupquote{a}} (\sphinxstyleliteralemphasis{\sphinxupquote{{[}}}\sphinxstyleliteralemphasis{\sphinxupquote{N\_dims}}\sphinxstyleliteralemphasis{\sphinxupquote{{]} }}\sphinxstyleliteralemphasis{\sphinxupquote{np.ndarray}}) \textendash{} lower bound of input data

\item {} 
\sphinxstyleliteralstrong{\sphinxupquote{b}} (\sphinxstyleliteralemphasis{\sphinxupquote{{[}}}\sphinxstyleliteralemphasis{\sphinxupquote{N\_dims}}\sphinxstyleliteralemphasis{\sphinxupquote{{]} }}\sphinxstyleliteralemphasis{\sphinxupquote{np.ndarray}}) \textendash{} upper bound of input data

\end{itemize}

\item[{Returns}] \leavevmode
\sphinxstylestrong{y} \textendash{} output data

\item[{Return type}] \leavevmode
{[}N\_input{]} np.ndarray

\end{description}\end{quote}

\end{fulllineitems}

\index{welch\_1992() (in module pygpc.testfun)}

\begin{fulllineitems}
\phantomsection\label{\detokenize{pygpc:pygpc.testfun.welch_1992}}\pysiglinewithargsret{\sphinxcode{\sphinxupquote{pygpc.testfun.}}\sphinxbfcode{\sphinxupquote{welch\_1992}}}{\emph{x}}{}
20-dimensional test function of WELCH (1992)

For input variable screening purposes, it can be found that some input
variables of this function have a very high effect on the output,
compared to other input variables. As Welch et al. (1992) point out,
interactions and nonlinear effects make this function challenging.

Welch, W. J., Buck, R. J., Sacks, J., Wynn, H. P., Mitchell, T. J., \&
Morris, M. D. (1992). Screening, predicting, and computer experiments.
Technometrics, 34(1), 15-25.

y = welch\_1992(x)
\begin{quote}\begin{description}
\item[{Parameters}] \leavevmode
\sphinxstyleliteralstrong{\sphinxupquote{x}} (\sphinxstyleliteralemphasis{\sphinxupquote{{[}}}\sphinxstyleliteralemphasis{\sphinxupquote{N\_input x 20}}\sphinxstyleliteralemphasis{\sphinxupquote{{]} }}\sphinxstyleliteralemphasis{\sphinxupquote{np.ndarray}}) \textendash{} input data
xi \textasciitilde{} U(-0.5, 0.5), for all i = 1,…, 20.

\item[{Returns}] \leavevmode
\sphinxstylestrong{y} \textendash{} output data

\item[{Return type}] \leavevmode
{[}N\_input x 1{]} np.ndarray

\end{description}\end{quote}

\end{fulllineitems}

\index{wing\_weight() (in module pygpc.testfun)}

\begin{fulllineitems}
\phantomsection\label{\detokenize{pygpc:pygpc.testfun.wing_weight}}\pysiglinewithargsret{\sphinxcode{\sphinxupquote{pygpc.testfun.}}\sphinxbfcode{\sphinxupquote{wing\_weight}}}{\emph{x}}{}
10-dimensional test function which models a light aircraft wing

Forrester, A., Sobester, A., \& Keane, A. (2008). Engineering design via
surrogate modelling: a practical guide. Wiley.

y  = wing\_weight(x)
\begin{quote}\begin{description}
\item[{Parameters}] \leavevmode
\sphinxstyleliteralstrong{\sphinxupquote{x}} (\sphinxstyleliteralemphasis{\sphinxupquote{{[}}}\sphinxstyleliteralemphasis{\sphinxupquote{N\_input x 10}}\sphinxstyleliteralemphasis{\sphinxupquote{{]} }}\sphinxstyleliteralemphasis{\sphinxupquote{np.ndarray}}) \textendash{} input data
x1(Sw)  is element of {[}150, 200{]}
x2(Wfw) is element of {[}220, 300{]}
x3(A)   is element of {[}6, 10{]}
x4(Lambda)   is element of {[}-10, 10{]}
x5(q)   is element of {[}16, 45{]}
x6(lambda)   is element of {[}0.5, 1{]}
x7(tc)  is element of {[}0.08, 0.18{]}
x8(Nz)  is element of {[}2.5, 6{]}
x9(Wdg) is element of {[}1700, 2500{]}
x10(Wp) is element of {[}0.025, 0.08{]}

\item[{Returns}] \leavevmode
\sphinxstylestrong{y} \textendash{} output data

\item[{Return type}] \leavevmode
{[}N\_input x 1{]} np.ndarray

\end{description}\end{quote}

\end{fulllineitems}



\section{pygpc.vis module}
\label{\detokenize{pygpc:module-pygpc.vis}}\label{\detokenize{pygpc:pygpc-vis-module}}\index{pygpc.vis (module)}
Functions and classes that provide visualisation functionalities
\index{Visualization (class in pygpc.vis)}

\begin{fulllineitems}
\phantomsection\label{\detokenize{pygpc:pygpc.vis.Visualization}}\pysiglinewithargsret{\sphinxbfcode{\sphinxupquote{class }}\sphinxcode{\sphinxupquote{pygpc.vis.}}\sphinxbfcode{\sphinxupquote{Visualization}}}{\emph{dims=(10}, \emph{10)}}{}
Creates a new visualization in a new window. Any added subcharts will be added to this window.

Visualisation(dims=(10, 10))
\index{figure\_number (pygpc.vis.Visualization.Visualisation attribute)}

\begin{fulllineitems}
\phantomsection\label{\detokenize{pygpc:pygpc.vis.Visualization.Visualisation.figure_number}}\pysigline{\sphinxcode{\sphinxupquote{Visualisation.}}\sphinxbfcode{\sphinxupquote{figure\_number}}}
number of figures that have been created
\begin{quote}\begin{description}
\item[{Type}] \leavevmode
int, begin=0

\end{description}\end{quote}

\end{fulllineitems}

\index{horizontal\_padding (pygpc.vis.Visualization.Visualisation attribute)}

\begin{fulllineitems}
\phantomsection\label{\detokenize{pygpc:pygpc.vis.Visualization.Visualisation.horizontal_padding}}\pysigline{\sphinxcode{\sphinxupquote{Visualisation.}}\sphinxbfcode{\sphinxupquote{horizontal\_padding}}}
horizontal padding of plot
\begin{quote}\begin{description}
\item[{Type}] \leavevmode
float, default=0.4

\end{description}\end{quote}

\end{fulllineitems}

\index{font\_size\_label (pygpc.vis.Visualization.Visualisation attribute)}

\begin{fulllineitems}
\phantomsection\label{\detokenize{pygpc:pygpc.vis.Visualization.Visualisation.font_size_label}}\pysigline{\sphinxcode{\sphinxupquote{Visualisation.}}\sphinxbfcode{\sphinxupquote{font\_size\_label}}}
font size of title
\begin{quote}\begin{description}
\item[{Type}] \leavevmode
int, default=12

\end{description}\end{quote}

\end{fulllineitems}

\index{font\_size\_label (pygpc.vis.Visualization.Visualisation attribute)}

\begin{fulllineitems}
\pysigline{\sphinxcode{\sphinxupquote{Visualisation.}}\sphinxbfcode{\sphinxupquote{font\_size\_label}}}
font size of label
\begin{quote}\begin{description}
\item[{Type}] \leavevmode
int, default=12

\end{description}\end{quote}

\end{fulllineitems}

\index{graph\_lind\_width (pygpc.vis.Visualization.Visualisation attribute)}

\begin{fulllineitems}
\phantomsection\label{\detokenize{pygpc:pygpc.vis.Visualization.Visualisation.graph_lind_width}}\pysigline{\sphinxcode{\sphinxupquote{Visualisation.}}\sphinxbfcode{\sphinxupquote{graph\_lind\_width}}}
line width of graph
\begin{quote}\begin{description}
\item[{Type}] \leavevmode
int, default 2

\end{description}\end{quote}

\end{fulllineitems}

\index{fig (pygpc.vis.Visualization attribute)}

\begin{fulllineitems}
\phantomsection\label{\detokenize{pygpc:pygpc.vis.Visualization.fig}}\pysigline{\sphinxbfcode{\sphinxupquote{fig}}}
handle of figure created by matplotlib.pyplot
\begin{quote}\begin{description}
\item[{Type}] \leavevmode
mpl.figure

\end{description}\end{quote}

\end{fulllineitems}

\begin{quote}\begin{description}
\item[{Parameters}] \leavevmode
\sphinxstyleliteralstrong{\sphinxupquote{dims}} (\sphinxstyleliteralemphasis{\sphinxupquote{list of int}}\sphinxstyleliteralemphasis{\sphinxupquote{, }}\sphinxstyleliteralemphasis{\sphinxupquote{optional}}\sphinxstyleliteralemphasis{\sphinxupquote{, }}\sphinxstyleliteralemphasis{\sphinxupquote{default=}}\sphinxstyleliteralemphasis{\sphinxupquote{(}}\sphinxstyleliteralemphasis{\sphinxupquote{10}}\sphinxstyleliteralemphasis{\sphinxupquote{,}}\sphinxstyleliteralemphasis{\sphinxupquote{10}}\sphinxstyleliteralemphasis{\sphinxupquote{)}}) \textendash{} size of the newly created window

\end{description}\end{quote}
\index{add\_heat\_map() (pygpc.vis.Visualization method)}

\begin{fulllineitems}
\phantomsection\label{\detokenize{pygpc:pygpc.vis.Visualization.add_heat_map}}\pysiglinewithargsret{\sphinxbfcode{\sphinxupquote{add\_heat\_map}}}{\emph{title}, \emph{labels}, \emph{grid\_points}, \emph{data\_points}, \emph{v\_lim=(None}, \emph{None)}, \emph{x\_lim=None}, \emph{y\_lim=None}, \emph{colormap=None}}{}
Draw a 2D heatmap into the current figure.

add\_heat\_map(title, labels, grid\_points, data\_points, v\_lim=(None, None), x\_lim=None, y\_lim=None, colormap=None)
\begin{quote}\begin{description}
\item[{Parameters}] \leavevmode\begin{itemize}
\item {} 
\sphinxstyleliteralstrong{\sphinxupquote{title}} (\sphinxstyleliteralemphasis{\sphinxupquote{str}}) \textendash{} title of the graph

\item {} 
\sphinxstyleliteralstrong{\sphinxupquote{labels}} (\sphinxstyleliteralemphasis{\sphinxupquote{\{str:str\} dict}}) \textendash{} \{‘x’: name of x-axis, ‘y’: name of y-axis\}

\item {} 
\sphinxstyleliteralstrong{\sphinxupquote{grid\_points}} (\sphinxstyleliteralemphasis{\sphinxupquote{{[}}}\sphinxstyleliteralemphasis{\sphinxupquote{2}}\sphinxstyleliteralemphasis{\sphinxupquote{{]} }}\sphinxstyleliteralemphasis{\sphinxupquote{list of np.ndarray}}) \textendash{} arrays of the x and y positions of the grid points

\item {} 
\sphinxstyleliteralstrong{\sphinxupquote{data\_points}} (\sphinxstyleliteralemphasis{\sphinxupquote{np.ndarray of the data points that are placed into the grid}}) \textendash{} 

\item {} 
\sphinxstyleliteralstrong{\sphinxupquote{x\_lim}} (\sphinxstyleliteralemphasis{\sphinxupquote{{[}}}\sphinxstyleliteralemphasis{\sphinxupquote{2}}\sphinxstyleliteralemphasis{\sphinxupquote{{]} }}\sphinxstyleliteralemphasis{\sphinxupquote{list of float}}\sphinxstyleliteralemphasis{\sphinxupquote{, }}\sphinxstyleliteralemphasis{\sphinxupquote{optional}}\sphinxstyleliteralemphasis{\sphinxupquote{, }}\sphinxstyleliteralemphasis{\sphinxupquote{default=None}}) \textendash{} x limits for the function argument or value

\item {} 
\sphinxstyleliteralstrong{\sphinxupquote{y\_lim}} (\sphinxstyleliteralemphasis{\sphinxupquote{{[}}}\sphinxstyleliteralemphasis{\sphinxupquote{2}}\sphinxstyleliteralemphasis{\sphinxupquote{{]} }}\sphinxstyleliteralemphasis{\sphinxupquote{list of float}}\sphinxstyleliteralemphasis{\sphinxupquote{, }}\sphinxstyleliteralemphasis{\sphinxupquote{optional}}\sphinxstyleliteralemphasis{\sphinxupquote{, }}\sphinxstyleliteralemphasis{\sphinxupquote{default=None}}) \textendash{} y limits for the function argument or value

\item {} 
\sphinxstyleliteralstrong{\sphinxupquote{v\_lim}} (\sphinxstyleliteralemphasis{\sphinxupquote{{[}}}\sphinxstyleliteralemphasis{\sphinxupquote{2}}\sphinxstyleliteralemphasis{\sphinxupquote{{]} }}\sphinxstyleliteralemphasis{\sphinxupquote{list of float}}\sphinxstyleliteralemphasis{\sphinxupquote{, }}\sphinxstyleliteralemphasis{\sphinxupquote{optional}}\sphinxstyleliteralemphasis{\sphinxupquote{, }}\sphinxstyleliteralemphasis{\sphinxupquote{default=}}\sphinxstyleliteralemphasis{\sphinxupquote{(}}\sphinxstyleliteralemphasis{\sphinxupquote{None}}\sphinxstyleliteralemphasis{\sphinxupquote{,}}\sphinxstyleliteralemphasis{\sphinxupquote{None}}\sphinxstyleliteralemphasis{\sphinxupquote{)}}) \textendash{} limits of the color scale

\item {} 
\sphinxstyleliteralstrong{\sphinxupquote{colormap}} (\sphinxstyleliteralemphasis{\sphinxupquote{str}}\sphinxstyleliteralemphasis{\sphinxupquote{, }}\sphinxstyleliteralemphasis{\sphinxupquote{optional}}\sphinxstyleliteralemphasis{\sphinxupquote{, }}\sphinxstyleliteralemphasis{\sphinxupquote{default=None}}) \textendash{} the colormap to use

\end{itemize}

\end{description}\end{quote}

\end{fulllineitems}

\index{add\_line\_plot() (pygpc.vis.Visualization method)}

\begin{fulllineitems}
\phantomsection\label{\detokenize{pygpc:pygpc.vis.Visualization.add_line_plot}}\pysiglinewithargsret{\sphinxbfcode{\sphinxupquote{add\_line\_plot}}}{\emph{title}, \emph{labels}, \emph{data}, \emph{x\_lim=None}, \emph{y\_lim=None}}{}
Draw a 1D line graph into the current figure.

add\_line\_plot(title, labels, data, x\_lim=None, y\_lim=None)
\begin{quote}\begin{description}
\item[{Parameters}] \leavevmode\begin{itemize}
\item {} 
\sphinxstyleliteralstrong{\sphinxupquote{title}} (\sphinxstyleliteralemphasis{\sphinxupquote{str}}) \textendash{} title of the graph

\item {} 
\sphinxstyleliteralstrong{\sphinxupquote{labels}} (\sphinxstyleliteralemphasis{\sphinxupquote{\{str:str\} dict}}) \textendash{} \{‘x’: name of x-axis, ‘y’: name of y-axis\}

\item {} 
\sphinxstyleliteralstrong{\sphinxupquote{x\_lim}} (\sphinxstyleliteralemphasis{\sphinxupquote{{[}}}\sphinxstyleliteralemphasis{\sphinxupquote{2}}\sphinxstyleliteralemphasis{\sphinxupquote{{]} }}\sphinxstyleliteralemphasis{\sphinxupquote{list of float}}\sphinxstyleliteralemphasis{\sphinxupquote{, }}\sphinxstyleliteralemphasis{\sphinxupquote{optional}}\sphinxstyleliteralemphasis{\sphinxupquote{, }}\sphinxstyleliteralemphasis{\sphinxupquote{default=None}}) \textendash{} x limits for the function argument or value

\item {} 
\sphinxstyleliteralstrong{\sphinxupquote{y\_lim}} (\sphinxstyleliteralemphasis{\sphinxupquote{{[}}}\sphinxstyleliteralemphasis{\sphinxupquote{2}}\sphinxstyleliteralemphasis{\sphinxupquote{{]} }}\sphinxstyleliteralemphasis{\sphinxupquote{list of float}}\sphinxstyleliteralemphasis{\sphinxupquote{, }}\sphinxstyleliteralemphasis{\sphinxupquote{optional}}\sphinxstyleliteralemphasis{\sphinxupquote{, }}\sphinxstyleliteralemphasis{\sphinxupquote{default=None}}) \textendash{} y limits for the function argument or value

\item {} 
\sphinxstyleliteralstrong{\sphinxupquote{data}} (\sphinxstyleliteralemphasis{\sphinxupquote{np.ndarray}}) \textendash{} data that should be plotted

\end{itemize}

\end{description}\end{quote}

\end{fulllineitems}

\index{add\_scatter\_plot() (pygpc.vis.Visualization static method)}

\begin{fulllineitems}
\phantomsection\label{\detokenize{pygpc:pygpc.vis.Visualization.add_scatter_plot}}\pysiglinewithargsret{\sphinxbfcode{\sphinxupquote{static }}\sphinxbfcode{\sphinxupquote{add\_scatter\_plot}}}{\emph{shape}, \emph{plot\_size}, \emph{color\_sequence}, \emph{colormap=None}, \emph{v\_lim=(None}, \emph{None)}}{}
Draw a scatter plot onto the current chart.

add\_scatter\_plot(shape, plot\_size, color\_sequence, colormap=None, v\_lim=(None, None))
\begin{quote}\begin{description}
\item[{Parameters}] \leavevmode\begin{itemize}
\item {} 
\sphinxstyleliteralstrong{\sphinxupquote{shape}} (\sphinxstyleliteralemphasis{\sphinxupquote{\{str: np.ndarray\} dict}}) \textendash{} \{‘x’: positions on x-axis, ‘y’: positions on y-axis\}

\item {} 
\sphinxstyleliteralstrong{\sphinxupquote{plot\_size}} (\sphinxstyleliteralemphasis{\sphinxupquote{np.ndarray}}) \textendash{} the marker size in the squared number of points

\item {} 
\sphinxstyleliteralstrong{\sphinxupquote{color\_sequence}} (\sphinxstyleliteralemphasis{\sphinxupquote{str}}\sphinxstyleliteralemphasis{\sphinxupquote{ or }}\sphinxstyleliteralemphasis{\sphinxupquote{list}}) \textendash{} marker colors

\item {} 
\sphinxstyleliteralstrong{\sphinxupquote{colormap}} (\sphinxstyleliteralemphasis{\sphinxupquote{str}}\sphinxstyleliteralemphasis{\sphinxupquote{, }}\sphinxstyleliteralemphasis{\sphinxupquote{optional}}\sphinxstyleliteralemphasis{\sphinxupquote{, }}\sphinxstyleliteralemphasis{\sphinxupquote{default=None}}) \textendash{} the colormap to use

\item {} 
\sphinxstyleliteralstrong{\sphinxupquote{v\_lim}} (\sphinxstyleliteralemphasis{\sphinxupquote{{[}}}\sphinxstyleliteralemphasis{\sphinxupquote{2}}\sphinxstyleliteralemphasis{\sphinxupquote{{]} }}\sphinxstyleliteralemphasis{\sphinxupquote{list of float}}\sphinxstyleliteralemphasis{\sphinxupquote{, }}\sphinxstyleliteralemphasis{\sphinxupquote{optional}}\sphinxstyleliteralemphasis{\sphinxupquote{, }}\sphinxstyleliteralemphasis{\sphinxupquote{default=}}\sphinxstyleliteralemphasis{\sphinxupquote{(}}\sphinxstyleliteralemphasis{\sphinxupquote{None}}\sphinxstyleliteralemphasis{\sphinxupquote{,}}\sphinxstyleliteralemphasis{\sphinxupquote{None}}\sphinxstyleliteralemphasis{\sphinxupquote{)}}) \textendash{} limits of the color scale

\end{itemize}

\end{description}\end{quote}

\end{fulllineitems}

\index{create\_new\_chart() (pygpc.vis.Visualization method)}

\begin{fulllineitems}
\phantomsection\label{\detokenize{pygpc:pygpc.vis.Visualization.create_new_chart}}\pysiglinewithargsret{\sphinxbfcode{\sphinxupquote{create\_new\_chart}}}{\emph{layout\_id=None}}{}
Add a new subplot to the current visualization, so that multiple graphs can be overlaid onto one chart
(e.g. scatterplot over heatmap).

create\_new\_chart(layout\_id=None)
\begin{quote}\begin{description}
\item[{Parameters}] \leavevmode
\sphinxstyleliteralstrong{\sphinxupquote{layout\_id}} (\sphinxstyleliteralemphasis{\sphinxupquote{(}}\sphinxstyleliteralemphasis{\sphinxupquote{3-digit}}\sphinxstyleliteralemphasis{\sphinxupquote{) }}\sphinxstyleliteralemphasis{\sphinxupquote{int}}\sphinxstyleliteralemphasis{\sphinxupquote{, }}\sphinxstyleliteralemphasis{\sphinxupquote{optional}}\sphinxstyleliteralemphasis{\sphinxupquote{, }}\sphinxstyleliteralemphasis{\sphinxupquote{default=None}}) \textendash{} denoting the position of the graph in figure (xyn : ‘x’=width, ‘y’=height of grid, ‘n’=position within grid)

\end{description}\end{quote}

\end{fulllineitems}

\index{create\_sub\_plot() (pygpc.vis.Visualization static method)}

\begin{fulllineitems}
\phantomsection\label{\detokenize{pygpc:pygpc.vis.Visualization.create_sub_plot}}\pysiglinewithargsret{\sphinxbfcode{\sphinxupquote{static }}\sphinxbfcode{\sphinxupquote{create\_sub\_plot}}}{\emph{title}, \emph{labels}, \emph{x\_lim}, \emph{y\_lim}}{}
Set the title, labels and the axis limits of a plot.

create\_sub\_plot(title, labels, x\_lim, y\_lim)
\begin{quote}\begin{description}
\item[{Parameters}] \leavevmode\begin{itemize}
\item {} 
\sphinxstyleliteralstrong{\sphinxupquote{title}} (\sphinxstyleliteralemphasis{\sphinxupquote{str}}) \textendash{} title of the plot

\item {} 
\sphinxstyleliteralstrong{\sphinxupquote{labels}} (\sphinxstyleliteralemphasis{\sphinxupquote{\{str:str\} dict}}) \textendash{} \{‘x’: name of x-axis, ‘y’: name of y-axis\}

\item {} 
\sphinxstyleliteralstrong{\sphinxupquote{x\_lim}} (\sphinxstyleliteralemphasis{\sphinxupquote{{[}}}\sphinxstyleliteralemphasis{\sphinxupquote{2}}\sphinxstyleliteralemphasis{\sphinxupquote{{]} }}\sphinxstyleliteralemphasis{\sphinxupquote{list of float}}) \textendash{} x limits for the function argument or value

\item {} 
\sphinxstyleliteralstrong{\sphinxupquote{y\_lim}} (\sphinxstyleliteralemphasis{\sphinxupquote{{[}}}\sphinxstyleliteralemphasis{\sphinxupquote{2}}\sphinxstyleliteralemphasis{\sphinxupquote{{]} }}\sphinxstyleliteralemphasis{\sphinxupquote{list of float}}) \textendash{} y limits for the function argument or value

\end{itemize}

\end{description}\end{quote}

\end{fulllineitems}

\index{figure\_number (pygpc.vis.Visualization attribute)}

\begin{fulllineitems}
\phantomsection\label{\detokenize{pygpc:pygpc.vis.Visualization.figure_number}}\pysigline{\sphinxbfcode{\sphinxupquote{figure\_number}}\sphinxbfcode{\sphinxupquote{ = 0}}}
\end{fulllineitems}

\index{font\_size\_label (pygpc.vis.Visualization attribute)}

\begin{fulllineitems}
\phantomsection\label{\detokenize{pygpc:pygpc.vis.Visualization.font_size_label}}\pysigline{\sphinxbfcode{\sphinxupquote{font\_size\_label}}\sphinxbfcode{\sphinxupquote{ = 12}}}
\end{fulllineitems}

\index{font\_size\_title (pygpc.vis.Visualization attribute)}

\begin{fulllineitems}
\phantomsection\label{\detokenize{pygpc:pygpc.vis.Visualization.font_size_title}}\pysigline{\sphinxbfcode{\sphinxupquote{font\_size\_title}}\sphinxbfcode{\sphinxupquote{ = 12}}}
\end{fulllineitems}

\index{graph\_line\_width (pygpc.vis.Visualization attribute)}

\begin{fulllineitems}
\phantomsection\label{\detokenize{pygpc:pygpc.vis.Visualization.graph_line_width}}\pysigline{\sphinxbfcode{\sphinxupquote{graph\_line\_width}}\sphinxbfcode{\sphinxupquote{ = 2}}}
\end{fulllineitems}

\index{horizontal\_padding (pygpc.vis.Visualization attribute)}

\begin{fulllineitems}
\phantomsection\label{\detokenize{pygpc:pygpc.vis.Visualization.horizontal_padding}}\pysigline{\sphinxbfcode{\sphinxupquote{horizontal\_padding}}\sphinxbfcode{\sphinxupquote{ = 0.4}}}
\end{fulllineitems}

\index{show() (pygpc.vis.Visualization static method)}

\begin{fulllineitems}
\phantomsection\label{\detokenize{pygpc:pygpc.vis.Visualization.show}}\pysiglinewithargsret{\sphinxbfcode{\sphinxupquote{static }}\sphinxbfcode{\sphinxupquote{show}}}{}{}
Show plots.

\end{fulllineitems}


\end{fulllineitems}

\index{plot\_sobol\_indices() (in module pygpc.vis)}

\begin{fulllineitems}
\phantomsection\label{\detokenize{pygpc:pygpc.vis.plot_sobol_indices}}\pysiglinewithargsret{\sphinxcode{\sphinxupquote{pygpc.vis.}}\sphinxbfcode{\sphinxupquote{plot\_sobol\_indices}}}{\emph{sobol\_rel\_order\_mean}, \emph{sobol\_rel\_1st\_order\_mean}, \emph{fn\_plot}, \emph{random\_vars}}{}
Plot the Sobol indices into different sub-plots.

plot\_sobol\_indices(sobol\_rel\_order\_mean, sobol\_rel\_1st\_order\_mean, fn\_plot, random\_vars)
\begin{quote}\begin{description}
\item[{Parameters}] \leavevmode\begin{itemize}
\item {} 
\sphinxstyleliteralstrong{\sphinxupquote{sobol\_rel\_order\_mean}} (\sphinxstyleliteralemphasis{\sphinxupquote{np.ndarray}}) \textendash{} average proportion of the Sobol indices of the different order to the total variance (1st, 2nd, etc..,)
over all output quantities

\item {} 
\sphinxstyleliteralstrong{\sphinxupquote{sobol\_rel\_1st\_order\_mean}} (\sphinxstyleliteralemphasis{\sphinxupquote{np.ndarray}}) \textendash{} average proportion of the random variables of the 1st order Sobol indices to the total variance over all
output quantities

\item {} 
\sphinxstyleliteralstrong{\sphinxupquote{fn\_plot}} (\sphinxstyleliteralemphasis{\sphinxupquote{str}}) \textendash{} filename of plot

\item {} 
\sphinxstyleliteralstrong{\sphinxupquote{random\_vars}} (\sphinxstyleliteralemphasis{\sphinxupquote{{[}}}\sphinxstyleliteralemphasis{\sphinxupquote{dim}}\sphinxstyleliteralemphasis{\sphinxupquote{{]} }}\sphinxstyleliteralemphasis{\sphinxupquote{list of str}}) \textendash{} string labels of the random variables

\end{itemize}

\end{description}\end{quote}

\end{fulllineitems}



\chapter{Indices and tables}
\label{\detokenize{index:indices-and-tables}}


\renewcommand{\indexname}{Python Module Index}
\begin{sphinxtheindex}
\let\bigletter\sphinxstyleindexlettergroup
\bigletter{p}
\item\relax\sphinxstyleindexentry{pygpc}\sphinxstyleindexpageref{pygpc:\detokenize{module-pygpc}}
\item\relax\sphinxstyleindexentry{pygpc.gpc}\sphinxstyleindexpageref{pygpc:\detokenize{module-pygpc.gpc}}
\item\relax\sphinxstyleindexentry{pygpc.grid}\sphinxstyleindexpageref{pygpc:\detokenize{module-pygpc.grid}}
\item\relax\sphinxstyleindexentry{pygpc.misc}\sphinxstyleindexpageref{pygpc:\detokenize{module-pygpc.misc}}
\item\relax\sphinxstyleindexentry{pygpc.ni}\sphinxstyleindexpageref{pygpc:\detokenize{module-pygpc.ni}}
\item\relax\sphinxstyleindexentry{pygpc.postproc}\sphinxstyleindexpageref{pygpc:\detokenize{module-pygpc.postproc}}
\item\relax\sphinxstyleindexentry{pygpc.quad}\sphinxstyleindexpageref{pygpc:\detokenize{module-pygpc.quad}}
\item\relax\sphinxstyleindexentry{pygpc.reg}\sphinxstyleindexpageref{pygpc:\detokenize{module-pygpc.reg}}
\item\relax\sphinxstyleindexentry{pygpc.rw}\sphinxstyleindexpageref{pygpc:\detokenize{module-pygpc.rw}}
\item\relax\sphinxstyleindexentry{pygpc.testfun}\sphinxstyleindexpageref{pygpc:\detokenize{module-pygpc.testfun}}
\item\relax\sphinxstyleindexentry{pygpc.vis}\sphinxstyleindexpageref{pygpc:\detokenize{module-pygpc.vis}}
\end{sphinxtheindex}

\renewcommand{\indexname}{Index}
\printindex
\end{document}
